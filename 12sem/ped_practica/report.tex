\documentclass[12pt]{extarticle}
\usepackage[pdftex,a5paper,colorlinks,linkcolor=blue,citecolor=blue]{hyperref}	

\usepackage[utf8]{inputenc}
\usepackage[english,russian]{babel}
\usepackage{vmargin}
\usepackage{indentfirst}
\usepackage[T2A]{fontenc}
\usepackage{graphics}
\usepackage{amsthm}
\usepackage{amsbsy}
\usepackage{amsmath}
\usepackage{amssymb}
\usepackage{amsfonts}
\usepackage{mathtext}
\usepackage{mathrsfs}

\usepackage[pdftex]{graphicx}
\usepackage{array}
\usepackage{graphicx,xcolor}
\usepackage{xcolor}
\usepackage{float}
\usepackage{longtable}
\usepackage{hhline}
\usepackage{tikz}
\usepackage{verbatim}

\parindent = 30pt
\hoffset = 0pt
\voffset = 0pt
\oddsidemargin = 0pt
\topmargin = 0pt
\headheight = 0pt
\headsep = 25pt
\textheight = 700pt
\textwidth = 460pt
\marginparsep = 0pt
\marginparwidth = 0pt

\pagestyle{empty}
\makeatletter
\makeatother
\newcommand{\IntroPattern}[1]
{
\begin{flushright}
\textbf{Приложение} \\
к Договору на прохождение практики студентами \\
Московского государственного университета имени М.{\,}В.{\,}Ломоносова \\
в МАОУ <<Гимназия №77>>
\end{flushright}

\vspace{3em}

\begin{center}
\begin{Large}
{#1}
\end{Large}
\end{center}

\vspace{2em}

\noindent Студент 6 курса \quad {Сибгатуллин Артур Петрович} \\
\rule[1em]{8em}{0mm} \rule[1em]{31em}{0.5pt}

\noindent Руководитель от МГУ имени М.В. Ломоносова \quad {Арушанян Игорь Олегович} \\
\rule[1em]{22em}{0mm} \rule[1em]{17em}{0.5pt}

\noindent Руководитель от предприятия \quad {Бариева Гульнара Айратовна} \\
\rule[1em]{14em}{0mm} \rule[1em]{25em}{0.5pt}

\noindent Тема \quad {Начала математического анализа: Предел последовательности. } \\
\rule[1em]{2.5em}{0mm} \rule[1em]{36.5em}{0.5pt}

\noindent {Предел функции. Определение производной. Вычисление производной.} \\
\rule[1em]{39.3em}{0.5pt}

\vspace{1em}
}


\newcommand{\SignaturesPattern}
{
\vspace{3em}

\noindent
\parbox[c][]{17em}{
Организация: 
\rule[1em]{17em}{0pt}
МАОУ <<Гимназия №77>> \\
\rule[1em]{17em}{0.5pt}
директор \\
\rule[1em]{17em}{0.5pt}
Бариева Гульнара Айратовна\\
\rule[1em]{17em}{0.5pt}\\

\rule[1em]{17em}{0.5pt}}
\hspace{4em}
\parbox[c][]{17em}{
МГУ имени М.В. Ломоносова: 
\\
\rule[1em]{17em}{0pt}
\\ 

\rule[1em]{17em}{0.5pt}
\\

\rule[1em]{17em}{0.5pt}
\\

\rule[1em]{17em}{0.5pt}
\\

\rule[1em]{17em}{0.5pt}}
}


\begin{document}

\IntroPattern{Отчет о практике студента-практиканта}

В ходе подготовки к педагогической практике были изучены методические пособия по алгебре и началам математического анализа А. Г. Мордковича и А. П. Семенова (10 -- 11 класс), Алимова Ш. А. и Колягина Ю. М. (10 -- 11 класс)

Был составлен план занятий, рассчитанный на 8 академических часов лекционно-практических занятий, 1 занятия на повторение пройденного материала и подготовке к контрольному мероприятию, 1 контрольной работы.

Проведены 10 занятий ученикам 10 класса по темам:
\begin{itemize}
\item Предел:
\begin{itemize}
	\item[--] Пределы числовых последовательностей. Примеры. 1 урок.
	\item[--] Предел функции в точке. Приращения. 1 урок.
\end{itemize}
\item Определение производной:
\begin{itemize}
\item[--] Наивное определение производной. Примеры практических задач из которых оно получается. 1 урок
\item[--] Формальное определение производной, вычисление производной по ее определению. 2 урока.
\end{itemize}
\item Вычисление производных:
\begin{itemize}
\item[--] Правила дифференцирования. 1 урок
\item[--] Формулы производных для часто встречающихся функций. 2 урок
\end{itemize}
\item Контрольные мероприятия:
\begin{itemize}
\item[--] Повторение материала, подготовка к контрольной работе. 1 урок.
\item[--] Контрольная работа. 1 урок.
\end{itemize}
\end{itemize}

На уроках было два вида активности: лекционная (теория) и практическая (задачи), между этими активностями время делилось примерно пополам. Домашние задания были призваны закрепить практический материал, и их проверка проводилась в начале урока. Большинство учеников проявляли интерес к новым темам и активно участвовали на занятиях, выполняя домашние задания. Применение производных в реальных задачах было особенно интересным для учеников. Однако, понимание пределов и приращений оказалось сложным. Контрольная работа охватывала все изученные темы.

Студент-практикант прошел педагогическую практику с оценкой <<зачет>>.

\SignaturesPattern

\end{document}