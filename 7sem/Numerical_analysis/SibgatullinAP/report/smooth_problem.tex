\section{Отладочный тест}
\subsection{Постановка задачи}

Рассмотрим $Q = [0;1]$x$[0;1]$\\

Зададим функции $\tilde{\rho}(t, x)$ и $\tilde{u}(t, x)$ так, чтобы они являлись гладким решением задачи $(1)$.
\begin{equation}
	\begin{array}{lc}
		\tilde{\rho}(t, x) = e^t(cos(3\pi x) + 1.5),\\
		\tilde{u}(t, x) = cos(2\pi t)sin(4\pi x)\\
	\end{array}
\end{equation}

Теперь определим функции $f_0$ и $f$, так, чтобы они удовлетворяли уравнениям:
\begin{equation}
	\begin{array}{lc}
		\frac{\partial\tilde{\rho}}{\partial t} + \frac{\partial\tilde{\rho}\tilde{u}}{\partial x} = f_0,\\
		\tilde{\rho}\frac{\partial\tilde{u}}{\partial t} + \tilde{u}\tilde{\rho}\frac{\partial\tilde{u}}{\partial x} + \frac{\partial p}{\partial x} = \mu\frac{\partial^2\tilde{u}}{\partial x^2} + \tilde{\rho}f
	\end{array}
\end{equation}

\begin{equation}
	\begin{array}{lc}
		\frac{\partial\tilde{\rho}}{\partial t}  = e^t(cos(3\pi x) + 1.5), \\
		\frac{\partial\tilde{\rho}\tilde{u}}{\partial t}  = \pi e^t cos(2\pi t) * (4(cos(3\pi x) + 1.5) cos (4\pi x) - 3sin(3\pi x )sin(4\pi x)), \\
		
		\frac{\partial\tilde{u}}{\partial t} = -2\pi sin(2\pi t)sin(4\pi x),\\				\frac{\partial\tilde{u}}{\partial x} = 4\pi cos(2\pi t)cos(4\pi x),\\
		\frac{\partial^2\tilde{u}}{\partial x^2} = -16\pi^2 cos(2\pi t)sin(4\pi x),\\
		\frac{\partial\tilde{ru}}{\partial x} = \pi e^t cos(2\pi t) (4(cos(3\pi x) + 1.5)cos(4\pi x) - 3sin(3\pi x)sin(4\pi x)),\\
		\frac{\partial\tilde{p}}{\partial x} = C\gamma \rho^{\gamma - 1} \frac{\partial \rho}{\partial x}		
	\end{array}
\end{equation}
\newpage

\subsection{Численные эксперименты}
\subsubsection{Обычная сетка}

\begin{center}
Table of times.
  
\begin{tabular}{|p{1in}|p{1in}|p{1in}|p{1in}|} \hline
$k / \tau = h$ &1.000e-01 &1.000e-02 &1.000e-03 \\ \hline 
0.000e+00 &6.300e-05 &5.076e-03 &3.549e+00 \\ \hline 
1.000e+00 &9.780e-04 &4.994e-01 &3.386e+01 \\ \hline 
2.000e+00 &1.634e-02 &1.789e+00 &1.067e+02 \\ \hline 
3.000e+00 &1.031e-01 &6.765e+00 &2.565e+02 \\ \hline 

\end{tabular}\\[20pt]
\end{center}


\begin{center}
Table of norms for H. $\mu = 0.0010$ \, $C = 100.0000$, $\gamma = 1.0000$
  
\begin{tabular}{|p{1in}|p{1in}|p{1in}|p{1in}|p{1in}|} \hline
$\tau / h$ &1.000e-01 &1.000e-02 &1.000e-03 &1.000e-04 \\ \hline 
1.000e-01 & $1.584e+07$  $1.042e+07$  $4.357e+07$  & $4.294e+10$  $6.986e+09$  $9.938e+11$  & $1.689e+15$  $1.246e+14$  $2.378e+17$  & $4.351e+16$  $5.594e+15$  $8.077e+19$  \\ \hline 
1.000e-02 & $6.821e+35$  $3.634e+35$  $7.863e+36$  & $9.897e+89$  $1.347e+89$  $2.183e+91$  & $3.820e+113$  $5.373e+112$  $6.665e+115$  & $1.583e+183$  $inf$  $inf$  \\ \hline 
1.000e-03 & $1.065e+236$  $inf$  $inf$  & $nan$  $-nan$  $-nan$  & $nan$  $-nan$  $-nan$  & $nan$  $-nan$  $-nan$  \\ \hline 
1.000e-04 & $nan$  $-nan$  $-nan$  & $nan$  $-nan$  $-nan$  & $nan$  $-nan$  $-nan$  & $nan$  $-nan$  $-nan$  \\ \hline 

\end{tabular}\\[20pt]
\end{center}

\newpage

\begin{center}
Table of norms for V. $\mu = 0.0010$ \, $C = 100.0000$, $\gamma = 1.0000$
  
\begin{tabular}{|p{1in}|p{1in}|p{1in}|p{1in}|} \hline
$k / \tau = h$ &1.000e-01 &1.000e-02 &1.000e-03 \\ \hline 
0.000e+00 & $1.034e+02$  $5.728e+01$  $8.862e+02$  & $1.043e+03$  $3.108e+02$  $4.887e+04$  & $nan$  $nan$  $nan$  \\ \hline 
1.000e+00 & $9.785e+00$  $5.707e+00$  $5.288e+01$  & $nan$  $-nan$  $-nan$  & $nan$  $nan$  $nan$  \\ \hline 
2.000e+00 & $5.901e+02$  $2.373e+02$  $2.618e+03$  & $nan$  $nan$  $nan$  & $nan$  $nan$  $nan$  \\ \hline 
3.000e+00 & $2.592e+02$  $1.449e+02$  $1.883e+03$  & $nan$  $nan$  $nan$  & $nan$  $nan$  $nan$  \\ \hline 

\end{tabular}\\[20pt]
\end{center}

\begin{center}
Table of norms for H. $\mu = 0.0010$ \, $C = 10.0000$, $\gamma = 1.0000$
  
\begin{tabular}{|p{1in}|p{1in}|p{1in}|p{1in}|p{1in}|} \hline
$\tau / h$ &1.000e-01 &1.000e-02 &1.000e-03 &1.000e-04 \\ \hline 
1.000e-01 & $3.090e+04$  $1.439e+04$  $1.701e+05$  & $5.084e+06$  $6.602e+05$  $1.094e+08$  & $1.411e+08$  $3.608e+07$  $4.934e+10$  & $8.968e+11$  $5.488e+10$  $7.003e+14$  \\ \hline 
1.000e-02 & $3.100e+39$  $1.131e+39$  $5.512e+39$  & $4.683e+59$  $1.051e+59$  $2.281e+61$  & $1.080e+99$  $8.138e+97$  $1.285e+101$  & $2.012e+142$  $6.477e+140$  $9.644e+144$  \\ \hline 
1.000e-03 & $1.325e+156$  $inf$  $inf$  & $3.595e+221$  $inf$  $inf$  & $nan$  $-nan$  $-nan$  & $nan$  $-nan$  $-nan$  \\ \hline 
1.000e-04 & $6.715e+251$  $inf$  $inf$  & $2.001e-02$  $2.598e-03$  $2.318e-01$  & $nan$  $-nan$  $-nan$  & $nan$  $-nan$  $-nan$  \\ \hline 

\end{tabular}\\[20pt]
\end{center}


\newpage
\begin{center}
Table of norms for V. $\mu = 0.0010$ \, $C = 10.0000$, $\gamma = 1.0000$
  
\begin{tabular}{|p{1in}|p{1in}|p{1in}|p{1in}|p{1in}|} \hline
$\tau / h$ &1.000e-01 &1.000e-02 &1.000e-03 &1.000e-04 \\ \hline 
1.000e-01 & $1.346e+01$  $6.953e+00$  $9.841e+01$  & $1.000e+04$  $1.373e+03$  $2.189e+05$  & $5.955e+01$  $6.702e+00$  $8.298e+03$  & $1.435e+02$  $2.333e+01$  $1.245e+04$  \\ \hline 
1.000e-02 & $3.654e+01$  $2.468e+01$  $3.697e+02$  & $9.486e+01$  $1.691e+01$  $2.359e+03$  & $4.059e+04$  $2.829e+03$  $4.193e+06$  & $1.590e+03$  $7.636e+01$  $1.046e+06$  \\ \hline 
1.000e-03 & $5.806e+01$  $3.838e+01$  $5.856e+02$  & $6.277e+01$  $3.504e+01$  $4.747e+03$  & $nan$  $nan$  $nan$  & $nan$  $nan$  $nan$  \\ \hline 
1.000e-04 & $1.267e+02$  $8.832e+01$  $1.312e+03$  & $4.506e-03$  $2.344e-03$  $6.926e-02$  & $nan$  $nan$  $nan$  & $nan$  $nan$  $nan$  \\ \hline 

\end{tabular}\\[20pt]
\end{center}

\begin{center}
Table of norms for H. $\mu = 0.0010$ \, $C = 1.0000$, $\gamma = 1.0000$
  
\begin{tabular}{|p{1in}|p{1in}|p{1in}|p{1in}|p{1in}|} \hline
$\tau / h$ &1.000e-01 &1.000e-02 &1.000e-03 &1.000e-04 \\ \hline 
1.000e-01 & $1.140e+02$  $4.643e+01$  $6.830e+02$  & $2.303e+03$  $4.776e+02$  $9.043e+04$  & $1.457e+02$  $9.078e+00$  $1.202e+04$  & $8.885e+00$  $4.294e+00$  $8.702e+02$  \\ \hline 
1.000e-02 & $3.773e+04$  $8.832e+03$  $2.074e+05$  & $2.865e+25$  $3.835e+24$  $7.577e+26$  & $5.438e+41$  $3.827e+40$  $6.606e+43$  & $2.910e+39$  $6.688e+37$  $1.054e+42$  \\ \hline 
1.000e-03 & $8.363e+43$  $1.969e+43$  $4.653e+44$  & $1.958e+00$  $1.584e-01$  $2.211e+01$  & $6.365e+257$  $inf$  $inf$  & $nan$  $-nan$  $-nan$  \\ \hline 
1.000e-04 & $5.700e+49$  $1.284e+49$  $1.965e+50$  & $3.963e-02$  $8.792e-03$  $1.213e+00$  & $1.290e-02$  $2.311e-03$  $2.981e-01$  & $1.253e-02$  $2.319e-03$  $2.996e-01$  \\ \hline 

\end{tabular}\\[20pt]
\end{center}


\newpage
\begin{center}
Table of norms for V. $\mu = 0.0010$ \, $C = 1.0000$, $\gamma = 1.0000$
  
\begin{tabular}{|p{1in}|p{1in}|p{1in}|p{1in}|p{1in}|} \hline
$\tau / h$ &1.000e-01 &1.000e-02 &1.000e-03 &1.000e-04 \\ \hline 
1.000e-01 & $6.763e+00$  $3.636e+00$  $4.417e+01$  & $1.806e+02$  $2.261e+01$  $2.501e+03$  & $4.581e+00$  $1.432e+00$  $5.801e+01$  & $6.133e+00$  $3.709e+00$  $3.239e+01$  \\ \hline 
1.000e-02 & $9.705e+00$  $4.596e+00$  $6.103e+01$  & $9.693e+01$  $1.074e+01$  $1.588e+03$  & $1.537e+03$  $8.712e+01$  $1.470e+05$  & $1.731e+03$  $3.155e+01$  $4.365e+05$  \\ \hline 
1.000e-03 & $2.396e+01$  $1.489e+01$  $2.089e+02$  & $3.431e-02$  $1.313e-02$  $1.777e+00$  & $1.712e+02$  $7.741e+00$  $1.111e+04$  & $nan$  $-nan$  $-nan$  \\ \hline 
1.000e-04 & $7.905e+01$  $4.418e+01$  $7.942e+02$  & $1.048e-02$  $2.817e-03$  $2.426e-01$  & $2.261e-03$  $6.061e-04$  $7.027e-02$  & $2.311e-03$  $6.073e-04$  $7.104e-02$  \\ \hline 

\end{tabular}\\[20pt]
\end{center}

\begin{center}
Table of norms for H. $\mu = 0.0010$ \, $C = 1.0000$, $\gamma = 1.4000$
  
\begin{tabular}{|p{1in}|p{1in}|p{1in}|p{1in}|p{1in}|} \hline
$\tau / h$ &1.000e-01 &1.000e-02 &1.000e-03 &1.000e-04 \\ \hline 
1.000e-01 & $nan$  $-nan$  $-nan$  & $nan$  $-nan$  $-nan$  & $nan$  $-nan$  $-nan$  & $nan$  $-nan$  $-nan$  \\ \hline 
1.000e-02 & $nan$  $-nan$  $-nan$  & $nan$  $-nan$  $-nan$  & $nan$  $-nan$  $-nan$  & $nan$  $-nan$  $-nan$  \\ \hline 
1.000e-03 & $nan$  $-nan$  $-nan$  & $nan$  $-nan$  $-nan$  & $nan$  $-nan$  $-nan$  & $nan$  $-nan$  $-nan$  \\ \hline 
1.000e-04 & $nan$  $-nan$  $-nan$  & $1.498e-02$  $4.873e-03$  $5.353e-01$  & $6.867e-04$  $2.722e-04$  $5.344e-03$  & $nan$  $-nan$  $-nan$  \\ \hline 

\end{tabular}\\[20pt]
\end{center}


\newpage
\begin{center}
Table of norms for V. $\mu = 0.0010$ \, $C = 1.0000$, $\gamma = 1.4000$
  
\begin{tabular}{|p{1in}|p{1in}|p{1in}|p{1in}|p{1in}|} \hline
$\tau / h$ &1.000e-01 &1.000e-02 &1.000e-03 &1.000e-04 \\ \hline 
1.000e-01 & $nan$  $nan$  $nan$  & $nan$  $-nan$  $-nan$  & $nan$  $-nan$  $-nan$  & $nan$  $-nan$  $-nan$  \\ \hline 
1.000e-02 & $nan$  $nan$  $nan$  & $nan$  $-nan$  $-nan$  & $nan$  $nan$  $nan$  & $nan$  $nan$  $nan$  \\ \hline 
1.000e-03 & $nan$  $nan$  $nan$  & $nan$  $-nan$  $-nan$  & $nan$  $-nan$  $-nan$  & $nan$  $-nan$  $-nan$  \\ \hline 
1.000e-04 & $nan$  $-nan$  $-nan$  & $7.856e-03$  $2.837e-03$  $2.038e-01$  & $2.161e-04$  $1.035e-04$  $1.978e-03$  & $nan$  $nan$  $nan$  \\ \hline 

\end{tabular}\\[20pt]
\end{center}

\begin{center}
Table of norms for H. $\mu = 0.0100$ \, $C = 100.0000$, $\gamma = 1.0000$
  
\begin{tabular}{|p{1in}|p{1in}|p{1in}|p{1in}|p{1in}|} \hline
$k / \tau = h$ &1.000e-01 &1.000e-02 &1.000e-03 &1.000e-04 \\ \hline 
0.000e+00 & $1.989e+10$  $8.497e+09$  $1.238e+11$  & $7.628e+80$  $1.226e+80$  $1.901e+82$  & $nan$  $-nan$  $-nan$  & $nan$  $-nan$  $-nan$  \\ \hline 
1.000e+00 & $6.687e+34$  $7.942e+33$  $7.276e+35$  & $nan$  $-nan$  $-nan$  & $nan$  $-nan$  $-nan$  & $nan$  $-nan$  $-nan$  \\ \hline 
2.000e+00 & $5.651e+66$  $9.826e+65$  $3.858e+67$  & $nan$  $-nan$  $-nan$  & $nan$  $-nan$  $-nan$  & $nan$  $-nan$  $-nan$  \\ \hline 
3.000e+00 & $1.141e+141$  $6.398e+139$  $1.510e+142$  & $nan$  $-nan$  $-nan$  & $nan$  $-nan$  $-nan$  & $nan$  $-nan$  $-nan$  \\ \hline 

\end{tabular}\\[20pt]
\end{center}


\newpage
\begin{center}
Table of norms for V. $\mu = 0.0100$ \, $C = 100.0000$, $\gamma = 1.0000$
  
\begin{tabular}{|p{1in}|p{1in}|p{1in}|p{1in}|} \hline
$k / \tau = h$ &1.000e-01 &1.000e-02 &1.000e-03 \\ \hline 
0.000e+00 & $7.861e+02$  $2.878e+02$  $4.036e+03$  & $7.356e+02$  $1.340e+02$  $2.280e+04$  & $nan$  $-nan$  $-nan$  \\ \hline 
1.000e+00 & $1.426e+02$  $5.667e+01$  $6.430e+02$  & $nan$  $nan$  $nan$  & $nan$  $nan$  $nan$  \\ \hline 
2.000e+00 & $3.310e+04$  $1.047e+04$  $1.824e+05$  & $nan$  $nan$  $nan$  & $nan$  $nan$  $nan$  \\ \hline 
3.000e+00 & $1.090e+02$  $5.249e+01$  $6.161e+02$  & $nan$  $nan$  $nan$  & $nan$  $nan$  $nan$  \\ \hline 

\end{tabular}\\[20pt]
\end{center}

\begin{center}
Table of norms for H. $\mu = 0.0100$ \, $C = 10.0000$, $\gamma = 1.0000$
  
\begin{tabular}{|p{1in}|p{1in}|p{1in}|p{1in}|p{1in}|} \hline
$\tau / h$ &1.000e-01 &1.000e-02 &1.000e-03 &1.000e-04 \\ \hline 
1.000e-01 & $2.542e+05$  $1.075e+05$  $2.196e+06$  & $1.920e+03$  $3.960e+02$  $7.949e+04$  & $1.463e+03$  $4.806e+01$  $8.785e+04$  & $2.767e+02$  $1.104e+01$  $2.032e+05$  \\ \hline 
1.000e-02 & $2.070e+21$  $1.228e+21$  $2.673e+22$  & $6.209e+57$  $7.450e+56$  $1.364e+59$  & $1.553e+59$  $1.705e+58$  $2.382e+61$  & $1.167e+15$  $5.468e+13$  $7.926e+17$  \\ \hline 
1.000e-03 & $3.535e+73$  $8.893e+72$  $1.600e+74$  & $6.788e+146$  $9.242e+145$  $2.193e+148$  & $nan$  $-nan$  $-nan$  & $nan$  $-nan$  $-nan$  \\ \hline 
1.000e-04 & $2.105e+242$  $inf$  $inf$  & $3.099e-03$  $1.747e-03$  $2.052e-02$  & $4.080e-04$  $1.875e-04$  $1.195e-03$  & $4.225e-04$  $1.939e-04$  $1.259e-03$  \\ \hline 

\end{tabular}\\[20pt]
\end{center}


\newpage
\begin{center}
Table of norms for H. $\mu = 0.0100$ \, $C = 10.0000$, $\gamma = 1.0000$
  
\begin{tabular}{|p{1in}|p{1in}|p{1in}|p{1in}|p{1in}|} \hline
$\tau / h$ &1.000e-01 &1.000e-02 &1.000e-03 &1.000e-04 \\ \hline 
1.000e-01 & $2.159e+01$  $9.587e+00$  $1.198e+02$  & $2.471e+02$  $3.984e+01$  $4.609e+03$  & $2.133e+05$  $8.964e+03$  $1.237e+07$  & $9.071e+06$  $9.628e+04$  $1.372e+09$  \\ \hline 
1.000e-02 & $3.757e+01$  $1.960e+01$  $2.769e+02$  & $7.268e+03$  $7.591e+02$  $1.296e+05$  & $4.183e+03$  $2.453e+02$  $3.613e+05$  & $1.706e+04$  $5.245e+02$  $7.946e+06$  \\ \hline 
1.000e-03 & $8.250e+01$  $5.484e+01$  $9.331e+02$  & $5.382e+01$  $2.538e+01$  $3.733e+03$  & $nan$  $-nan$  $-nan$  & $nan$  $-nan$  $-nan$  \\ \hline 
1.000e-04 & $6.557e+01$  $4.134e+01$  $6.523e+02$  & $4.257e-03$  $2.137e-03$  $3.011e-02$  & $nan$  $-nan$  $-nan$  & $nan$  $-nan$  $-nan$  \\ \hline 

\end{tabular}\\[20pt]
\end{center}

\begin{center}
Table of norms for H. $\mu = 0.0100$ \, $C = 1.0000$, $\gamma = 1.0000$
  
\begin{tabular}{|p{1in}|p{1in}|p{1in}|p{1in}|p{1in}|} \hline
$k / \tau = h$ &1.000e-01 &1.000e-02 &1.000e-03 &1.000e-04 \\ \hline 
0.000e+00 & $6.976e+01$  $3.338e+01$  $6.133e+02$  & $1.138e+36$  $1.241e+35$  $2.314e+37$  & $nan$  $-nan$  $-nan$  & $nan$  $-nan$  $-nan$  \\ \hline 
1.000e+00 & $2.047e+09$  $2.297e+08$  $1.740e+10$  & $1.040e+154$  $3.849e+152$  $inf$  & $nan$  $-nan$  $-nan$  & $nan$  $-nan$  $-nan$  \\ \hline 
2.000e+00 & $1.520e+18$  $1.829e+17$  $1.101e+19$  & $2.689e+271$  $inf$  $inf$  & $nan$  $-nan$  $-nan$  & $nan$  $-nan$  $-nan$  \\ \hline 
3.000e+00 & $3.045e+39$  $1.702e+38$  $3.889e+40$  & $nan$  $-nan$  $-nan$  & $nan$  $-nan$  $-nan$  & $nan$  $-nan$  $-nan$  \\ \hline 

\end{tabular}\\[20pt]
\end{center}


\newpage
\begin{center}
Table of norms for H. $\mu = 0.0100$ \, $C = 1.0000$, $\gamma = 1.0000$
  
\begin{tabular}{|p{1in}|p{1in}|p{1in}|p{1in}|p{1in}|} \hline
$\tau / h$ &1.000e-01 &1.000e-02 &1.000e-03 &1.000e-04 \\ \hline 
1.000e-01 & $6.402e+00$  $3.503e+00$  $6.085e+01$  & $1.104e+02$  $1.829e+01$  $2.648e+03$  & $5.696e+07$  $2.433e+06$  $3.442e+09$  & $4.113e+07$  $1.169e+06$  $1.652e+10$  \\ \hline 
1.000e-02 & $1.294e+01$  $6.238e+00$  $1.109e+02$  & $3.998e+01$  $7.511e+00$  $1.111e+03$  & $1.491e+03$  $9.452e+01$  $1.439e+05$  & $nan$  $-nan$  $-nan$  \\ \hline 
1.000e-03 & $1.213e+01$  $7.867e+00$  $1.109e+02$  & $1.218e-02$  $4.318e-03$  $1.627e-01$  & $nan$  $nan$  $nan$  & $nan$  $nan$  $nan$  \\ \hline 
1.000e-04 & $1.310e+01$  $8.536e+00$  $1.386e+02$  & $5.115e-03$  $2.362e-03$  $4.356e-02$  & $nan$  $-nan$  $-nan$  & $nan$  $nan$  $nan$  \\ \hline 

\end{tabular}\\[20pt]
\end{center}

\begin{center}
Table of norms for H. $\mu = 0.0100$ \, $C = 1.0000$, $\gamma = 1.4000$
  
\begin{tabular}{|p{0.8in}|p{0.8in}|p{0.8in}|p{0.8in}|p{0.8in}|p{0.8in}|p{0.8in}|} \hline
$K$ &$N_0$ &$N_0 \tau$ &$n = \frac{N_0}{4}$ &$n = \frac{N_0}{2}$ &$n = \frac{3N_0}{4}$ &$n = N_0$ \\ \hline 
0 & & & & & & \\ \hline 
1 & & & & & & \\ \hline 
2 & & & & & & \\ \hline 
3 & & & & & & \\ \hline 

\end{tabular}\\[20pt]
\end{center}


\newpage
\begin{center}
Table of norms for V. $\mu = 0.0100$ \, $C = 1.0000$, $\gamma = 1.4000$
  
\begin{tabular}{|p{0.8in}|p{0.8in}|p{0.8in}|p{0.8in}|p{0.8in}|p{0.8in}|p{0.8in}|} \hline
$K$ &$N_0$ &$N_0 \tau$ &$n = \frac{N_0}{4}$ &$n = \frac{N_0}{2}$ &$n = \frac{3N_0}{4}$ &$n = N_0$ \\ \hline 
0 &919330 &9.193e+02 &1.980e-02 &8.076e-03 &6.651e-03 &9.998e-04 \\ \hline 
1 &1838659 &9.193e+02 &1.891e-02 &8.395e-03 &5.863e-03 &1.467e-03 \\ \hline 
2 &3677317 &9.193e+02 &1.853e-02 &8.487e-03 &5.518e-03 &1.632e-03 \\ \hline 
3 &7354633 &9.193e+02 &1.835e-02 &8.521e-03 &5.358e-03 &1.701e-03 \\ \hline 

\end{tabular}\\[20pt]
\end{center}

\begin{center}
Table of norms for H. $\mu = 0.1000$ \, $C = 100.0000$, $\gamma = 1.0000$
  
\begin{tabular}{|p{1in}|p{1in}|p{1in}|p{1in}|p{1in}|} \hline
$\tau / h$ &1.000e-01 &1.000e-02 &1.000e-03 &1.000e-04 \\ \hline 
1.000e-01 & $1.092e+13$  $7.623e+12$  $1.475e+14$  & $2.732e+09$  $7.216e+08$  $1.095e+11$  & $3.079e+13$  $3.724e+12$  $5.262e+15$  & $4.809e+18$  $2.406e+18$  $3.401e+22$  \\ \hline 
1.000e-02 & $3.692e+57$  $1.636e+57$  $4.589e+57$  & $6.120e+95$  $7.712e+94$  $1.331e+97$  & $4.267e+125$  $1.511e+124$  $2.109e+127$  & $1.905e+160$  $inf$  $inf$  \\ \hline 
1.000e-03 & $1.727e+182$  $inf$  $inf$  & $nan$  $-nan$  $-nan$  & $nan$  $-nan$  $-nan$  & $nan$  $-nan$  $-nan$  \\ \hline 
1.000e-04 & $6.774e+272$  $inf$  $inf$  & $3.805e-03$  $2.345e-03$  $2.391e-02$  & $nan$  $-nan$  $-nan$  & $nan$  $-nan$  $-nan$  \\ \hline 

\end{tabular}\\[20pt]
\end{center}


\newpage
\begin{center}
Table of norms for H. $\mu = 0.1000$ \, $C = 100.0000$, $\gamma = 1.0000$
  
\begin{tabular}{|p{1in}|p{1in}|p{1in}|p{1in}|p{1in}|} \hline
$\tau / h$ &1.000e-01 &1.000e-02 &1.000e-03 &1.000e-04 \\ \hline 
1.000e-01 & $1.396e+02$  $7.026e+01$  $1.006e+03$  & $6.052e+04$  $1.539e+04$  $2.373e+06$  & $4.018e+03$  $3.036e+02$  $4.364e+05$  & $3.551e+06$  $4.198e+04$  $5.933e+08$  \\ \hline 
1.000e-02 & $7.060e+01$  $4.460e+01$  $6.486e+02$  & $1.619e+03$  $2.911e+02$  $4.415e+04$  & $4.477e+03$  $6.658e+02$  $9.019e+05$  & $1.287e+05$  $4.590e+03$  $6.990e+07$  \\ \hline 
1.000e-03 & $3.685e+02$  $2.594e+02$  $3.850e+03$  & $nan$  $-nan$  $-nan$  & $nan$  $-nan$  $-nan$  & $nan$  $-nan$  $-nan$  \\ \hline 
1.000e-04 & $3.042e+02$  $2.134e+02$  $3.178e+03$  & $3.322e-03$  $1.908e-03$  $2.599e-02$  & $nan$  $-nan$  $-nan$  & $nan$  $-nan$  $-nan$  \\ \hline 

\end{tabular}\\[20pt]
\end{center}

\begin{center}
Table of norms for H. $\mu = 0.1000$ \, $C = 10.0000$, $\gamma = 1.0000$
  
\begin{tabular}{|p{1in}|p{1in}|p{1in}|p{1in}|} \hline
$k / \tau = h$ &1.000e-01 &1.000e-02 &1.000e-03 \\ \hline 
0.000e+00 & $1.098e+06$  $2.630e+05$  $3.775e+06$  & $1.780e+07$  $2.985e+06$  $5.849e+08$  & $4.464e-03$  $2.043e-03$  $1.222e-02$  \\ \hline 
1.000e+00 & $2.667e+04$  $1.528e+04$  $1.183e+05$  & $4.318e+09$  $5.246e+08$  $3.967e+10$  & $2.217e-03$  $1.016e-03$  $6.080e-03$  \\ \hline 
2.000e+00 & $1.591e+13$  $3.559e+12$  $5.136e+13$  & $1.146e-02$  $5.196e-03$  $3.122e-02$  & $1.105e-03$  $5.069e-04$  $3.033e-03$  \\ \hline 
3.000e+00 & $1.794e+13$  $4.216e+12$  $6.360e+13$  & $5.599e-03$  $2.561e-03$  $1.530e-02$  & $5.517e-04$  $2.532e-04$  $1.515e-03$  \\ \hline 
4.000e+00 & $1.262e+13$  $4.458e+12$  $1.937e+13$  & $2.775e-03$  $1.272e-03$  $7.603e-03$  & $2.756e-04$  $1.265e-04$  $7.572e-04$  \\ \hline 

\end{tabular}\\[20pt]
\end{center}


\newpage
\begin{center}
Table of norms for V. $\mu = 0.1000$ \, $C = 10.0000$, $\gamma = 1.0000$
  
\begin{tabular}{|p{0.8in}|p{0.8in}|p{0.8in}|p{0.8in}|p{0.8in}|p{0.8in}|p{0.8in}|} \hline
$K$ &$N_0$ &$N_0 \tau$ &$n = \frac{N_0}{4}$ &$n = \frac{N_0}{2}$ &$n = \frac{3N_0}{4}$ &$n = N_0$ \\ \hline 
0 &163946 &1.639e+02 &1.087e-01 &3.157e-02 &8.669e-03 &9.950e-04 \\ \hline 
1 &327891 &1.639e+02 &1.041e-01 &2.907e-02 &7.359e-03 &9.549e-04 \\ \hline 
2 &655781 &1.639e+02 &1.018e-01 &2.785e-02 &6.758e-03 &9.451e-04 \\ \hline 
3 &1311561 &1.639e+02 &1.007e-01 &2.724e-02 &6.465e-03 &9.426e-04 \\ \hline 

\end{tabular}\\[20pt]
\end{center}

\begin{center}
Table of norms for H. $\mu = 0.1000$ \, $C = 1.0000$, $\gamma = 1.0000$
  
\begin{tabular}{|p{1in}|p{1in}|p{1in}|p{1in}|} \hline
$k / \tau = h$ &1.000e-01 &1.000e-02 &1.000e-03 \\ \hline 
0.000e+00 & $6.007e+02$  $3.064e+02$  $7.033e+03$  & $1.568e+00$  $4.277e-01$  $2.557e+01$  & $7.259e-02$  $1.972e-02$  $4.754e-01$  \\ \hline 
1.000e+00 & $9.374e+03$  $2.131e+03$  $3.814e+04$  & $4.478e-01$  $1.176e-01$  $3.736e+00$  & $3.541e-02$  $9.683e-03$  $2.289e-01$  \\ \hline 
2.000e+00 & $3.181e+03$  $7.419e+02$  $8.624e+03$  & $1.947e-01$  $5.225e-02$  $1.339e+00$  & $1.749e-02$  $4.799e-03$  $1.124e-01$  \\ \hline 
3.000e+00 & $1.757e+00$  $8.537e-01$  $1.009e+01$  & $9.129e-02$  $2.488e-02$  $5.997e-01$  & $8.691e-03$  $2.389e-03$  $5.570e-02$  \\ \hline 
4.000e+00 & $5.943e-01$  $1.971e-01$  $2.880e+00$  & $4.440e-02$  $1.216e-02$  $2.859e-01$  & $4.332e-03$  $1.192e-03$  $2.773e-02$  \\ \hline 

\end{tabular}\\[20pt]
\end{center}


\newpage
\begin{center}
Table of norms for V. $\mu = 0.1000$ \, $C = 1.0000$, $\gamma = 1.0000$
  
\begin{tabular}{|p{1in}|p{1in}|p{1in}|p{1in}|p{1in}|} \hline
$\tau / h$ &1.000e-01 &1.000e-02 &1.000e-03 &1.000e-04 \\ \hline 
1.000e-01 & $5.855e+00$  $3.789e+00$  $6.347e+01$  & $2.239e+00$  $1.058e+00$  $3.760e+01$  & $1.067e+01$  $5.882e+00$  $2.601e+01$  & $1.850e+00$  $9.454e-01$  $1.013e+01$  \\ \hline 
1.000e-02 & $9.409e+00$  $5.648e+00$  $8.658e+01$  & $2.318e-01$  $7.808e-02$  $1.927e+00$  & $2.597e-01$  $8.324e-02$  $2.107e+00$  & $2.676e-01$  $8.564e-02$  $2.166e+00$  \\ \hline 
1.000e-03 & $2.178e+01$  $1.149e+01$  $1.968e+02$  & $1.818e-02$  $6.728e-03$  $1.157e-01$  & $1.392e-02$  $5.080e-03$  $9.453e-02$  & $1.388e-02$  $5.066e-03$  $9.436e-02$  \\ \hline 
1.000e-04 & $4.808e+01$  $3.105e+01$  $4.787e+02$  & $5.519e-03$  $2.421e-03$  $3.727e-02$  & $1.376e-03$  $5.080e-04$  $9.219e-03$  & $1.337e-03$  $4.946e-04$  $9.061e-03$  \\ \hline 

\end{tabular}\\[20pt]
\end{center}

\begin{center}
Table of norms for H. $\mu = 0.1000$ \, $C = 1.0000$, $\gamma = 1.4000$
  
\begin{tabular}{|p{1in}|p{1in}|p{1in}|p{1in}|} \hline
$k / \tau = h$ &1.000e-01 &1.000e-02 &1.000e-03 \\ \hline 
0.000e+00 & $nan$  $nan$  $nan$  & $1.685e+00$  $2.629e-01$  $1.310e+01$  & $1.074e-02$  $4.336e-03$  $5.749e-02$  \\ \hline 
1.000e+00 & $nan$  $-nan$  $-nan$  & $6.222e-02$  $2.425e-02$  $3.757e-01$  & $5.250e-03$  $2.136e-03$  $2.840e-02$  \\ \hline 
2.000e+00 & $nan$  $nan$  $nan$  & $2.830e-02$  $1.129e-02$  $1.477e-01$  & $2.595e-03$  $1.060e-03$  $1.411e-02$  \\ \hline 
3.000e+00 & $nan$  $nan$  $nan$  & $1.355e-02$  $5.459e-03$  $7.179e-02$  & $1.290e-03$  $5.281e-04$  $7.035e-03$  \\ \hline 
4.000e+00 & $9.378e-02$  $3.781e-02$  $3.758e-01$  & $6.579e-03$  $2.680e-03$  $3.536e-02$  & $6.431e-04$  $2.635e-04$  $3.512e-03$  \\ \hline 

\end{tabular}\\[20pt]
\end{center}


\newpage
\begin{center}
Table of norms for V. $\mu = 0.1000$ \, $C = 1.0000$, $\gamma = 1.4000$
  
\begin{tabular}{|p{1in}|p{1in}|p{1in}|p{1in}|p{1in}|} \hline
$\tau / h$ &1.000e-01 &1.000e-02 &1.000e-03 &1.000e-04 \\ \hline 
1.000e-01 & $nan$  $nan$  $nan$  & $nan$  $-nan$  $-nan$  & $nan$  $-nan$  $-nan$  & $nan$  $-nan$  $-nan$  \\ \hline 
1.000e-02 & $nan$  $-nan$  $-nan$  & $1.079e-01$  $4.405e-02$  $1.733e+00$  & $8.880e-02$  $2.896e-02$  $1.526e+00$  & $8.242e-02$  $2.863e-02$  $1.415e+00$  \\ \hline 
1.000e-03 & $nan$  $nan$  $nan$  & $6.253e-03$  $3.009e-03$  $3.736e-02$  & $3.227e-03$  $1.517e-03$  $2.056e-02$  & $3.228e-03$  $1.507e-03$  $2.050e-02$  \\ \hline 
1.000e-04 & $nan$  $nan$  $nan$  & $3.758e-03$  $1.963e-03$  $2.731e-02$  & $3.241e-04$  $1.593e-04$  $2.093e-03$  & $3.202e-04$  $1.480e-04$  $2.016e-03$  \\ \hline 

\end{tabular}\\[20pt]
\end{center}


\subsection{Выводы}
По результатам численного эксперимента можно сделать вывод, что схема является условно сходящейся. Также, обратим внимание на то, что сходимость сильно зависит от $C$, $\gamma$, $\mu$. Худшая сходимость при больших $С$ и маленьких $\mu$. Обратим внимание на то, что при $\tau < h$ невзяки наименьшие. Сходимость схемы порядка $\tau + h^2$

\newpage
\subsubsection{Вложенная сетка}
\begin{center}
Table of times.
  
\begin{tabular}{|p{1in}|p{1in}|p{1in}|p{1in}|} \hline
$k / \tau = h$ &1.000e-01 &1.000e-02 &1.000e-03 \\ \hline 
0.000e+00 &6.300e-05 &5.076e-03 &3.549e+00 \\ \hline 
1.000e+00 &9.780e-04 &4.994e-01 &3.386e+01 \\ \hline 
2.000e+00 &1.634e-02 &1.789e+00 &1.067e+02 \\ \hline 
3.000e+00 &1.031e-01 &6.765e+00 &2.565e+02 \\ \hline 

\end{tabular}\\[20pt]
\end{center}


\begin{center}
Table of norms for H. $\mu = 0.0010$ \, $C = 100.0000$, $\gamma = 1.0000$
  
\begin{tabular}{|p{1in}|p{1in}|p{1in}|p{1in}|p{1in}|} \hline
$\tau / h$ &1.000e-01 &1.000e-02 &1.000e-03 &1.000e-04 \\ \hline 
1.000e-01 & $1.584e+07$  $1.042e+07$  $4.357e+07$  & $4.294e+10$  $6.986e+09$  $9.938e+11$  & $1.689e+15$  $1.246e+14$  $2.378e+17$  & $4.351e+16$  $5.594e+15$  $8.077e+19$  \\ \hline 
1.000e-02 & $6.821e+35$  $3.634e+35$  $7.863e+36$  & $9.897e+89$  $1.347e+89$  $2.183e+91$  & $3.820e+113$  $5.373e+112$  $6.665e+115$  & $1.583e+183$  $inf$  $inf$  \\ \hline 
1.000e-03 & $1.065e+236$  $inf$  $inf$  & $nan$  $-nan$  $-nan$  & $nan$  $-nan$  $-nan$  & $nan$  $-nan$  $-nan$  \\ \hline 
1.000e-04 & $nan$  $-nan$  $-nan$  & $nan$  $-nan$  $-nan$  & $nan$  $-nan$  $-nan$  & $nan$  $-nan$  $-nan$  \\ \hline 

\end{tabular}\\[20pt]
\end{center}

\newpage

\begin{center}
Table of norms for V. $\mu = 0.0010$ \, $C = 100.0000$, $\gamma = 1.0000$
  
\begin{tabular}{|p{1in}|p{1in}|p{1in}|p{1in}|} \hline
$k / \tau = h$ &1.000e-01 &1.000e-02 &1.000e-03 \\ \hline 
0.000e+00 & $1.034e+02$  $5.728e+01$  $8.862e+02$  & $1.043e+03$  $3.108e+02$  $4.887e+04$  & $nan$  $nan$  $nan$  \\ \hline 
1.000e+00 & $9.785e+00$  $5.707e+00$  $5.288e+01$  & $nan$  $-nan$  $-nan$  & $nan$  $nan$  $nan$  \\ \hline 
2.000e+00 & $5.901e+02$  $2.373e+02$  $2.618e+03$  & $nan$  $nan$  $nan$  & $nan$  $nan$  $nan$  \\ \hline 
3.000e+00 & $2.592e+02$  $1.449e+02$  $1.883e+03$  & $nan$  $nan$  $nan$  & $nan$  $nan$  $nan$  \\ \hline 

\end{tabular}\\[20pt]
\end{center}

\newpage
\begin{center}
Table of norms for H. $\mu = 0.0010$ \, $C = 10.0000$, $\gamma = 1.0000$
  
\begin{tabular}{|p{1in}|p{1in}|p{1in}|p{1in}|p{1in}|} \hline
$\tau / h$ &1.000e-01 &1.000e-02 &1.000e-03 &1.000e-04 \\ \hline 
1.000e-01 & $3.090e+04$  $1.439e+04$  $1.701e+05$  & $5.084e+06$  $6.602e+05$  $1.094e+08$  & $1.411e+08$  $3.608e+07$  $4.934e+10$  & $8.968e+11$  $5.488e+10$  $7.003e+14$  \\ \hline 
1.000e-02 & $3.100e+39$  $1.131e+39$  $5.512e+39$  & $4.683e+59$  $1.051e+59$  $2.281e+61$  & $1.080e+99$  $8.138e+97$  $1.285e+101$  & $2.012e+142$  $6.477e+140$  $9.644e+144$  \\ \hline 
1.000e-03 & $1.325e+156$  $inf$  $inf$  & $3.595e+221$  $inf$  $inf$  & $nan$  $-nan$  $-nan$  & $nan$  $-nan$  $-nan$  \\ \hline 
1.000e-04 & $6.715e+251$  $inf$  $inf$  & $2.001e-02$  $2.598e-03$  $2.318e-01$  & $nan$  $-nan$  $-nan$  & $nan$  $-nan$  $-nan$  \\ \hline 

\end{tabular}\\[20pt]
\end{center}

\newpage
\begin{center}
Table of norms for V. $\mu = 0.0010$ \, $C = 10.0000$, $\gamma = 1.0000$
  
\begin{tabular}{|p{1in}|p{1in}|p{1in}|p{1in}|p{1in}|} \hline
$\tau / h$ &1.000e-01 &1.000e-02 &1.000e-03 &1.000e-04 \\ \hline 
1.000e-01 & $1.346e+01$  $6.953e+00$  $9.841e+01$  & $1.000e+04$  $1.373e+03$  $2.189e+05$  & $5.955e+01$  $6.702e+00$  $8.298e+03$  & $1.435e+02$  $2.333e+01$  $1.245e+04$  \\ \hline 
1.000e-02 & $3.654e+01$  $2.468e+01$  $3.697e+02$  & $9.486e+01$  $1.691e+01$  $2.359e+03$  & $4.059e+04$  $2.829e+03$  $4.193e+06$  & $1.590e+03$  $7.636e+01$  $1.046e+06$  \\ \hline 
1.000e-03 & $5.806e+01$  $3.838e+01$  $5.856e+02$  & $6.277e+01$  $3.504e+01$  $4.747e+03$  & $nan$  $nan$  $nan$  & $nan$  $nan$  $nan$  \\ \hline 
1.000e-04 & $1.267e+02$  $8.832e+01$  $1.312e+03$  & $4.506e-03$  $2.344e-03$  $6.926e-02$  & $nan$  $nan$  $nan$  & $nan$  $nan$  $nan$  \\ \hline 

\end{tabular}\\[20pt]
\end{center}

\newpage
\begin{center}
Table of norms for H. $\mu = 0.0010$ \, $C = 1.0000$, $\gamma = 1.0000$
  
\begin{tabular}{|p{1in}|p{1in}|p{1in}|p{1in}|p{1in}|} \hline
$\tau / h$ &1.000e-01 &1.000e-02 &1.000e-03 &1.000e-04 \\ \hline 
1.000e-01 & $1.140e+02$  $4.643e+01$  $6.830e+02$  & $2.303e+03$  $4.776e+02$  $9.043e+04$  & $1.457e+02$  $9.078e+00$  $1.202e+04$  & $8.885e+00$  $4.294e+00$  $8.702e+02$  \\ \hline 
1.000e-02 & $3.773e+04$  $8.832e+03$  $2.074e+05$  & $2.865e+25$  $3.835e+24$  $7.577e+26$  & $5.438e+41$  $3.827e+40$  $6.606e+43$  & $2.910e+39$  $6.688e+37$  $1.054e+42$  \\ \hline 
1.000e-03 & $8.363e+43$  $1.969e+43$  $4.653e+44$  & $1.958e+00$  $1.584e-01$  $2.211e+01$  & $6.365e+257$  $inf$  $inf$  & $nan$  $-nan$  $-nan$  \\ \hline 
1.000e-04 & $5.700e+49$  $1.284e+49$  $1.965e+50$  & $3.963e-02$  $8.792e-03$  $1.213e+00$  & $1.290e-02$  $2.311e-03$  $2.981e-01$  & $1.253e-02$  $2.319e-03$  $2.996e-01$  \\ \hline 

\end{tabular}\\[20pt]
\end{center}

\newpage
\begin{center}
Table of norms for V. $\mu = 0.0010$ \, $C = 1.0000$, $\gamma = 1.0000$
  
\begin{tabular}{|p{1in}|p{1in}|p{1in}|p{1in}|p{1in}|} \hline
$\tau / h$ &1.000e-01 &1.000e-02 &1.000e-03 &1.000e-04 \\ \hline 
1.000e-01 & $6.763e+00$  $3.636e+00$  $4.417e+01$  & $1.806e+02$  $2.261e+01$  $2.501e+03$  & $4.581e+00$  $1.432e+00$  $5.801e+01$  & $6.133e+00$  $3.709e+00$  $3.239e+01$  \\ \hline 
1.000e-02 & $9.705e+00$  $4.596e+00$  $6.103e+01$  & $9.693e+01$  $1.074e+01$  $1.588e+03$  & $1.537e+03$  $8.712e+01$  $1.470e+05$  & $1.731e+03$  $3.155e+01$  $4.365e+05$  \\ \hline 
1.000e-03 & $2.396e+01$  $1.489e+01$  $2.089e+02$  & $3.431e-02$  $1.313e-02$  $1.777e+00$  & $1.712e+02$  $7.741e+00$  $1.111e+04$  & $nan$  $-nan$  $-nan$  \\ \hline 
1.000e-04 & $7.905e+01$  $4.418e+01$  $7.942e+02$  & $1.048e-02$  $2.817e-03$  $2.426e-01$  & $2.261e-03$  $6.061e-04$  $7.027e-02$  & $2.311e-03$  $6.073e-04$  $7.104e-02$  \\ \hline 

\end{tabular}\\[20pt]
\end{center}

\newpage
\begin{center}
Table of norms for H. $\mu = 0.0010$ \, $C = 1.0000$, $\gamma = 1.4000$
  
\begin{tabular}{|p{1in}|p{1in}|p{1in}|p{1in}|p{1in}|} \hline
$\tau / h$ &1.000e-01 &1.000e-02 &1.000e-03 &1.000e-04 \\ \hline 
1.000e-01 & $nan$  $-nan$  $-nan$  & $nan$  $-nan$  $-nan$  & $nan$  $-nan$  $-nan$  & $nan$  $-nan$  $-nan$  \\ \hline 
1.000e-02 & $nan$  $-nan$  $-nan$  & $nan$  $-nan$  $-nan$  & $nan$  $-nan$  $-nan$  & $nan$  $-nan$  $-nan$  \\ \hline 
1.000e-03 & $nan$  $-nan$  $-nan$  & $nan$  $-nan$  $-nan$  & $nan$  $-nan$  $-nan$  & $nan$  $-nan$  $-nan$  \\ \hline 
1.000e-04 & $nan$  $-nan$  $-nan$  & $1.498e-02$  $4.873e-03$  $5.353e-01$  & $6.867e-04$  $2.722e-04$  $5.344e-03$  & $nan$  $-nan$  $-nan$  \\ \hline 

\end{tabular}\\[20pt]
\end{center}

\newpage
\begin{center}
Table of norms for V. $\mu = 0.0010$ \, $C = 1.0000$, $\gamma = 1.4000$
  
\begin{tabular}{|p{1in}|p{1in}|p{1in}|p{1in}|p{1in}|} \hline
$\tau / h$ &1.000e-01 &1.000e-02 &1.000e-03 &1.000e-04 \\ \hline 
1.000e-01 & $nan$  $nan$  $nan$  & $nan$  $-nan$  $-nan$  & $nan$  $-nan$  $-nan$  & $nan$  $-nan$  $-nan$  \\ \hline 
1.000e-02 & $nan$  $nan$  $nan$  & $nan$  $-nan$  $-nan$  & $nan$  $nan$  $nan$  & $nan$  $nan$  $nan$  \\ \hline 
1.000e-03 & $nan$  $nan$  $nan$  & $nan$  $-nan$  $-nan$  & $nan$  $-nan$  $-nan$  & $nan$  $-nan$  $-nan$  \\ \hline 
1.000e-04 & $nan$  $-nan$  $-nan$  & $7.856e-03$  $2.837e-03$  $2.038e-01$  & $2.161e-04$  $1.035e-04$  $1.978e-03$  & $nan$  $nan$  $nan$  \\ \hline 

\end{tabular}\\[20pt]
\end{center}

\newpage
\begin{center}
Table of norms for H. $\mu = 0.0100$ \, $C = 100.0000$, $\gamma = 1.0000$
  
\begin{tabular}{|p{1in}|p{1in}|p{1in}|p{1in}|p{1in}|} \hline
$k / \tau = h$ &1.000e-01 &1.000e-02 &1.000e-03 &1.000e-04 \\ \hline 
0.000e+00 & $1.989e+10$  $8.497e+09$  $1.238e+11$  & $7.628e+80$  $1.226e+80$  $1.901e+82$  & $nan$  $-nan$  $-nan$  & $nan$  $-nan$  $-nan$  \\ \hline 
1.000e+00 & $6.687e+34$  $7.942e+33$  $7.276e+35$  & $nan$  $-nan$  $-nan$  & $nan$  $-nan$  $-nan$  & $nan$  $-nan$  $-nan$  \\ \hline 
2.000e+00 & $5.651e+66$  $9.826e+65$  $3.858e+67$  & $nan$  $-nan$  $-nan$  & $nan$  $-nan$  $-nan$  & $nan$  $-nan$  $-nan$  \\ \hline 
3.000e+00 & $1.141e+141$  $6.398e+139$  $1.510e+142$  & $nan$  $-nan$  $-nan$  & $nan$  $-nan$  $-nan$  & $nan$  $-nan$  $-nan$  \\ \hline 

\end{tabular}\\[20pt]
\end{center}

\newpage
\begin{center}
Table of norms for V. $\mu = 0.0100$ \, $C = 100.0000$, $\gamma = 1.0000$
  
\begin{tabular}{|p{1in}|p{1in}|p{1in}|p{1in}|} \hline
$k / \tau = h$ &1.000e-01 &1.000e-02 &1.000e-03 \\ \hline 
0.000e+00 & $7.861e+02$  $2.878e+02$  $4.036e+03$  & $7.356e+02$  $1.340e+02$  $2.280e+04$  & $nan$  $-nan$  $-nan$  \\ \hline 
1.000e+00 & $1.426e+02$  $5.667e+01$  $6.430e+02$  & $nan$  $nan$  $nan$  & $nan$  $nan$  $nan$  \\ \hline 
2.000e+00 & $3.310e+04$  $1.047e+04$  $1.824e+05$  & $nan$  $nan$  $nan$  & $nan$  $nan$  $nan$  \\ \hline 
3.000e+00 & $1.090e+02$  $5.249e+01$  $6.161e+02$  & $nan$  $nan$  $nan$  & $nan$  $nan$  $nan$  \\ \hline 

\end{tabular}\\[20pt]
\end{center}

\newpage
\begin{center}
Table of norms for H. $\mu = 0.0100$ \, $C = 10.0000$, $\gamma = 1.0000$
  
\begin{tabular}{|p{1in}|p{1in}|p{1in}|p{1in}|p{1in}|} \hline
$\tau / h$ &1.000e-01 &1.000e-02 &1.000e-03 &1.000e-04 \\ \hline 
1.000e-01 & $2.542e+05$  $1.075e+05$  $2.196e+06$  & $1.920e+03$  $3.960e+02$  $7.949e+04$  & $1.463e+03$  $4.806e+01$  $8.785e+04$  & $2.767e+02$  $1.104e+01$  $2.032e+05$  \\ \hline 
1.000e-02 & $2.070e+21$  $1.228e+21$  $2.673e+22$  & $6.209e+57$  $7.450e+56$  $1.364e+59$  & $1.553e+59$  $1.705e+58$  $2.382e+61$  & $1.167e+15$  $5.468e+13$  $7.926e+17$  \\ \hline 
1.000e-03 & $3.535e+73$  $8.893e+72$  $1.600e+74$  & $6.788e+146$  $9.242e+145$  $2.193e+148$  & $nan$  $-nan$  $-nan$  & $nan$  $-nan$  $-nan$  \\ \hline 
1.000e-04 & $2.105e+242$  $inf$  $inf$  & $3.099e-03$  $1.747e-03$  $2.052e-02$  & $4.080e-04$  $1.875e-04$  $1.195e-03$  & $4.225e-04$  $1.939e-04$  $1.259e-03$  \\ \hline 

\end{tabular}\\[20pt]
\end{center}

\newpage
\begin{center}
Table of norms for H. $\mu = 0.0100$ \, $C = 10.0000$, $\gamma = 1.0000$
  
\begin{tabular}{|p{1in}|p{1in}|p{1in}|p{1in}|p{1in}|} \hline
$\tau / h$ &1.000e-01 &1.000e-02 &1.000e-03 &1.000e-04 \\ \hline 
1.000e-01 & $2.159e+01$  $9.587e+00$  $1.198e+02$  & $2.471e+02$  $3.984e+01$  $4.609e+03$  & $2.133e+05$  $8.964e+03$  $1.237e+07$  & $9.071e+06$  $9.628e+04$  $1.372e+09$  \\ \hline 
1.000e-02 & $3.757e+01$  $1.960e+01$  $2.769e+02$  & $7.268e+03$  $7.591e+02$  $1.296e+05$  & $4.183e+03$  $2.453e+02$  $3.613e+05$  & $1.706e+04$  $5.245e+02$  $7.946e+06$  \\ \hline 
1.000e-03 & $8.250e+01$  $5.484e+01$  $9.331e+02$  & $5.382e+01$  $2.538e+01$  $3.733e+03$  & $nan$  $-nan$  $-nan$  & $nan$  $-nan$  $-nan$  \\ \hline 
1.000e-04 & $6.557e+01$  $4.134e+01$  $6.523e+02$  & $4.257e-03$  $2.137e-03$  $3.011e-02$  & $nan$  $-nan$  $-nan$  & $nan$  $-nan$  $-nan$  \\ \hline 

\end{tabular}\\[20pt]
\end{center}

\newpage
\begin{center}
Table of norms for H. $\mu = 0.0100$ \, $C = 1.0000$, $\gamma = 1.0000$
  
\begin{tabular}{|p{1in}|p{1in}|p{1in}|p{1in}|p{1in}|} \hline
$k / \tau = h$ &1.000e-01 &1.000e-02 &1.000e-03 &1.000e-04 \\ \hline 
0.000e+00 & $6.976e+01$  $3.338e+01$  $6.133e+02$  & $1.138e+36$  $1.241e+35$  $2.314e+37$  & $nan$  $-nan$  $-nan$  & $nan$  $-nan$  $-nan$  \\ \hline 
1.000e+00 & $2.047e+09$  $2.297e+08$  $1.740e+10$  & $1.040e+154$  $3.849e+152$  $inf$  & $nan$  $-nan$  $-nan$  & $nan$  $-nan$  $-nan$  \\ \hline 
2.000e+00 & $1.520e+18$  $1.829e+17$  $1.101e+19$  & $2.689e+271$  $inf$  $inf$  & $nan$  $-nan$  $-nan$  & $nan$  $-nan$  $-nan$  \\ \hline 
3.000e+00 & $3.045e+39$  $1.702e+38$  $3.889e+40$  & $nan$  $-nan$  $-nan$  & $nan$  $-nan$  $-nan$  & $nan$  $-nan$  $-nan$  \\ \hline 

\end{tabular}\\[20pt]
\end{center}

\newpage
\begin{center}
Table of norms for H. $\mu = 0.0100$ \, $C = 1.0000$, $\gamma = 1.0000$
  
\begin{tabular}{|p{1in}|p{1in}|p{1in}|p{1in}|p{1in}|} \hline
$\tau / h$ &1.000e-01 &1.000e-02 &1.000e-03 &1.000e-04 \\ \hline 
1.000e-01 & $6.402e+00$  $3.503e+00$  $6.085e+01$  & $1.104e+02$  $1.829e+01$  $2.648e+03$  & $5.696e+07$  $2.433e+06$  $3.442e+09$  & $4.113e+07$  $1.169e+06$  $1.652e+10$  \\ \hline 
1.000e-02 & $1.294e+01$  $6.238e+00$  $1.109e+02$  & $3.998e+01$  $7.511e+00$  $1.111e+03$  & $1.491e+03$  $9.452e+01$  $1.439e+05$  & $nan$  $-nan$  $-nan$  \\ \hline 
1.000e-03 & $1.213e+01$  $7.867e+00$  $1.109e+02$  & $1.218e-02$  $4.318e-03$  $1.627e-01$  & $nan$  $nan$  $nan$  & $nan$  $nan$  $nan$  \\ \hline 
1.000e-04 & $1.310e+01$  $8.536e+00$  $1.386e+02$  & $5.115e-03$  $2.362e-03$  $4.356e-02$  & $nan$  $-nan$  $-nan$  & $nan$  $nan$  $nan$  \\ \hline 

\end{tabular}\\[20pt]
\end{center}

\newpage
\begin{center}
Table of norms for H. $\mu = 0.0100$ \, $C = 1.0000$, $\gamma = 1.4000$
  
\begin{tabular}{|p{0.8in}|p{0.8in}|p{0.8in}|p{0.8in}|p{0.8in}|p{0.8in}|p{0.8in}|} \hline
$K$ &$N_0$ &$N_0 \tau$ &$n = \frac{N_0}{4}$ &$n = \frac{N_0}{2}$ &$n = \frac{3N_0}{4}$ &$n = N_0$ \\ \hline 
0 & & & & & & \\ \hline 
1 & & & & & & \\ \hline 
2 & & & & & & \\ \hline 
3 & & & & & & \\ \hline 

\end{tabular}\\[20pt]
\end{center}

\newpage
\begin{center}
Table of norms for V. $\mu = 0.0100$ \, $C = 1.0000$, $\gamma = 1.4000$
  
\begin{tabular}{|p{0.8in}|p{0.8in}|p{0.8in}|p{0.8in}|p{0.8in}|p{0.8in}|p{0.8in}|} \hline
$K$ &$N_0$ &$N_0 \tau$ &$n = \frac{N_0}{4}$ &$n = \frac{N_0}{2}$ &$n = \frac{3N_0}{4}$ &$n = N_0$ \\ \hline 
0 &919330 &9.193e+02 &1.980e-02 &8.076e-03 &6.651e-03 &9.998e-04 \\ \hline 
1 &1838659 &9.193e+02 &1.891e-02 &8.395e-03 &5.863e-03 &1.467e-03 \\ \hline 
2 &3677317 &9.193e+02 &1.853e-02 &8.487e-03 &5.518e-03 &1.632e-03 \\ \hline 
3 &7354633 &9.193e+02 &1.835e-02 &8.521e-03 &5.358e-03 &1.701e-03 \\ \hline 

\end{tabular}\\[20pt]
\end{center}

\newpage
\begin{center}
Table of norms for H. $\mu = 0.1000$ \, $C = 100.0000$, $\gamma = 1.0000$
  
\begin{tabular}{|p{1in}|p{1in}|p{1in}|p{1in}|p{1in}|} \hline
$\tau / h$ &1.000e-01 &1.000e-02 &1.000e-03 &1.000e-04 \\ \hline 
1.000e-01 & $1.092e+13$  $7.623e+12$  $1.475e+14$  & $2.732e+09$  $7.216e+08$  $1.095e+11$  & $3.079e+13$  $3.724e+12$  $5.262e+15$  & $4.809e+18$  $2.406e+18$  $3.401e+22$  \\ \hline 
1.000e-02 & $3.692e+57$  $1.636e+57$  $4.589e+57$  & $6.120e+95$  $7.712e+94$  $1.331e+97$  & $4.267e+125$  $1.511e+124$  $2.109e+127$  & $1.905e+160$  $inf$  $inf$  \\ \hline 
1.000e-03 & $1.727e+182$  $inf$  $inf$  & $nan$  $-nan$  $-nan$  & $nan$  $-nan$  $-nan$  & $nan$  $-nan$  $-nan$  \\ \hline 
1.000e-04 & $6.774e+272$  $inf$  $inf$  & $3.805e-03$  $2.345e-03$  $2.391e-02$  & $nan$  $-nan$  $-nan$  & $nan$  $-nan$  $-nan$  \\ \hline 

\end{tabular}\\[20pt]
\end{center}

\newpage
\begin{center}
Table of norms for H. $\mu = 0.1000$ \, $C = 100.0000$, $\gamma = 1.0000$
  
\begin{tabular}{|p{1in}|p{1in}|p{1in}|p{1in}|p{1in}|} \hline
$\tau / h$ &1.000e-01 &1.000e-02 &1.000e-03 &1.000e-04 \\ \hline 
1.000e-01 & $1.396e+02$  $7.026e+01$  $1.006e+03$  & $6.052e+04$  $1.539e+04$  $2.373e+06$  & $4.018e+03$  $3.036e+02$  $4.364e+05$  & $3.551e+06$  $4.198e+04$  $5.933e+08$  \\ \hline 
1.000e-02 & $7.060e+01$  $4.460e+01$  $6.486e+02$  & $1.619e+03$  $2.911e+02$  $4.415e+04$  & $4.477e+03$  $6.658e+02$  $9.019e+05$  & $1.287e+05$  $4.590e+03$  $6.990e+07$  \\ \hline 
1.000e-03 & $3.685e+02$  $2.594e+02$  $3.850e+03$  & $nan$  $-nan$  $-nan$  & $nan$  $-nan$  $-nan$  & $nan$  $-nan$  $-nan$  \\ \hline 
1.000e-04 & $3.042e+02$  $2.134e+02$  $3.178e+03$  & $3.322e-03$  $1.908e-03$  $2.599e-02$  & $nan$  $-nan$  $-nan$  & $nan$  $-nan$  $-nan$  \\ \hline 

\end{tabular}\\[20pt]
\end{center}

\newpage
\begin{center}
Table of norms for H. $\mu = 0.1000$ \, $C = 10.0000$, $\gamma = 1.0000$
  
\begin{tabular}{|p{1in}|p{1in}|p{1in}|p{1in}|} \hline
$k / \tau = h$ &1.000e-01 &1.000e-02 &1.000e-03 \\ \hline 
0.000e+00 & $1.098e+06$  $2.630e+05$  $3.775e+06$  & $1.780e+07$  $2.985e+06$  $5.849e+08$  & $4.464e-03$  $2.043e-03$  $1.222e-02$  \\ \hline 
1.000e+00 & $2.667e+04$  $1.528e+04$  $1.183e+05$  & $4.318e+09$  $5.246e+08$  $3.967e+10$  & $2.217e-03$  $1.016e-03$  $6.080e-03$  \\ \hline 
2.000e+00 & $1.591e+13$  $3.559e+12$  $5.136e+13$  & $1.146e-02$  $5.196e-03$  $3.122e-02$  & $1.105e-03$  $5.069e-04$  $3.033e-03$  \\ \hline 
3.000e+00 & $1.794e+13$  $4.216e+12$  $6.360e+13$  & $5.599e-03$  $2.561e-03$  $1.530e-02$  & $5.517e-04$  $2.532e-04$  $1.515e-03$  \\ \hline 
4.000e+00 & $1.262e+13$  $4.458e+12$  $1.937e+13$  & $2.775e-03$  $1.272e-03$  $7.603e-03$  & $2.756e-04$  $1.265e-04$  $7.572e-04$  \\ \hline 

\end{tabular}\\[20pt]
\end{center}

\newpage
\begin{center}
Table of norms for V. $\mu = 0.1000$ \, $C = 10.0000$, $\gamma = 1.0000$
  
\begin{tabular}{|p{0.8in}|p{0.8in}|p{0.8in}|p{0.8in}|p{0.8in}|p{0.8in}|p{0.8in}|} \hline
$K$ &$N_0$ &$N_0 \tau$ &$n = \frac{N_0}{4}$ &$n = \frac{N_0}{2}$ &$n = \frac{3N_0}{4}$ &$n = N_0$ \\ \hline 
0 &163946 &1.639e+02 &1.087e-01 &3.157e-02 &8.669e-03 &9.950e-04 \\ \hline 
1 &327891 &1.639e+02 &1.041e-01 &2.907e-02 &7.359e-03 &9.549e-04 \\ \hline 
2 &655781 &1.639e+02 &1.018e-01 &2.785e-02 &6.758e-03 &9.451e-04 \\ \hline 
3 &1311561 &1.639e+02 &1.007e-01 &2.724e-02 &6.465e-03 &9.426e-04 \\ \hline 

\end{tabular}\\[20pt]
\end{center}

\newpage
\begin{center}
Table of norms for H. $\mu = 0.1000$ \, $C = 1.0000$, $\gamma = 1.0000$
  
\begin{tabular}{|p{1in}|p{1in}|p{1in}|p{1in}|} \hline
$k / \tau = h$ &1.000e-01 &1.000e-02 &1.000e-03 \\ \hline 
0.000e+00 & $6.007e+02$  $3.064e+02$  $7.033e+03$  & $1.568e+00$  $4.277e-01$  $2.557e+01$  & $7.259e-02$  $1.972e-02$  $4.754e-01$  \\ \hline 
1.000e+00 & $9.374e+03$  $2.131e+03$  $3.814e+04$  & $4.478e-01$  $1.176e-01$  $3.736e+00$  & $3.541e-02$  $9.683e-03$  $2.289e-01$  \\ \hline 
2.000e+00 & $3.181e+03$  $7.419e+02$  $8.624e+03$  & $1.947e-01$  $5.225e-02$  $1.339e+00$  & $1.749e-02$  $4.799e-03$  $1.124e-01$  \\ \hline 
3.000e+00 & $1.757e+00$  $8.537e-01$  $1.009e+01$  & $9.129e-02$  $2.488e-02$  $5.997e-01$  & $8.691e-03$  $2.389e-03$  $5.570e-02$  \\ \hline 
4.000e+00 & $5.943e-01$  $1.971e-01$  $2.880e+00$  & $4.440e-02$  $1.216e-02$  $2.859e-01$  & $4.332e-03$  $1.192e-03$  $2.773e-02$  \\ \hline 

\end{tabular}\\[20pt]
\end{center}

\newpage
\begin{center}
Table of norms for V. $\mu = 0.1000$ \, $C = 1.0000$, $\gamma = 1.0000$
  
\begin{tabular}{|p{1in}|p{1in}|p{1in}|p{1in}|p{1in}|} \hline
$\tau / h$ &1.000e-01 &1.000e-02 &1.000e-03 &1.000e-04 \\ \hline 
1.000e-01 & $5.855e+00$  $3.789e+00$  $6.347e+01$  & $2.239e+00$  $1.058e+00$  $3.760e+01$  & $1.067e+01$  $5.882e+00$  $2.601e+01$  & $1.850e+00$  $9.454e-01$  $1.013e+01$  \\ \hline 
1.000e-02 & $9.409e+00$  $5.648e+00$  $8.658e+01$  & $2.318e-01$  $7.808e-02$  $1.927e+00$  & $2.597e-01$  $8.324e-02$  $2.107e+00$  & $2.676e-01$  $8.564e-02$  $2.166e+00$  \\ \hline 
1.000e-03 & $2.178e+01$  $1.149e+01$  $1.968e+02$  & $1.818e-02$  $6.728e-03$  $1.157e-01$  & $1.392e-02$  $5.080e-03$  $9.453e-02$  & $1.388e-02$  $5.066e-03$  $9.436e-02$  \\ \hline 
1.000e-04 & $4.808e+01$  $3.105e+01$  $4.787e+02$  & $5.519e-03$  $2.421e-03$  $3.727e-02$  & $1.376e-03$  $5.080e-04$  $9.219e-03$  & $1.337e-03$  $4.946e-04$  $9.061e-03$  \\ \hline 

\end{tabular}\\[20pt]
\end{center}

\newpage
\begin{center}
Table of norms for H. $\mu = 0.1000$ \, $C = 1.0000$, $\gamma = 1.4000$
  
\begin{tabular}{|p{1in}|p{1in}|p{1in}|p{1in}|} \hline
$k / \tau = h$ &1.000e-01 &1.000e-02 &1.000e-03 \\ \hline 
0.000e+00 & $nan$  $nan$  $nan$  & $1.685e+00$  $2.629e-01$  $1.310e+01$  & $1.074e-02$  $4.336e-03$  $5.749e-02$  \\ \hline 
1.000e+00 & $nan$  $-nan$  $-nan$  & $6.222e-02$  $2.425e-02$  $3.757e-01$  & $5.250e-03$  $2.136e-03$  $2.840e-02$  \\ \hline 
2.000e+00 & $nan$  $nan$  $nan$  & $2.830e-02$  $1.129e-02$  $1.477e-01$  & $2.595e-03$  $1.060e-03$  $1.411e-02$  \\ \hline 
3.000e+00 & $nan$  $nan$  $nan$  & $1.355e-02$  $5.459e-03$  $7.179e-02$  & $1.290e-03$  $5.281e-04$  $7.035e-03$  \\ \hline 
4.000e+00 & $9.378e-02$  $3.781e-02$  $3.758e-01$  & $6.579e-03$  $2.680e-03$  $3.536e-02$  & $6.431e-04$  $2.635e-04$  $3.512e-03$  \\ \hline 

\end{tabular}\\[20pt]
\end{center}

\newpage
\begin{center}
Table of norms for V. $\mu = 0.1000$ \, $C = 1.0000$, $\gamma = 1.4000$
  
\begin{tabular}{|p{1in}|p{1in}|p{1in}|p{1in}|p{1in}|} \hline
$\tau / h$ &1.000e-01 &1.000e-02 &1.000e-03 &1.000e-04 \\ \hline 
1.000e-01 & $nan$  $nan$  $nan$  & $nan$  $-nan$  $-nan$  & $nan$  $-nan$  $-nan$  & $nan$  $-nan$  $-nan$  \\ \hline 
1.000e-02 & $nan$  $-nan$  $-nan$  & $1.079e-01$  $4.405e-02$  $1.733e+00$  & $8.880e-02$  $2.896e-02$  $1.526e+00$  & $8.242e-02$  $2.863e-02$  $1.415e+00$  \\ \hline 
1.000e-03 & $nan$  $nan$  $nan$  & $6.253e-03$  $3.009e-03$  $3.736e-02$  & $3.227e-03$  $1.517e-03$  $2.056e-02$  & $3.228e-03$  $1.507e-03$  $2.050e-02$  \\ \hline 
1.000e-04 & $nan$  $nan$  $nan$  & $3.758e-03$  $1.963e-03$  $2.731e-02$  & $3.241e-04$  $1.593e-04$  $2.093e-03$  & $3.202e-04$  $1.480e-04$  $2.016e-03$  \\ \hline 

\end{tabular}\\[20pt]
\end{center}


\subsection{Выводы}
По результатам численного эксперимента со вложенными сетками можно увидеть оценку точности снизу для схемы для случаев, когда схема сходится. Наименьшая ошибка имеет порядок $10^{-5}$.