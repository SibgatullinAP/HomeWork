\section{Разностная схема}
\subsection{Описание схемы}

Для поиска численного решения задачи $(1)$ можно использовать разностную схему, в которой при апроксимации конвективных членов используются центральные разности, но апроксимация не всех производных вынесена на верхний временной слой.

\begin{equation}
	\begin{array}{lc}
		H_t + 0.5(V\hat{H}_{\mathring{x}} + (V\hat{H})_{\mathring{x}} + HV_{\mathring{x}}) = 0, w \in \omega_h\\
		H_{t,0} + 0.5((V\hat{H})_{x,0} + H_0V_{x,0}) - 0.5h((HV)_{x\bar{x},1} - 0.5(HV)_{x\bar{x},2} + \\
		+ H_0(V_{x\bar{x},1} - 0.5 V_{x\bar{x},2})) = 0, x \in \gamma_h^-\\
		H_{t,M} + 0.5((V\hat{H})_{\bar{x},M} + H_MV_{\bar{x},M}) + 0.5h((HV)_{x\bar{x},M-1} - 0.5(HV)_{x\bar{x},M-2} + \\
		+ H_M(V_{x\bar{x},M-1} - 0.5 V_{x\bar{x},M-2})) = 0, x \in \gamma_h^+\\
		V_t + \frac{1}{3}(V\hat{V}_{\mathring{x}} + (V\hat{V})_{\mathring{x}}) + \frac{p(H)_{\mathring{x}}}{H} = \tilde{\mu}\hat{V}_{x\bar{x}} - \left(\tilde{\mu}-\frac{\mu}{H}\right)V_{x\bar{x}} + f, x \in \omega_h
	\end{array}
\end{equation}



\subsection{Координатная запись}
Распишем схему приведенных выше обозначениях, и выделим коэффиценты при $H\,$ и $V$ на $n + 1$ временном слое:
1 уравнение:\\

$
H_t + 0.5(V\hat{H}_{\mathring{x}} + (V\hat{H})_{\mathring{x}} + HV_{\mathring{x}}) = 0
$\\

$
\frac{H^{n+1}_m - H^n_m}{\tau} + \frac{V(\hat{H}_{m+1}^n - \hat{H}_{m-1}^n)}{4h} + \frac{(V\hat{H})^n_{m+1} - (V\hat{H})^n_{m-1}}{4h} + \frac{H(V^n_{m+1} - V^n_{m-1})}{4h} = 0
$

$
H_{m-1}^{n+1}\left(-\frac{(V_m^n + V_{m-1}^n)}{4h}\right) + H_{m}^{n+1}\left(\frac{1}{\tau}\right) + H_{m+1}^{n+1}\left(\frac{V_m^n + V_{m+1}^n}{4h}\right) = H_m^n\left(\frac{1}{\tau} - \frac{(V_{m+1}^n - V_{m-1}^n)}{4h} \right)
$\\

2 уравнение:\\
$
H_{t,0} + 0.5((V\hat{H})_{x,0} + H_0V_{x,0}) - 0.5h((HV)_{x\bar{x},1} - 0.5(HV)_{x\bar{x},2} + \\
+ H_0(V_{x\bar{x},1} - 0.5 V_{x\bar{x},2})) = 0
$\\

$
\frac{H_0^{n+1} - H_0^n}{\tau} + 0.5\left(\frac{V_1^{n}H_1^{n+1} - V_0^nH_0^{n+1}}{h} + H_0^n\left(\frac{V_1^n-V_0^n}{h}\right)\right) - \\
- \frac{h}{2}\left(\frac{H_0^nV_0^n-2H_1^nV_1^n + H_2^nV_2^n}{h^2} -\frac{1}{2}\left(\frac{H_1^nV_1^n-2H_2^nV_2^n + H_3^nV_3^n}{h^2}\right)\right) -\\
- \frac{h}{2}\left(H_0\left(\frac{V_0^n-2V_1^n+V_2^n}{h^2} -\frac{1}{2}\left(\frac{V_1^n - 2V_2^n + V_3^n}{h^2}\right)\right)\right) = 0
$\\

$
H_0^{n+1}\left(\frac{1}{\tau} - \frac{V_0^n}{2h}\right) + H_1^{n+1}\left(\frac{V_1^n}{2h}\right) = \frac{H_0^n}{\tau} - \frac{H_0^n(V_1^n-V_0^n)}{2h} + \\
+ \frac{h}{2}\left(\frac{H_0^nV_0^n-2H_1^nV_1^n + H_2^nV_2^n}{h^2} -\frac{1}{2}\left(\frac{H_1^nV_1^n-2H_2^nV_2^n + H_3^nV_3^n}{h^2}\right)\right) +\\
+ \frac{h}{2}\left(H_0\left(\frac{V_0^n-2V_1^n+V_2^n}{h^2} -\frac{1}{2}\left(\frac{V_1^n - 2V_2^n + V_3^n}{h^2}\right)\right)\right) = 0
$\\

3 уравнение:\\
$
H_{t,M} + 0.5((V\hat{H})_{\bar{x},M} + H_MV_{\bar{x},M}) + 0.5h((HV)_{x\bar{x},M-1} - 0.5(HV)_{x\bar{x},M-2} + \\
+ H_M(V_{x\bar{x},M-1} - 0.5 V_{x\bar{x},M-2})) = 0
$\\

$
\frac{H_M^{n+1} - H_M^n}{\tau} + 0.5\left(\frac{V_M^{n}H_{M}^{n+1} - V_{M-1}^nH_{M-1}^{n+1}}{h} + H_M^n\left(\frac{V_{M}^n-V_{M-1}^n}{h}\right)\right) + \\
+ \frac{h}{2}\left(\frac{H_{M-2}^nV_{M-2}^n-2H_{M-1}^nV_{M-1}^n + H_{M}^nV_{M}^n}{h^2} -\frac{1}{2}\left(\frac{H_{M-3}^nV_{M-3}^n-2H_{M-2}^nV_{M-2}^n + H_{M-1}^nV_{M-1}^n}{h^2}\right)\right) +\\
+ \frac{h}{2}\left(H_M\left(\frac{V_{M-2}^n-2V_{M-1}^n+V_{M}^n}{h^2} -\frac{1}{2}\left(\frac{V_{M-3}^n - 2V_{M-2}^n + V_{M-1}^n}{h^2}\right)\right)\right)
$\\

$
H_M^{n+1}\left(\frac{1}{\tau} + \frac{V_M^n}{2h}\right) + H_{M-1}^{n+1}\left(-\frac{V_{M-1}^n}{2h}\right) = \frac{H_M^n}{\tau} - \frac{H_M^n(V_{M}^n-V_{M-1}^n)}{2h} - \\
- \frac{h}{2}\left(\frac{H_{M-2}^nV_{M-2}^n-2H_{M-1}^nV_{M-1}^n + H_{M}^nV_{M}^n}{h^2} -\frac{1}{2}\left(\frac{H_{M-3}^nV_{M-3}^n-2H_{M-2}^nV_{M-2}^n + H_{M-1}^nV_{M-1}^n}{h^2}\right)\right) -\\
- \frac{h}{2}\left(H_M\left(\frac{V_{M-2}^n-2V_{M-1}^n+V_{M}^n}{h^2} -\frac{1}{2}\left(\frac{V_{M-3}^n - 2V_{M-2}^n + V_{M-1}^n}{h^2}\right)\right)\right)
$\\

4 уравнение:\\
$
V_t + \frac{1}{3}(V\hat{V}_{\mathring{x}} + (V\hat{V})_{\mathring{x}}) + \frac{p(H)_{\mathring{x}}}{H} = \tilde{\mu}\hat{V}_{x\bar{x}} - \left(\tilde{\mu}-\frac{\mu}{H}\right)V_{x\bar{x}} + f
$\\

$
\frac{V_m^{n+1} - V_m^n}{\tau} + \frac{1}{3}\left(V_m^n \frac{V_{m+1}^{n+1} - V_{m-1}^{n+1}}{2h} + \frac{V_{m+1}^nV_{m+1}^{n+1} - V_{m-1}^nV_{m-1}^{n+1}}{2h}\right) + \\ + \frac{p(H)_{m+1}^n - p(H)_{m-1}^n}{2h H_m^n} - \tilde{\mu}\frac{V_{m-1}^{n+1} - 2V_m^{n+1} + V_{m+1}^{n+1}}{h^2} + \left(\tilde{\mu} - \frac{\mu}{H_m^n}\right)\frac{V_{m-1}^n - 2V_m^n + V_{m+1}^n}{h^2} - f_m^n= 0
$\\

$
V_{m-1}^{n+1}\left(-\frac{V_m^n + V_{m-1}^n}{6h} - \frac{\tilde{\mu}}{h^2}\right) + V_m^{n+1}\left(\frac{1}{\tau} + \frac{2\tilde{\mu}}{h^2}\right) + V_{m+1}^{n-1}\left(\frac{V_m^n + V_{m+1}^n}{6h} - \frac{\tilde{\mu}}{h^2}\right)  = \\
\frac{V_m^n}{\tau} - \frac{p(H)_{m+1}^n - p(H)_{m-1}^n}{2hH_m^n} - \left(\tilde{\mu}-\frac{\mu}{H_m^n}\right)\frac{V_{m-1}^n - 2V_m^n + V_{m+1}^n }{h^2} + f_m^n = 0
$

\newpage