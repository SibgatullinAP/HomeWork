\newpage
\section{Негладкие начальные данные}
\subsection{Постановка задачи}

Пусть $\Omega_x=[0;10]$, тогда для системы $(1)$ зададим две задачи, начальные условия которых, определяются следующим образом.

\begin{equation}
	\begin{cases}
		\begin{array}{l}
			\rho(0, x) = 1, x \notin [4.5;5.5], \\
			\rho(0, x) = 2, x \in [4.5;5.5], \\
			u(0, x) \equiv 0, x\in [0;10]\\
			u(t, 0) = u(t, 10) = 0
		\end{array}
	\end{cases}
\end{equation}

\begin{equation}
	\begin{cases}
		\begin{array}{l}
			\rho(0, x) = 1, x \in [0;10]\\
			u(0, x) = 0, x \notin [4.5;5.5], \\
			u(0, x) = 1, x \in [4.5;5.5], \\
			u(t, 0) = u(t, 10) = 0
		\end{array}
	\end{cases}
\end{equation}

Функции $f$ и $f_0$ из гладкой задачи положим равными $f \equiv f_0 \equiv 0$. Вычисление будем проводить до времени $T_{st}$, при котором решение выходит на стационар. Выходом на стационар будем считать выполнение условия:
$$
\Vert (H^n, V^n) - (\tilde{H^n}, \tilde{V^n}) \Vert_C \leq \varepsilon
$$

\subsection{Численные эксперименты}
В качестве измельчения сетки возьмем $\tau = 10^{-3} \, h = 10^{-2}$. Эксперементальным путем было подобрано значение $\varepsilon = 10^{-3}$. Ввиду того, что при $C = 100$ и $\mu = 0.001$ сходимость схемы плохая, данные параметры не будем расссматривать при расчете различных вариантов начальных условий.

\subsubsection{Первая негладкая задача}
Ниже приведены таблицы времени стабилизации в зависимости от разных параметров $\mu$, $C$, $\gamma$

За $k$ обозначим степень разбиения сетки, т.е в строках таблиц будут находится результаты вычислений с измельчениями по $T$ и $X$, $\frac{\tau}{2^k}$ и $\frac{h}{2^k}$ соотвественно.

\newpage
\begin{center}
Table of norms for H. $\mu = 0.0100$ \, $C = 10.0000$, $\gamma = 1.0000$
  
\begin{tabular}{|p{1in}|p{1in}|p{1in}|p{1in}|p{1in}|} \hline
$\tau / h$ &1.000e-01 &1.000e-02 &1.000e-03 &1.000e-04 \\ \hline 
1.000e-01 & $2.159e+01$  $9.587e+00$  $1.198e+02$  & $2.471e+02$  $3.984e+01$  $4.609e+03$  & $2.133e+05$  $8.967e+03$  $1.237e+07$  & $1.928e+08$  $3.463e+06$  $4.953e+10$  \\ \hline 
1.000e-02 & $3.757e+01$  $1.960e+01$  $2.769e+02$  & $7.268e+03$  $7.591e+02$  $1.296e+05$  & $1.227e+05$  $5.647e+03$  $6.299e+06$  & $3.534e+04$  $1.529e+03$  $2.238e+07$  \\ \hline 
1.000e-03 & $8.250e+01$  $5.484e+01$  $9.331e+02$  & $5.382e+01$  $2.538e+01$  $3.733e+03$  & $nan$  $-nan$  $-nan$  & $nan$  $nan$  $nan$  \\ \hline 
1.000e-04 & $6.557e+01$  $4.134e+01$  $6.523e+02$  & $4.257e-03$  $2.137e-03$  $3.011e-02$  & $nan$  $-nan$  $-nan$  & $nan$  $nan$  $nan$  \\ \hline 

\end{tabular}\\[20pt]
\end{center}

\begin{center}
Table of norms for V. $\mu = 0.0100$ \, $C = 1.0000$, $\gamma = 1.0000$
  
\begin{tabular}{|p{0.8in}|p{0.8in}|p{0.8in}|p{0.8in}|p{0.8in}|p{0.8in}|p{0.8in}|} \hline
$K$ &$N_0$ &$N_0 \tau$ &$n = \frac{N_0}{4}$ &$n = \frac{N_0}{2}$ &$n = \frac{3N_0}{4}$ &$n = N_0$ \\ \hline 
0 &943131 &9.431e+02 &3.344e-02 &1.147e-02 &4.332e-03 &9.996e-04 \\ \hline 
1 &1886261 &9.431e+02 &3.311e-02 &1.129e-02 &4.185e-03 &9.050e-04 \\ \hline 
2 &3772521 &9.431e+02 &3.293e-02 &1.120e-02 &4.114e-03 &8.613e-04 \\ \hline 
3 &7545041 &9.431e+02 &3.285e-02 &1.115e-02 &4.079e-03 &8.405e-04 \\ \hline 

\end{tabular}\\[20pt]
\end{center}

\begin{center}
Table of norms for V. $\mu = 0.0100$ \, $C = 1.0000$, $\gamma = 1.4000$
  
\begin{tabular}{|p{0.8in}|p{0.8in}|p{0.8in}|p{0.8in}|p{0.8in}|p{0.8in}|p{0.8in}|} \hline
$K$ &$N_0$ &$N_0 \tau$ &$n = \frac{N_0}{4}$ &$n = \frac{N_0}{2}$ &$n = \frac{3N_0}{4}$ &$n = N_0$ \\ \hline 
0 &919330 &9.193e+02 &1.980e-02 &8.076e-03 &6.651e-03 &9.998e-04 \\ \hline 
1 &1838659 &9.193e+02 &1.891e-02 &8.395e-03 &5.863e-03 &1.467e-03 \\ \hline 
2 &3677317 &9.193e+02 &1.853e-02 &8.487e-03 &5.518e-03 &1.632e-03 \\ \hline 
3 &7354633 &9.193e+02 &1.835e-02 &8.521e-03 &5.358e-03 &1.701e-03 \\ \hline 

\end{tabular}\\[20pt]
\end{center}


\newpage
\begin{center}
Table of norms for V. $\mu = 0.1000$ \, $C = 10.0000$, $\gamma = 1.0000$
  
\begin{tabular}{|p{1in}|p{1in}|p{1in}|p{1in}|p{1in}|} \hline
$\tau / h$ &1.000e-01 &1.000e-02 &1.000e-03 &1.000e-04 \\ \hline 
1.000e-01 & $2.738e+01$  $1.215e+01$  $1.757e+02$  & $1.947e+01$  $7.508e+00$  $1.838e+02$  & $2.007e+01$  $1.102e+01$  $1.134e+02$  & $3.650e+00$  $2.328e+00$  $2.934e+01$  \\ \hline 
1.000e-02 & $2.981e+01$  $2.047e+01$  $3.420e+02$  & $2.449e+01$  $7.235e+00$  $9.146e+02$  & $5.059e+02$  $3.319e+01$  $4.690e+04$  & $2.571e+01$  $1.635e+01$  $6.742e+01$  \\ \hline 
1.000e-03 & $5.927e+00$  $2.487e+00$  $2.139e+01$  & $4.360e-03$  $2.429e-03$  $2.671e-02$  & $3.065e-03$  $1.361e-03$  $9.235e-03$  & $3.059e-03$  $1.358e-03$  $9.250e-03$  \\ \hline 
1.000e-04 & $2.395e+00$  $1.058e+00$  $8.296e+00$  & $3.182e-03$  $1.827e-03$  $2.550e-02$  & $3.125e-04$  $1.387e-04$  $9.413e-04$  & $3.066e-04$  $1.347e-04$  $9.258e-04$  \\ \hline 

\end{tabular}\\[20pt]
\end{center}

\begin{center}
Table of norms for V. $\mu = 0.1000$ \, $C = 1.0000$, $\gamma = 1.0000$
  
\begin{tabular}{|p{1in}|p{1in}|p{1in}|p{1in}|} \hline
$k / \tau = h$ &1.000e-01 &1.000e-02 &1.000e-03 \\ \hline 
0.000e+00 & $5.855e+00$  $3.789e+00$  $6.347e+01$  & $2.318e-01$  $7.808e-02$  $1.927e+00$  & $1.392e-02$  $5.080e-03$  $9.453e-02$  \\ \hline 
1.000e+00 & $6.710e+01$  $3.127e+01$  $2.341e+02$  & $3.701e-02$  $1.330e-02$  $2.537e-01$  & $3.365e-03$  $1.242e-03$  $2.281e-02$  \\ \hline 
2.000e+00 & $3.947e+00$  $2.138e+00$  $9.851e+00$  & $1.754e-02$  $6.397e-03$  $1.189e-01$  & $1.673e-03$  $6.188e-04$  $1.134e-02$  \\ \hline 
3.000e+00 & $9.862e-02$  $4.161e-02$  $5.053e-01$  & $8.544e-03$  $3.140e-03$  $5.773e-02$  & $8.341e-04$  $3.088e-04$  $5.653e-03$  \\ \hline 

\end{tabular}\\[20pt]
\end{center}

\begin{center}
Table of norms for V. $\mu = 0.1000$ \, $C = 1.0000$, $\gamma = 1.4000$
  
\begin{tabular}{|p{1in}|p{1in}|p{1in}|p{1in}|} \hline
$k / \tau = h$ &1.000e-01 &1.000e-02 &1.000e-03 \\ \hline 
0.000e+00 & $nan$  $-nan$  $-nan$  & $1.079e-01$  $4.405e-02$  $1.733e+00$  & $3.227e-03$  $1.517e-03$  $2.056e-02$  \\ \hline 
1.000e+00 & $nan$  $-nan$  $-nan$  & $8.190e-03$  $3.946e-03$  $5.295e-02$  & $8.015e-04$  $3.715e-04$  $5.057e-03$  \\ \hline 
2.000e+00 & $nan$  $-nan$  $-nan$  & $4.044e-03$  $1.909e-03$  $2.579e-02$  & $4.003e-04$  $1.851e-04$  $2.522e-03$  \\ \hline 
3.000e+00 & $2.354e-02$  $1.143e-02$  $1.210e-01$  & $2.010e-03$  $9.385e-04$  $1.272e-02$  & $2.001e-04$  $9.241e-05$  $1.259e-03$  \\ \hline 

\end{tabular}\\[20pt]
\end{center}


\newpage
Теперь рассмотрим динамику процесса на графиках затухания и временных срезах для различных значений параметров. 

\begin{figure}[H]
	\centering
	\includegraphics[scale=0.5]{../graphs_data_nonsmooth_1/value/Graph_H_mu0.010_C10.000_gamma1.000.png}
	\includegraphics[scale=0.5]{../graphs_data_nonsmooth_1/value/Graph_V_mu0.010_C10.000_gamma1.000.png}	
	\includegraphics[scale=0.5]{../graphs_data_nonsmooth_1/slices/Graph_H_mu0.010_C10.000_gamma1.000.png}
	\includegraphics[scale=0.5]{../graphs_data_nonsmooth_1/slices/Graph_V_mu0.010_C10.000_gamma1.000.png}
\end{figure}

\begin{figure}[H]
	\centering
	\includegraphics[scale=0.5]{../graphs_data_nonsmooth_1/value/Graph_H_mu0.010_C1.000_gamma1.000.png}
	\includegraphics[scale=0.5]{../graphs_data_nonsmooth_1/value/Graph_V_mu0.010_C1.000_gamma1.000.png}	
	\includegraphics[scale=0.5]{../graphs_data_nonsmooth_1/slices/Graph_H_mu0.010_C1.000_gamma1.000.png}
	\includegraphics[scale=0.5]{../graphs_data_nonsmooth_1/slices/Graph_V_mu0.010_C1.000_gamma1.000.png}
\end{figure}

\begin{figure}[H]
	\centering
	\includegraphics[scale=0.5]{../graphs_data_nonsmooth_1/value/Graph_H_mu0.010_C1.000_gamma1.400.png}
	\includegraphics[scale=0.5]{../graphs_data_nonsmooth_1/value/Graph_V_mu0.010_C1.000_gamma1.400.png}	
	\includegraphics[scale=0.5]{../graphs_data_nonsmooth_1/slices/Graph_H_mu0.010_C1.000_gamma1.400.png}
	\includegraphics[scale=0.5]{../graphs_data_nonsmooth_1/slices/Graph_V_mu0.010_C1.000_gamma1.400.png}
\end{figure}

\begin{figure}[H]
	\centering
	\includegraphics[scale=0.5]{../graphs_data_nonsmooth_1/value/Graph_H_mu0.100_C10.000_gamma1.000.png}
	\includegraphics[scale=0.5]{../graphs_data_nonsmooth_1/value/Graph_V_mu0.100_C10.000_gamma1.000.png}	
	\includegraphics[scale=0.5]{../graphs_data_nonsmooth_1/slices/Graph_H_mu0.100_C10.000_gamma1.000.png}
	\includegraphics[scale=0.5]{../graphs_data_nonsmooth_1/slices/Graph_V_mu0.100_C10.000_gamma1.000.png}
\end{figure}

\begin{figure}[H]
	\centering
	\includegraphics[scale=0.5]{../graphs_data_nonsmooth_1/value/Graph_H_mu0.100_C1.000_gamma1.000.png}
	\includegraphics[scale=0.5]{../graphs_data_nonsmooth_1/value/Graph_V_mu0.100_C1.000_gamma1.000.png}	
	\includegraphics[scale=0.5]{../graphs_data_nonsmooth_1/slices/Graph_H_mu0.100_C1.000_gamma1.000.png}
	\includegraphics[scale=0.5]{../graphs_data_nonsmooth_1/slices/Graph_V_mu0.100_C1.000_gamma1.000.png}
\end{figure}

\begin{figure}[H]
	\centering
	\includegraphics[scale=0.5]{../graphs_data_nonsmooth_1/value/Graph_H_mu0.100_C1.000_gamma1.400.png}
	\includegraphics[scale=0.5]{../graphs_data_nonsmooth_1/value/Graph_V_mu0.100_C1.000_gamma1.400.png}	
	\includegraphics[scale=0.5]{../graphs_data_nonsmooth_1/slices/Graph_H_mu0.100_C1.000_gamma1.400.png}
	\includegraphics[scale=0.5]{../graphs_data_nonsmooth_1/slices/Graph_V_mu0.100_C1.000_gamma1.400.png}
\end{figure}

Период колебаний не зависит от $\mu$, однако при увеличении $C$ и $\gamma$ частота колебаний увеличивается. Однако при уменьшении $\mu$ увеличивается время стабилизации процесса.


\newpage
Приведем также графики норм для различных начальных параметров.
\begin{figure}[H]
	\centering
	\includegraphics[scale=0.65]{../graphs_data_nonsmooth_1/norms/Graph_V_norms_mu0.100_C10.000_gamma1.000.png}
	\includegraphics[scale=0.65]{../graphs_data_nonsmooth_1/norms/Graph_V_norms_mu0.100_C1.000_gamma1.000.png}	
	\includegraphics[scale=0.65]{../graphs_data_nonsmooth_1/norms/Graph_V_norms_mu0.100_C1.000_gamma1.400.png}
\end{figure}


\begin{figure}[H]
	\centering
	\includegraphics[scale=0.65]{../graphs_data_nonsmooth_1/norms/Graph_V_norms_mu0.010_C10.000_gamma1.000.png}
	\includegraphics[scale=0.65]{../graphs_data_nonsmooth_1/norms/Graph_V_norms_mu0.010_C1.000_gamma1.000.png}	
	\includegraphics[scale=0.65]{../graphs_data_nonsmooth_1/norms/Graph_V_norms_mu0.010_C1.000_gamma1.400.png}
\end{figure}

\newpage
Далее приведены значения $\Delta m(n)$ на измельченных и обычных сетках. А также их графическое представление.

\begin{center}
Table of mass loss. $\mu = 0.1000$ \, $C = 10.0000$, $\gamma = 1.0000$
  
\begin{tabular}{|p{0.8in}|p{0.8in}|p{0.8in}|p{0.8in}|p{0.8in}|p{0.8in}|p{0.8in}|} \hline
$K$ &$N_0$ &$N_0 \tau$ &$n = \frac{N_0}{4}$ &$n = \frac{N_0}{2}$ &$n = \frac{3N_0}{4}$ &$n = N_0$ \\ \hline 
0 &163946 &1.639e+02 &2.222e-03 &2.221e-03 &2.215e-03 &2.212e-03 \\ \hline 
1 &327891 &1.639e+02 &1.033e-03 &1.032e-03 &1.029e-03 &1.027e-03 \\ \hline 
2 &655781 &1.639e+02 &4.993e-04 &4.983e-04 &4.968e-04 &4.959e-04 \\ \hline 
3 &1311561 &1.639e+02 &2.455e-04 &2.449e-04 &2.442e-04 &2.437e-04 \\ \hline 

\end{tabular}\\[20pt]
\end{center}

\begin{center}
Table of mass loss. $\mu = 0.1000$ \, $C = 1.0000$, $\gamma = 1.0000$
  
\begin{tabular}{|p{0.8in}|p{0.8in}|p{0.8in}|p{0.8in}|p{0.8in}|p{0.8in}|p{0.8in}|} \hline
$K$ &$N_0$ &$N_0 \tau$ &$n = \frac{N_0}{4}$ &$n = \frac{N_0}{2}$ &$n = \frac{3N_0}{4}$ &$n = N_0$ \\ \hline 
0 &414339 &4.143e+02 &2.568e-04 &2.822e-04 &2.735e-04 &2.757e-04 \\ \hline 
1 &828677 &4.143e+02 &1.273e-04 &1.399e-04 &1.357e-04 &1.367e-04 \\ \hline 
2 &1657353 &4.143e+02 &6.339e-05 &6.971e-05 &6.757e-05 &6.811e-05 \\ \hline 
3 &3314705 &4.143e+02 &3.164e-05 &3.479e-05 &3.373e-05 &3.400e-05 \\ \hline 

\end{tabular}\\[20pt]
\end{center}

\begin{center}
Table of mass loss. $\mu = 0.1000$ \, $C = 1.0000$, $\gamma = 1.4000$
  
\begin{tabular}{|p{0.8in}|p{0.8in}|p{0.8in}|p{0.8in}|p{0.8in}|p{0.8in}|p{0.8in}|} \hline
$K$ &$N_0$ &$N_0 \tau$ &$n = \frac{N_0}{4}$ &$n = \frac{N_0}{2}$ &$n = \frac{3N_0}{4}$ &$n = N_0$ \\ \hline 
0 &400813 &4.008e+02 &2.850e-04 &2.777e-04 &2.739e-04 &2.725e-04 \\ \hline 
1 &801625 &4.008e+02 &1.410e-04 &1.374e-04 &1.355e-04 &1.349e-04 \\ \hline 
2 &1603249 &4.008e+02 &7.019e-05 &6.838e-05 &6.743e-05 &6.710e-05 \\ \hline 
3 &3206497 &4.008e+02 &3.502e-05 &3.411e-05 &3.364e-05 &3.348e-05 \\ \hline 

\end{tabular}\\[20pt]
\end{center}

\begin{center}
Table of mass loss. $\mu = 0.0100$ \, $C = 10.0000$, $\gamma = 1.0000$
  
\begin{tabular}{|p{0.8in}|p{0.8in}|p{0.8in}|p{0.8in}|p{0.8in}|p{0.8in}|p{0.8in}|} \hline
$K$ &$N_0$ &$N_0 \tau$ &$n = \frac{N_0}{4}$ &$n = \frac{N_0}{2}$ &$n = \frac{3N_0}{4}$ &$n = N_0$ \\ \hline 
0 &5197350 &5.197e+03 &2.025e-01 &2.025e-01 &2.025e-01 &2.025e-01 \\ \hline 
1 &10394699 &5.197e+03 &2.631e-03 &2.632e-03 &2.632e-03 &2.632e-03 \\ \hline 
2 &20789397 &5.197e+03 &8.500e-04 &8.502e-04 &8.502e-04 &8.502e-04 \\ \hline 
3 &41578793 &5.197e+03 &3.625e-04 &3.625e-04 &3.626e-04 &3.625e-04 \\ \hline 

\end{tabular}\\[20pt]
\end{center}


\newpage
\begin{center}
Table of mass loss. $\mu = 0.0100$ \, $C = 1.0000$, $\gamma = 1.0000$
  
\begin{tabular}{|p{0.8in}|p{0.8in}|p{0.8in}|p{0.8in}|p{0.8in}|p{0.8in}|p{0.8in}|} \hline
$K$ &$N_0$ &$N_0 \tau$ &$n = \frac{N_0}{4}$ &$n = \frac{N_0}{2}$ &$n = \frac{3N_0}{4}$ &$n = N_0$ \\ \hline 
0 &943131 &9.431e+02 &2.208e-03 &2.204e-03 &2.224e-03 &2.215e-03 \\ \hline 
1 &1886261 &9.431e+02 &1.020e-03 &1.018e-03 &1.028e-03 &1.023e-03 \\ \hline 
2 &3772521 &9.431e+02 &4.918e-04 &4.909e-04 &4.956e-04 &4.935e-04 \\ \hline 
3 &7545041 &9.431e+02 &2.417e-04 &2.412e-04 &2.436e-04 &2.425e-04 \\ \hline 

\end{tabular}\\[20pt]
\end{center}

\begin{center}
Table of mass loss. $\mu = 0.0100$ \, $C = 1.0000$, $\gamma = 1.4000$
  
\begin{tabular}{|p{0.8in}|p{0.8in}|p{0.8in}|p{0.8in}|p{0.8in}|p{0.8in}|p{0.8in}|} \hline
$K$ &$N_0$ &$N_0 \tau$ &$n = \frac{N_0}{4}$ &$n = \frac{N_0}{2}$ &$n = \frac{3N_0}{4}$ &$n = N_0$ \\ \hline 
0 &919330 &9.193e+02 &3.906e-03 &3.902e-03 &3.904e-03 &3.911e-03 \\ \hline 
1 &1838659 &9.193e+02 &1.715e-03 &1.712e-03 &1.713e-03 &1.717e-03 \\ \hline 
2 &3677317 &9.193e+02 &8.101e-04 &8.087e-04 &8.089e-04 &8.109e-04 \\ \hline 
3 &7354633 &9.193e+02 &3.944e-04 &3.937e-04 &3.937e-04 &3.948e-04 \\ \hline 

\end{tabular}\\[20pt]
\end{center}


\newpage
\begin{figure}[H]
	\centering
	\includegraphics[scale=0.65]{../graphs_data_nonsmooth_1/mass/Graph_mass_mu0.100_C10.000_gamma1.000.png}
	\includegraphics[scale=0.65]{../graphs_data_nonsmooth_1/mass/Graph_mass_mu0.100_C1.000_gamma1.000.png}	
	\includegraphics[scale=0.65]{../graphs_data_nonsmooth_1/mass/Graph_mass_mu0.100_C1.000_gamma1.400.png}
\end{figure}


\begin{figure}[H]
	\centering
	\includegraphics[scale=0.65]{../graphs_data_nonsmooth_1/mass/Graph_mass_mu0.010_C10.000_gamma1.000.png}
	\includegraphics[scale=0.65]{../graphs_data_nonsmooth_1/mass/Graph_mass_mu0.010_C1.000_gamma1.000.png}	
	\includegraphics[scale=0.65]{../graphs_data_nonsmooth_1/mass/Graph_mass_mu0.010_C1.000_gamma1.400.png}
\end{figure}

Как можно заметить по таблицам потери массы составляют меньше 1-2\%, из чего можно заключить, что закон сохранения массы выполняется. \\

\newpage
\subsubsection{Вторая негладкая задача}
Ниже приведены таблицы времени стабилизации в зависимости от разных параметров $\mu$, $C$, $\gamma$

За $k$ обозначим степень разбиения сетки, т.е в строках таблиц будут находится результаты вычислений с измельчениями по $T$ и $X$, $\frac{\tau}{2^k}$ и $\frac{h}{2^k}$ соотвественно.

\begin{center}
Table of norms for H. $\mu = 0.0100$ \, $C = 10.0000$, $\gamma = 1.0000$
  
\begin{tabular}{|p{1in}|p{1in}|p{1in}|p{1in}|p{1in}|} \hline
$\tau / h$ &1.000e-01 &1.000e-02 &1.000e-03 &1.000e-04 \\ \hline 
1.000e-01 & $2.159e+01$  $9.587e+00$  $1.198e+02$  & $2.471e+02$  $3.984e+01$  $4.609e+03$  & $2.133e+05$  $8.967e+03$  $1.237e+07$  & $1.928e+08$  $3.463e+06$  $4.953e+10$  \\ \hline 
1.000e-02 & $3.757e+01$  $1.960e+01$  $2.769e+02$  & $7.268e+03$  $7.591e+02$  $1.296e+05$  & $1.227e+05$  $5.647e+03$  $6.299e+06$  & $3.534e+04$  $1.529e+03$  $2.238e+07$  \\ \hline 
1.000e-03 & $8.250e+01$  $5.484e+01$  $9.331e+02$  & $5.382e+01$  $2.538e+01$  $3.733e+03$  & $nan$  $-nan$  $-nan$  & $nan$  $nan$  $nan$  \\ \hline 
1.000e-04 & $6.557e+01$  $4.134e+01$  $6.523e+02$  & $4.257e-03$  $2.137e-03$  $3.011e-02$  & $nan$  $-nan$  $-nan$  & $nan$  $nan$  $nan$  \\ \hline 

\end{tabular}\\[20pt]
\end{center}

\begin{center}
Table of norms for V. $\mu = 0.0100$ \, $C = 1.0000$, $\gamma = 1.0000$
  
\begin{tabular}{|p{0.8in}|p{0.8in}|p{0.8in}|p{0.8in}|p{0.8in}|p{0.8in}|p{0.8in}|} \hline
$K$ &$N_0$ &$N_0 \tau$ &$n = \frac{N_0}{4}$ &$n = \frac{N_0}{2}$ &$n = \frac{3N_0}{4}$ &$n = N_0$ \\ \hline 
0 &943131 &9.431e+02 &3.344e-02 &1.147e-02 &4.332e-03 &9.996e-04 \\ \hline 
1 &1886261 &9.431e+02 &3.311e-02 &1.129e-02 &4.185e-03 &9.050e-04 \\ \hline 
2 &3772521 &9.431e+02 &3.293e-02 &1.120e-02 &4.114e-03 &8.613e-04 \\ \hline 
3 &7545041 &9.431e+02 &3.285e-02 &1.115e-02 &4.079e-03 &8.405e-04 \\ \hline 

\end{tabular}\\[20pt]
\end{center}

\begin{center}
Table of norms for V. $\mu = 0.0100$ \, $C = 1.0000$, $\gamma = 1.4000$
  
\begin{tabular}{|p{0.8in}|p{0.8in}|p{0.8in}|p{0.8in}|p{0.8in}|p{0.8in}|p{0.8in}|} \hline
$K$ &$N_0$ &$N_0 \tau$ &$n = \frac{N_0}{4}$ &$n = \frac{N_0}{2}$ &$n = \frac{3N_0}{4}$ &$n = N_0$ \\ \hline 
0 &919330 &9.193e+02 &1.980e-02 &8.076e-03 &6.651e-03 &9.998e-04 \\ \hline 
1 &1838659 &9.193e+02 &1.891e-02 &8.395e-03 &5.863e-03 &1.467e-03 \\ \hline 
2 &3677317 &9.193e+02 &1.853e-02 &8.487e-03 &5.518e-03 &1.632e-03 \\ \hline 
3 &7354633 &9.193e+02 &1.835e-02 &8.521e-03 &5.358e-03 &1.701e-03 \\ \hline 

\end{tabular}\\[20pt]
\end{center}


\newpage
\begin{center}
Table of norms for V. $\mu = 0.1000$ \, $C = 10.0000$, $\gamma = 1.0000$
  
\begin{tabular}{|p{1in}|p{1in}|p{1in}|p{1in}|p{1in}|} \hline
$\tau / h$ &1.000e-01 &1.000e-02 &1.000e-03 &1.000e-04 \\ \hline 
1.000e-01 & $2.738e+01$  $1.215e+01$  $1.757e+02$  & $1.947e+01$  $7.508e+00$  $1.838e+02$  & $2.007e+01$  $1.102e+01$  $1.134e+02$  & $3.650e+00$  $2.328e+00$  $2.934e+01$  \\ \hline 
1.000e-02 & $2.981e+01$  $2.047e+01$  $3.420e+02$  & $2.449e+01$  $7.235e+00$  $9.146e+02$  & $5.059e+02$  $3.319e+01$  $4.690e+04$  & $2.571e+01$  $1.635e+01$  $6.742e+01$  \\ \hline 
1.000e-03 & $5.927e+00$  $2.487e+00$  $2.139e+01$  & $4.360e-03$  $2.429e-03$  $2.671e-02$  & $3.065e-03$  $1.361e-03$  $9.235e-03$  & $3.059e-03$  $1.358e-03$  $9.250e-03$  \\ \hline 
1.000e-04 & $2.395e+00$  $1.058e+00$  $8.296e+00$  & $3.182e-03$  $1.827e-03$  $2.550e-02$  & $3.125e-04$  $1.387e-04$  $9.413e-04$  & $3.066e-04$  $1.347e-04$  $9.258e-04$  \\ \hline 

\end{tabular}\\[20pt]
\end{center}

\begin{center}
Table of norms for V. $\mu = 0.1000$ \, $C = 1.0000$, $\gamma = 1.0000$
  
\begin{tabular}{|p{1in}|p{1in}|p{1in}|p{1in}|} \hline
$k / \tau = h$ &1.000e-01 &1.000e-02 &1.000e-03 \\ \hline 
0.000e+00 & $5.855e+00$  $3.789e+00$  $6.347e+01$  & $2.318e-01$  $7.808e-02$  $1.927e+00$  & $1.392e-02$  $5.080e-03$  $9.453e-02$  \\ \hline 
1.000e+00 & $6.710e+01$  $3.127e+01$  $2.341e+02$  & $3.701e-02$  $1.330e-02$  $2.537e-01$  & $3.365e-03$  $1.242e-03$  $2.281e-02$  \\ \hline 
2.000e+00 & $3.947e+00$  $2.138e+00$  $9.851e+00$  & $1.754e-02$  $6.397e-03$  $1.189e-01$  & $1.673e-03$  $6.188e-04$  $1.134e-02$  \\ \hline 
3.000e+00 & $9.862e-02$  $4.161e-02$  $5.053e-01$  & $8.544e-03$  $3.140e-03$  $5.773e-02$  & $8.341e-04$  $3.088e-04$  $5.653e-03$  \\ \hline 

\end{tabular}\\[20pt]
\end{center}

\begin{center}
Table of norms for V. $\mu = 0.1000$ \, $C = 1.0000$, $\gamma = 1.4000$
  
\begin{tabular}{|p{1in}|p{1in}|p{1in}|p{1in}|} \hline
$k / \tau = h$ &1.000e-01 &1.000e-02 &1.000e-03 \\ \hline 
0.000e+00 & $nan$  $-nan$  $-nan$  & $1.079e-01$  $4.405e-02$  $1.733e+00$  & $3.227e-03$  $1.517e-03$  $2.056e-02$  \\ \hline 
1.000e+00 & $nan$  $-nan$  $-nan$  & $8.190e-03$  $3.946e-03$  $5.295e-02$  & $8.015e-04$  $3.715e-04$  $5.057e-03$  \\ \hline 
2.000e+00 & $nan$  $-nan$  $-nan$  & $4.044e-03$  $1.909e-03$  $2.579e-02$  & $4.003e-04$  $1.851e-04$  $2.522e-03$  \\ \hline 
3.000e+00 & $2.354e-02$  $1.143e-02$  $1.210e-01$  & $2.010e-03$  $9.385e-04$  $1.272e-02$  & $2.001e-04$  $9.241e-05$  $1.259e-03$  \\ \hline 

\end{tabular}\\[20pt]
\end{center}


\newpage
Теперь рассмотрим динамику процесса на графиках затухания и временных срезах для различных значений параметров. 

\begin{figure}[H]
	\centering
	\includegraphics[scale=0.5]{../graphs_data_nonsmooth_2/value/Graph_H_mu0.010_C10.000_gamma1.000.png}
	\includegraphics[scale=0.5]{../graphs_data_nonsmooth_2/value/Graph_V_mu0.010_C10.000_gamma1.000.png}	
	\includegraphics[scale=0.5]{../graphs_data_nonsmooth_2/slices/Graph_H_mu0.010_C10.000_gamma1.000.png}
	\includegraphics[scale=0.5]{../graphs_data_nonsmooth_2/slices/Graph_V_mu0.010_C10.000_gamma1.000.png}
\end{figure}

\begin{figure}[H]
	\centering
	\includegraphics[scale=0.5]{../graphs_data_nonsmooth_2/value/Graph_H_mu0.010_C1.000_gamma1.000.png}
	\includegraphics[scale=0.5]{../graphs_data_nonsmooth_2/value/Graph_V_mu0.010_C1.000_gamma1.000.png}	
	\includegraphics[scale=0.5]{../graphs_data_nonsmooth_2/slices/Graph_H_mu0.010_C1.000_gamma1.000.png}
	\includegraphics[scale=0.5]{../graphs_data_nonsmooth_2/slices/Graph_V_mu0.010_C1.000_gamma1.000.png}
\end{figure}

\begin{figure}[H]
	\centering
	\includegraphics[scale=0.5]{../graphs_data_nonsmooth_2/value/Graph_H_mu0.010_C1.000_gamma1.400.png}
	\includegraphics[scale=0.5]{../graphs_data_nonsmooth_2/value/Graph_V_mu0.010_C1.000_gamma1.400.png}	
	\includegraphics[scale=0.5]{../graphs_data_nonsmooth_2/slices/Graph_H_mu0.010_C1.000_gamma1.400.png}
	\includegraphics[scale=0.5]{../graphs_data_nonsmooth_2/slices/Graph_V_mu0.010_C1.000_gamma1.400.png}
\end{figure}

\begin{figure}[H]
	\centering
	\includegraphics[scale=0.5]{../graphs_data_nonsmooth_2/value/Graph_H_mu0.100_C10.000_gamma1.000.png}
	\includegraphics[scale=0.5]{../graphs_data_nonsmooth_2/value/Graph_V_mu0.100_C10.000_gamma1.000.png}	
	\includegraphics[scale=0.5]{../graphs_data_nonsmooth_2/slices/Graph_H_mu0.100_C10.000_gamma1.000.png}
	\includegraphics[scale=0.5]{../graphs_data_nonsmooth_2/slices/Graph_V_mu0.100_C10.000_gamma1.000.png}
\end{figure}

\begin{figure}[H]
	\centering
	\includegraphics[scale=0.5]{../graphs_data_nonsmooth_2/value/Graph_H_mu0.100_C1.000_gamma1.000.png}
	\includegraphics[scale=0.5]{../graphs_data_nonsmooth_2/value/Graph_V_mu0.100_C1.000_gamma1.000.png}	
	\includegraphics[scale=0.5]{../graphs_data_nonsmooth_2/slices/Graph_H_mu0.100_C1.000_gamma1.000.png}
	\includegraphics[scale=0.5]{../graphs_data_nonsmooth_2/slices/Graph_V_mu0.100_C1.000_gamma1.000.png}
\end{figure}

\begin{figure}[H]
	\centering
	\includegraphics[scale=0.5]{../graphs_data_nonsmooth_2/value/Graph_H_mu0.100_C1.000_gamma1.400.png}
	\includegraphics[scale=0.5]{../graphs_data_nonsmooth_2/value/Graph_V_mu0.100_C1.000_gamma1.400.png}	
	\includegraphics[scale=0.5]{../graphs_data_nonsmooth_2/slices/Graph_H_mu0.100_C1.000_gamma1.400.png}
	\includegraphics[scale=0.5]{../graphs_data_nonsmooth_2/slices/Graph_V_mu0.100_C1.000_gamma1.400.png}
\end{figure}

Период колебаний не зависит от $\mu$, однако при увеличении $C$ и $\gamma$ частота колебаний увеличивается. Однако при уменьшении $\mu$ увеличивается время стабилизации процесса.


\newpage
Приведем также графики невязок для различных параметров.
\begin{figure}[H]
	\centering
	\includegraphics[scale=0.65]{../graphs_data_nonsmooth_2/norms/Graph_V_norms_mu0.100_C10.000_gamma1.000.png}
	\includegraphics[scale=0.65]{../graphs_data_nonsmooth_2/norms/Graph_V_norms_mu0.100_C1.000_gamma1.000.png}	
	\includegraphics[scale=0.65]{../graphs_data_nonsmooth_2/norms/Graph_V_norms_mu0.100_C1.000_gamma1.400.png}
\end{figure}


\begin{figure}[H]
	\centering
	\includegraphics[scale=0.65]{../graphs_data_nonsmooth_2/norms/Graph_V_norms_mu0.010_C10.000_gamma1.000.png}
	\includegraphics[scale=0.65]{../graphs_data_nonsmooth_2/norms/Graph_V_norms_mu0.010_C1.000_gamma1.000.png}	
	\includegraphics[scale=0.65]{../graphs_data_nonsmooth_2/norms/Graph_V_norms_mu0.010_C1.000_gamma1.400.png}
\end{figure}

\newpage
Далее приведены значения $\Delta m(n)$ на измельченных и обычных сетках. А также их граффическое представление

\begin{center}
Table of mass loss. $\mu = 0.1000$ \, $C = 10.0000$, $\gamma = 1.0000$
  
\begin{tabular}{|p{0.8in}|p{0.8in}|p{0.8in}|p{0.8in}|p{0.8in}|p{0.8in}|p{0.8in}|} \hline
$K$ &$N_0$ &$N_0 \tau$ &$n = \frac{N_0}{4}$ &$n = \frac{N_0}{2}$ &$n = \frac{3N_0}{4}$ &$n = N_0$ \\ \hline 
0 &163946 &1.639e+02 &2.222e-03 &2.221e-03 &2.215e-03 &2.212e-03 \\ \hline 
1 &327891 &1.639e+02 &1.033e-03 &1.032e-03 &1.029e-03 &1.027e-03 \\ \hline 
2 &655781 &1.639e+02 &4.993e-04 &4.983e-04 &4.968e-04 &4.959e-04 \\ \hline 
3 &1311561 &1.639e+02 &2.455e-04 &2.449e-04 &2.442e-04 &2.437e-04 \\ \hline 

\end{tabular}\\[20pt]
\end{center}

\begin{center}
Table of mass loss. $\mu = 0.1000$ \, $C = 1.0000$, $\gamma = 1.0000$
  
\begin{tabular}{|p{0.8in}|p{0.8in}|p{0.8in}|p{0.8in}|p{0.8in}|p{0.8in}|p{0.8in}|} \hline
$K$ &$N_0$ &$N_0 \tau$ &$n = \frac{N_0}{4}$ &$n = \frac{N_0}{2}$ &$n = \frac{3N_0}{4}$ &$n = N_0$ \\ \hline 
0 &414339 &4.143e+02 &2.568e-04 &2.822e-04 &2.735e-04 &2.757e-04 \\ \hline 
1 &828677 &4.143e+02 &1.273e-04 &1.399e-04 &1.357e-04 &1.367e-04 \\ \hline 
2 &1657353 &4.143e+02 &6.339e-05 &6.971e-05 &6.757e-05 &6.811e-05 \\ \hline 
3 &3314705 &4.143e+02 &3.164e-05 &3.479e-05 &3.373e-05 &3.400e-05 \\ \hline 

\end{tabular}\\[20pt]
\end{center}

\begin{center}
Table of mass loss. $\mu = 0.1000$ \, $C = 1.0000$, $\gamma = 1.4000$
  
\begin{tabular}{|p{0.8in}|p{0.8in}|p{0.8in}|p{0.8in}|p{0.8in}|p{0.8in}|p{0.8in}|} \hline
$K$ &$N_0$ &$N_0 \tau$ &$n = \frac{N_0}{4}$ &$n = \frac{N_0}{2}$ &$n = \frac{3N_0}{4}$ &$n = N_0$ \\ \hline 
0 &400813 &4.008e+02 &2.850e-04 &2.777e-04 &2.739e-04 &2.725e-04 \\ \hline 
1 &801625 &4.008e+02 &1.410e-04 &1.374e-04 &1.355e-04 &1.349e-04 \\ \hline 
2 &1603249 &4.008e+02 &7.019e-05 &6.838e-05 &6.743e-05 &6.710e-05 \\ \hline 
3 &3206497 &4.008e+02 &3.502e-05 &3.411e-05 &3.364e-05 &3.348e-05 \\ \hline 

\end{tabular}\\[20pt]
\end{center}

\begin{center}
Table of mass loss. $\mu = 0.0100$ \, $C = 10.0000$, $\gamma = 1.0000$
  
\begin{tabular}{|p{0.8in}|p{0.8in}|p{0.8in}|p{0.8in}|p{0.8in}|p{0.8in}|p{0.8in}|} \hline
$K$ &$N_0$ &$N_0 \tau$ &$n = \frac{N_0}{4}$ &$n = \frac{N_0}{2}$ &$n = \frac{3N_0}{4}$ &$n = N_0$ \\ \hline 
0 &5197350 &5.197e+03 &2.025e-01 &2.025e-01 &2.025e-01 &2.025e-01 \\ \hline 
1 &10394699 &5.197e+03 &2.631e-03 &2.632e-03 &2.632e-03 &2.632e-03 \\ \hline 
2 &20789397 &5.197e+03 &8.500e-04 &8.502e-04 &8.502e-04 &8.502e-04 \\ \hline 
3 &41578793 &5.197e+03 &3.625e-04 &3.625e-04 &3.626e-04 &3.625e-04 \\ \hline 

\end{tabular}\\[20pt]
\end{center}

\begin{center}
Table of mass loss. $\mu = 0.0100$ \, $C = 1.0000$, $\gamma = 1.0000$
  
\begin{tabular}{|p{0.8in}|p{0.8in}|p{0.8in}|p{0.8in}|p{0.8in}|p{0.8in}|p{0.8in}|} \hline
$K$ &$N_0$ &$N_0 \tau$ &$n = \frac{N_0}{4}$ &$n = \frac{N_0}{2}$ &$n = \frac{3N_0}{4}$ &$n = N_0$ \\ \hline 
0 &943131 &9.431e+02 &2.208e-03 &2.204e-03 &2.224e-03 &2.215e-03 \\ \hline 
1 &1886261 &9.431e+02 &1.020e-03 &1.018e-03 &1.028e-03 &1.023e-03 \\ \hline 
2 &3772521 &9.431e+02 &4.918e-04 &4.909e-04 &4.956e-04 &4.935e-04 \\ \hline 
3 &7545041 &9.431e+02 &2.417e-04 &2.412e-04 &2.436e-04 &2.425e-04 \\ \hline 

\end{tabular}\\[20pt]
\end{center}

\begin{center}
Table of mass loss. $\mu = 0.0100$ \, $C = 1.0000$, $\gamma = 1.4000$
  
\begin{tabular}{|p{0.8in}|p{0.8in}|p{0.8in}|p{0.8in}|p{0.8in}|p{0.8in}|p{0.8in}|} \hline
$K$ &$N_0$ &$N_0 \tau$ &$n = \frac{N_0}{4}$ &$n = \frac{N_0}{2}$ &$n = \frac{3N_0}{4}$ &$n = N_0$ \\ \hline 
0 &919330 &9.193e+02 &3.906e-03 &3.902e-03 &3.904e-03 &3.911e-03 \\ \hline 
1 &1838659 &9.193e+02 &1.715e-03 &1.712e-03 &1.713e-03 &1.717e-03 \\ \hline 
2 &3677317 &9.193e+02 &8.101e-04 &8.087e-04 &8.089e-04 &8.109e-04 \\ \hline 
3 &7354633 &9.193e+02 &3.944e-04 &3.937e-04 &3.937e-04 &3.948e-04 \\ \hline 

\end{tabular}\\[20pt]
\end{center}


\newpage
\begin{figure}[H]
	\centering
	\includegraphics[scale=0.65]{../graphs_data_nonsmooth_2/mass/Graph_mass_mu0.100_C10.000_gamma1.000.png}
	\includegraphics[scale=0.65]{../graphs_data_nonsmooth_2/mass/Graph_mass_mu0.100_C1.000_gamma1.000.png}	
	\includegraphics[scale=0.65]{../graphs_data_nonsmooth_2/mass/Graph_mass_mu0.100_C1.000_gamma1.400.png}
\end{figure}


\begin{figure}[H]
	\centering
	\includegraphics[scale=0.65]{../graphs_data_nonsmooth_2/mass/Graph_mass_mu0.010_C10.000_gamma1.000.png}
	\includegraphics[scale=0.65]{../graphs_data_nonsmooth_2/mass/Graph_mass_mu0.010_C1.000_gamma1.000.png}	
	\includegraphics[scale=0.65]{../graphs_data_nonsmooth_2/mass/Graph_mass_mu0.010_C1.000_gamma1.400.png}
\end{figure}

Как можно заметить по таблицам потери массы составляют меньше 1-2\%, из чего можно заключить, что закон сохранения массы также выполняется. \\


