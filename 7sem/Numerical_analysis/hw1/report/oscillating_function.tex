\section{Задача о стабилизации осцилирующей функции}
\subsection{Постановка задачи}

Пусть $\Omega_x=[0;1]$, тогда для системы $(1)$ зададим две задачи, начальные условия которых, определяются следующим образом.

\begin{equation}
	\begin{cases}
		\begin{array}{l}
			\rho(0, x) = 2 + sin(N\pi x)\\
			u(0, x) = 0,\\
			u(t, 0) = u(t, 1) = 0, t \in [0; T]
		\end{array}
	\end{cases}
\end{equation}

\begin{equation}
	\begin{cases}
		\begin{array}{l}
			\rho(0, x) = 1\\
			u(0, x) = sin(N\pi x), \\
			u(t, 0) = u(t, 1) = 0, t \in [0; T]
		\end{array}
	\end{cases}
\end{equation}

Функции $f$ и $f_0$ из гладкой задачи положим равными $f \equiv f_0 \equiv 0$. Вычисление будем проводить до времени $T_{st}$, при котором решение выходит на стационар. Выходом на стационар будем считать выполнение условия:
$$
\Vert (H^n, V^n) - (\tilde{H^n}, \tilde{V^n}) \Vert_C \leq \varepsilon
$$

Число $N$ будем брать исходя из неравенства $10Nh \leq 1$.

\newpage
\subsection{Численные эксперименты}
В качестве измельчения сетки возьмем $\tau = 10^{-4} \, h = 10^{-2}$. Эксперементальным путем было подобрано значение $\varepsilon = 10^{-4}$. Ввиду того, что при $C = 100$ и $\mu = 0.001$ сходимость схемы плохая, данные параметры не будем расссматривать при расчете различных вариантов начальных условий.

\subsection{Первая задача}

Приведем таблицы времени стабилизации в зависимости от различных $\mu$, $C$, $\gamma$.

\begin{center}
Table of stabilize time. $\mu = 0.0100$ \, $C = 10.0000$, $\gamma = 1.0000$
  
\begin{tabular}{|p{0.6in}|p{0.6in}|p{0.6in}|p{0.6in}|p{0.6in}|p{0.6in}|p{0.6in}|p{0.6in}|p{0.6in}|} \hline
$N$ &1 &2 &3 &4 &5 &6 &7 &8 \\ \hline 
$T_{st}$ &5.5e+01 &1.5e+02 &4.7e+01 &1.4e+02 &4.5e+01 &1.4e+02 &4.4e+01 &1.3e+02 \\ \hline 

\end{tabular}\\[20pt]
\end{center}

\begin{center}
Table of stabilize time. $\mu = 0.0100$ \, $C = 1.0000$, $\gamma = 1.0000$
  
\begin{tabular}{|p{0.6in}|p{0.6in}|p{0.6in}|p{0.6in}|p{0.6in}|p{0.6in}|p{0.6in}|p{0.6in}|p{0.6in}|} \hline
$N$ &1 &2 &3 &4 &5 &6 &7 &8 \\ \hline 
$T_{st}$ &4.5e+01 &1.3e+02 &4.0e+01 &1.2e+02 &3.7e+01 &1.2e+02 &3.4e+01 &1.1e+02 \\ \hline 

\end{tabular}\\[20pt]
\end{center}

\begin{center}
Table of stabilize time. $\mu = 0.0100$ \, $C = 1.0000$, $\gamma = 1.4000$
  
\begin{tabular}{|p{0.6in}|p{0.6in}|p{0.6in}|p{0.6in}|p{0.6in}|p{0.6in}|p{0.6in}|p{0.6in}|p{0.6in}|} \hline
$N$ &1 &2 &3 &4 &5 &6 &7 &8 \\ \hline 
$T_{st}$ &4.5e+01 &1.3e+02 &3.9e+01 &1.2e+02 &3.8e+01 &1.1e+02 &3.6e+01 &1.1e+02 \\ \hline 

\end{tabular}\\[20pt]
\end{center}

\begin{center}
Table of stabilize time. $\mu = 0.1000$ \, $C = 10.0000$, $\gamma = 1.0000$
  
\begin{tabular}{|p{0.6in}|p{0.6in}|p{0.6in}|p{0.6in}|p{0.6in}|p{0.6in}|p{0.6in}|p{0.6in}|p{0.6in}|} \hline
$N$ &1 &2 &3 &4 &5 &6 &7 &8 \\ \hline 
$T_{st}$ &8.7e+00 &2.1e+00 &1.0e+00 &5.2e-01 &3.5e-01 &2.4e-01 &1.6e-01 &1.5e-01 \\ \hline 

\end{tabular}\\[20pt]
\end{center}

\begin{center}
Table of stabilize time. $\mu = 0.1000$ \, $C = 1.0000$, $\gamma = 1.0000$
  
\begin{tabular}{|p{0.6in}|p{0.6in}|p{0.6in}|p{0.6in}|p{0.6in}|p{0.6in}|p{0.6in}|p{0.6in}|p{0.6in}|} \hline
$N$ &1 &2 &3 &4 &5 &6 &7 &8 \\ \hline 
$T_{st}$ &4.0e+00 &1.6e+01 &4.1e+00 &1.3e+01 &3.0e+00 &1.1e+01 &2.5e+00 &1.0e+01 \\ \hline 

\end{tabular}\\[20pt]
\end{center}

\begin{center}
Table of stabilize time. $\mu = 0.1000$ \, $C = 1.0000$, $\gamma = 1.4000$
  
\begin{tabular}{|p{0.6in}|p{0.6in}|p{0.6in}|p{0.6in}|p{0.6in}|p{0.6in}|p{0.6in}|p{0.6in}|p{0.6in}|} \hline
$N$ &1 &2 &3 &4 &5 &6 &7 &8 \\ \hline 
$T_{st}$ &8.9e+00 &2.4e+00 &1.0e+00 &5.7e-01 &5.1e-01 &4.9e-01 &4.3e-01 &4.5e-01 \\ \hline 

\end{tabular}\\[20pt]
\end{center}


Вывод: при нечетных $N$ сходимость на порядок быстрее чем при четных, при уменьшении $\mu$ также возрастает время стабилизации. От $C$ сходимость практически не зависит.

\subsection{Вторая задача}
Приведем таблицы времени стабилизации в зависимости от различных $\mu$, $C$, $\gamma$.

\begin{center}
Table of stabilize time. $\mu = 0.0100$ \, $C = 10.0000$, $\gamma = 1.0000$
  
\begin{tabular}{|p{0.6in}|p{0.6in}|p{0.6in}|p{0.6in}|p{0.6in}|p{0.6in}|p{0.6in}|p{0.6in}|p{0.6in}|} \hline
$N$ &1 &2 &3 &4 &5 &6 &7 &8 \\ \hline 
$T_{st}$ &5.5e+01 &1.5e+02 &4.7e+01 &1.4e+02 &4.5e+01 &1.4e+02 &4.4e+01 &1.3e+02 \\ \hline 

\end{tabular}\\[20pt]
\end{center}

\begin{center}
Table of stabilize time. $\mu = 0.0100$ \, $C = 1.0000$, $\gamma = 1.0000$
  
\begin{tabular}{|p{0.6in}|p{0.6in}|p{0.6in}|p{0.6in}|p{0.6in}|p{0.6in}|p{0.6in}|p{0.6in}|p{0.6in}|} \hline
$N$ &1 &2 &3 &4 &5 &6 &7 &8 \\ \hline 
$T_{st}$ &4.5e+01 &1.3e+02 &4.0e+01 &1.2e+02 &3.7e+01 &1.2e+02 &3.4e+01 &1.1e+02 \\ \hline 

\end{tabular}\\[20pt]
\end{center}

\begin{center}
Table of stabilize time. $\mu = 0.0100$ \, $C = 1.0000$, $\gamma = 1.4000$
  
\begin{tabular}{|p{0.6in}|p{0.6in}|p{0.6in}|p{0.6in}|p{0.6in}|p{0.6in}|p{0.6in}|p{0.6in}|p{0.6in}|} \hline
$N$ &1 &2 &3 &4 &5 &6 &7 &8 \\ \hline 
$T_{st}$ &4.5e+01 &1.3e+02 &3.9e+01 &1.2e+02 &3.8e+01 &1.1e+02 &3.6e+01 &1.1e+02 \\ \hline 

\end{tabular}\\[20pt]
\end{center}

\begin{center}
Table of stabilize time. $\mu = 0.1000$ \, $C = 10.0000$, $\gamma = 1.0000$
  
\begin{tabular}{|p{0.6in}|p{0.6in}|p{0.6in}|p{0.6in}|p{0.6in}|p{0.6in}|p{0.6in}|p{0.6in}|p{0.6in}|} \hline
$N$ &1 &2 &3 &4 &5 &6 &7 &8 \\ \hline 
$T_{st}$ &8.7e+00 &2.1e+00 &1.0e+00 &5.2e-01 &3.5e-01 &2.4e-01 &1.6e-01 &1.5e-01 \\ \hline 

\end{tabular}\\[20pt]
\end{center}

\begin{center}
Table of stabilize time. $\mu = 0.1000$ \, $C = 1.0000$, $\gamma = 1.0000$
  
\begin{tabular}{|p{0.6in}|p{0.6in}|p{0.6in}|p{0.6in}|p{0.6in}|p{0.6in}|p{0.6in}|p{0.6in}|p{0.6in}|} \hline
$N$ &1 &2 &3 &4 &5 &6 &7 &8 \\ \hline 
$T_{st}$ &4.0e+00 &1.6e+01 &4.1e+00 &1.3e+01 &3.0e+00 &1.1e+01 &2.5e+00 &1.0e+01 \\ \hline 

\end{tabular}\\[20pt]
\end{center}

\begin{center}
Table of stabilize time. $\mu = 0.1000$ \, $C = 1.0000$, $\gamma = 1.4000$
  
\begin{tabular}{|p{0.6in}|p{0.6in}|p{0.6in}|p{0.6in}|p{0.6in}|p{0.6in}|p{0.6in}|p{0.6in}|p{0.6in}|} \hline
$N$ &1 &2 &3 &4 &5 &6 &7 &8 \\ \hline 
$T_{st}$ &8.9e+00 &2.4e+00 &1.0e+00 &5.7e-01 &5.1e-01 &4.9e-01 &4.3e-01 &4.5e-01 \\ \hline 

\end{tabular}\\[20pt]
\end{center}


Вывод: при нечетных $N$ сходимость на порядок быстрее чем при четных, при уменьшении $\mu$ также возрастает время стабилизации. От $C$ сходимость практически не зависит.

\newpage 