\section{Отладочный тест}
\subsection{Постановка задачи}

Рассмотрим $Q = [0;1]$x$[0;1]$\\

Зададим функции $\tilde{\rho}(t, x)$ и $\tilde{u}(t, x)$ так, чтобы они являлись гладким решением задачи $(1)$.
\begin{equation}
	\begin{array}{lc}
		\tilde{\rho}(t, x) = e^t(cos(3\pi x) + 1.5),\\
		\tilde{u}(t, x) = cos(2\pi t)sin(4\pi x)\\
	\end{array}
\end{equation}

Теперь определим функции $f_0$ и $f$, так, чтобы они удовлетворяли уравнениям:
\begin{equation}
	\begin{array}{lc}
		\frac{\partial\tilde{\rho}}{\partial t} + \frac{\partial\tilde{\rho}\tilde{u}}{\partial x} = f_0,\\
		\tilde{\rho}\frac{\partial\tilde{u}}{\partial t} + \tilde{u}\tilde{\rho}\frac{\partial\tilde{u}}{\partial x} + \frac{\partial p}{\partial x} = \mu\frac{\partial^2\tilde{u}}{\partial x^2} + \tilde{\rho}f
	\end{array}
\end{equation}

\begin{equation}
	\begin{array}{lc}
		\frac{\partial\tilde{\rho}}{\partial t}  = e^t(cos(3\pi x) + 1.5), \\
		\frac{\partial\tilde{\rho}\tilde{u}}{\partial t}  = \pi e^t cos(2\pi t) * (4(cos(3\pi x) + 1.5) cos (4\pi x) - 3sin(3\pi x )sin(4\pi x)), \\
		
		\frac{\partial\tilde{u}}{\partial t} = -2\pi sin(2\pi t)sin(4\pi x),\\				\frac{\partial\tilde{u}}{\partial x} = 4\pi cos(2\pi t)cos(4\pi x),\\
		\frac{\partial^2\tilde{u}}{\partial x^2} = -16\pi^2 cos(2\pi t)sin(4\pi x),\\
		\frac{\partial\tilde{ru}}{\partial x} = \pi e^t cos(2\pi t) (4(cos(3\pi x) + 1.5)cos(4\pi x) - 3sin(3\pi x)sin(4\pi x)),\\
		\frac{\partial\tilde{p}}{\partial x} = C\gamma \rho^{\gamma - 1} \frac{\partial \rho}{\partial x}		
	\end{array}
\end{equation}
\newpage

\subsection{Численные эксперименты}
\subsubsection{Обычная сетка}

\begin{center}
Table of times
\\[2.0ex]  
  
\begin{tabular}{|c|c|c|c|c|} \hline
$\tau / h$ &1.000e-01 &1.000e-02 &1.000e-03 &1.000e-04 \\ \hline 
1.000e-01 &8.580e-04 &1.526e-03 &1.949e-02 &2.003e-01 \\ \hline 
1.000e-02 &1.016e-03 &1.885e-02 &1.555e-01 &8.970e-01 \\ \hline 
1.000e-03 &2.114e-02 &1.974e-01 &7.778e-01 &3.687e+00 \\ \hline 
1.000e-04 &2.013e-01 &9.031e-01 &3.373e+00 &1.511e+01 \\ \hline 

\end{tabular}\\[20pt]
\end{center}


\begin{center}
Table of norms for H. $\mu = 0.0010$ \, $C = 100.0000$, $\gamma = 1.0000$
  
\begin{tabular}{|p{1in}|p{1in}|p{1in}|p{1in}|p{1in}|} \hline
$\tau / h$ &1.000e-01 &1.000e-02 &1.000e-03 &1.000e-04 \\ \hline 
1.000e-01 & $2.459e+09$  $1.271e+09$  $1.787e+10$  & $2.617e+14$  $1.726e+14$  $2.442e+16$  & $5.237e+09$  $1.051e+09$  $1.478e+12$  & $1.501e+15$  $1.246e+13$  $1.762e+17$  \\ \hline 
1.000e-02 & $2.295e+47$  $9.846e+46$  $2.914e+47$  & $1.725e+89$  $4.601e+88$  $7.394e+90$  & $6.910e+94$  $8.469e+93$  $1.209e+97$  & $5.618e+174$  $inf$  $inf$  \\ \hline 
1.000e-03 & $4.877e+259$  $inf$  $inf$  & $nan$  $-nan$  $-nan$  & $nan$  $-nan$  $-nan$  & $nan$  $-nan$  $-nan$  \\ \hline 
1.000e-04 & $nan$  $-nan$  $-nan$  & $nan$  $-nan$  $-nan$  & $nan$  $-nan$  $-nan$  & $nan$  $-nan$  $-nan$  \\ \hline 

\end{tabular}\\[20pt]
\end{center}

\newpage

\begin{center}
Table of norms for H. $\mu = 0.0010$ \, $C = 100.0000$, $\gamma = 1.0000$
  
\begin{tabular}{|p{1in}|p{1in}|p{1in}|p{1in}|p{1in}|} \hline
$\tau / h$ &1.000e-01 &1.000e-02 &1.000e-03 &1.000e-04 \\ \hline 
1.000e-01 & $8.432e+01$  $4.331e+01$  $6.501e+02$  & $5.324e+02$  $1.322e+02$  $1.830e+04$  & $1.777e+07$  $3.365e+06$  $5.984e+09$  & $6.521e+07$  $1.029e+06$  $1.502e+10$  \\ \hline 
1.000e-02 & $6.617e+01$  $3.743e+01$  $5.206e+02$  & $5.845e+02$  $1.259e+02$  $2.333e+04$  & $3.629e+03$  $2.689e+02$  $4.544e+05$  & $4.387e+02$  $1.104e+02$  $1.412e+06$  \\ \hline 
1.000e-03 & $3.510e+02$  $2.474e+02$  $3.672e+03$  & $nan$  $-nan$  $-nan$  & $nan$  $nan$  $nan$  & $nan$  $nan$  $nan$  \\ \hline 
1.000e-04 & $nan$  $-nan$  $-nan$  & $nan$  $nan$  $nan$  & $nan$  $-nan$  $-nan$  & $nan$  $nan$  $nan$  \\ \hline 

\end{tabular}\\[20pt]
\end{center}

\begin{center}
Table of norms for H. $\mu = 0.0010$ \, $C = 10.0000$, $\gamma = 1.0000$
  
\begin{tabular}{|p{1in}|p{1in}|p{1in}|p{1in}|} \hline
$k / \tau = h$ &1.000e-01 &1.000e-02 &1.000e-03 \\ \hline 
0.000e+00 & $7.614e+04$  $3.000e+04$  $1.997e+05$  & $9.181e+60$  $1.911e+60$  $2.248e+62$  & $nan$  $-nan$  $-nan$  \\ \hline 
1.000e+00 & $4.572e+11$  $1.272e+11$  $9.031e+11$  & $1.543e+114$  $1.092e+113$  $1.506e+115$  & $nan$  $nan$  $nan$  \\ \hline 
2.000e+00 & $2.800e+23$  $6.592e+22$  $1.097e+24$  & $1.253e+227$  $inf$  $inf$  & $nan$  $-nan$  $-nan$  \\ \hline 
3.000e+00 & $2.884e+49$  $6.678e+48$  $7.670e+49$  & $nan$  $nan$  $nan$  & $nan$  $nan$  $nan$  \\ \hline 
4.000e+00 & $4.371e+91$  $1.075e+91$  $1.399e+92$  & $nan$  $nan$  $nan$  & $nan$  $nan$  $nan$  \\ \hline 

\end{tabular}\\[20pt]
\end{center}


\newpage
\begin{center}
Table of norms for H. $\mu = 0.0010$ \, $C = 10.0000$, $\gamma = 1.0000$
  
\begin{tabular}{|p{1in}|p{1in}|p{1in}|p{1in}|p{1in}|} \hline
$k / \tau = h$ &1.000e-01 &1.000e-02 &1.000e-03 &1.000e-04 \\ \hline 
0.000e+00 & $1.819e+01$  $9.487e+00$  $1.608e+02$  & $4.016e+01$  $1.084e+01$  $1.486e+03$  & $nan$  $-nan$  $-nan$  & $nan$  $-nan$  $-nan$  \\ \hline 
1.000e+00 & $3.509e+01$  $8.056e+00$  $4.023e+02$  & $7.851e+02$  $4.521e+01$  $2.854e+04$  & $nan$  $-nan$  $-nan$  & $nan$  $-nan$  $-nan$  \\ \hline 
2.000e+00 & $6.363e+02$  $7.266e+01$  $8.018e+03$  & $nan$  $-nan$  $-nan$  & $nan$  $-nan$  $-nan$  & $nan$  $-nan$  $-nan$  \\ \hline 
3.000e+00 & $1.456e+02$  $1.403e+01$  $3.668e+03$  & $nan$  $-nan$  $-nan$  & $nan$  $-nan$  $-nan$  & $nan$  $-nan$  $-nan$  \\ \hline 

\end{tabular}\\[20pt]
\end{center}

\begin{center}
Table of norms for H. $\mu = 0.0010$ \, $C = 1.0000$, $\gamma = 1.0000$
  
\begin{tabular}{|p{1in}|p{1in}|p{1in}|p{1in}|} \hline
$k / \tau = h$ &1.000e-01 &1.000e-02 &1.000e-03 \\ \hline 
0.000e+00 & $1.140e+02$  $4.643e+01$  $6.830e+02$  & $2.865e+25$  $3.835e+24$  $7.577e+26$  & $6.365e+257$  $inf$  $inf$  \\ \hline 
1.000e+00 & $2.711e+07$  $7.008e+06$  $6.578e+07$  & $2.876e+116$  $2.036e+115$  $2.955e+117$  & $1.706e+41$  $6.655e+39$  $3.932e+42$  \\ \hline 
2.000e+00 & $3.570e+20$  $8.480e+19$  $1.229e+21$  & $1.518e+182$  $inf$  $inf$  & $1.611e-02$  $2.921e-03$  $3.865e-01$  \\ \hline 
3.000e+00 & $1.044e+50$  $2.336e+49$  $3.300e+50$  & $nan$  $-nan$  $-nan$  & $7.515e-03$  $1.434e-03$  $1.781e-01$  \\ \hline 

\end{tabular}\\[20pt]
\end{center}


\newpage
\begin{center}
Table of norms for H. $\mu = 0.0010$ \, $C = 1.0000$, $\gamma = 1.0000$
  
\begin{tabular}{|p{1in}|p{1in}|p{1in}|p{1in}|p{1in}|} \hline
$k / \tau = h$ &1.000e-01 &1.000e-02 &1.000e-03 &1.000e-04 \\ \hline 
0.000e+00 & $8.932e+00$  $5.407e+00$  $9.547e+01$  & $3.517e+01$  $8.779e+00$  $1.286e+03$  & $nan$  $-nan$  $-nan$  & $nan$  $-nan$  $-nan$  \\ \hline 
1.000e+00 & $6.414e+00$  $1.548e+00$  $1.019e+02$  & $1.870e+02$  $1.009e+01$  $6.358e+03$  & $nan$  $-nan$  $-nan$  & $nan$  $-nan$  $-nan$  \\ \hline 
2.000e+00 & $1.052e+01$  $1.766e+00$  $1.631e+02$  & $5.731e+01$  $3.071e+00$  $3.337e+03$  & $nan$  $-nan$  $-nan$  & $nan$  $-nan$  $-nan$  \\ \hline 
3.000e+00 & $8.281e+00$  $9.728e-01$  $2.262e+02$  & $nan$  $-nan$  $-nan$  & $nan$  $-nan$  $-nan$  & $nan$  $-nan$  $-nan$  \\ \hline 

\end{tabular}\\[20pt]
\end{center}

\begin{center}
Table of norms for H. $\mu = 0.0010$ \, $C = 1.0000$, $\gamma = 1.4000$
  
\begin{tabular}{|p{1in}|p{1in}|p{1in}|p{1in}|p{1in}|} \hline
$\tau / h$ &1.000e-01 &1.000e-02 &1.000e-03 &1.000e-04 \\ \hline 
1.000e-01 & $nan$  $-nan$  $-nan$  & $nan$  $-nan$  $-nan$  & $nan$  $-nan$  $-nan$  & $nan$  $-nan$  $-nan$  \\ \hline 
1.000e-02 & $nan$  $-nan$  $-nan$  & $nan$  $-nan$  $-nan$  & $nan$  $-nan$  $-nan$  & $nan$  $-nan$  $-nan$  \\ \hline 
1.000e-03 & $nan$  $-nan$  $-nan$  & $nan$  $-nan$  $-nan$  & $nan$  $-nan$  $-nan$  & $nan$  $-nan$  $-nan$  \\ \hline 
1.000e-04 & $nan$  $-nan$  $-nan$  & $1.498e-02$  $4.873e-03$  $5.353e-01$  & $6.867e-04$  $2.722e-04$  $5.344e-03$  & $nan$  $-nan$  $-nan$  \\ \hline 

\end{tabular}\\[20pt]
\end{center}


\newpage
\begin{center}
Table of norms for V. $\mu = 0.0010$ \, $C = 1.0000$, $\gamma = 1.4000$
  
\begin{tabular}{|p{1in}|p{1in}|p{1in}|p{1in}|p{1in}|} \hline
$\tau / h$ &1.000e-01 &1.000e-02 &1.000e-03 &1.000e-04 \\ \hline 
1.000e-01 & $nan$  $-nan$  $-nan$  & $nan$  $-nan$  $-nan$  & $nan$  $-nan$  $-nan$  & $nan$  $-nan$  $-nan$  \\ \hline 
1.000e-02 & $nan$  $-nan$  $-nan$  & $nan$  $-nan$  $-nan$  & $nan$  $-nan$  $-nan$  & $nan$  $-nan$  $-nan$  \\ \hline 
1.000e-03 & $nan$  $-nan$  $-nan$  & $nan$  $-nan$  $-nan$  & $nan$  $-nan$  $-nan$  & $nan$  $-nan$  $-nan$  \\ \hline 
1.000e-04 & $nan$  $-nan$  $-nan$  & $7.856e-03$  $2.837e-03$  $2.038e-01$  & $2.161e-04$  $1.035e-04$  $1.978e-03$  & $nan$  $-nan$  $-nan$  \\ \hline 

\end{tabular}\\[20pt]
\end{center}

\begin{center}
Table of norms for H. $\mu = 0.0100$ \, $C = 100.0000$, $\gamma = 1.0000$
  
\begin{tabular}{|p{1in}|p{1in}|p{1in}|p{1in}|p{1in}|} \hline
$\tau / h$ &1.000e-01 &1.000e-02 &1.000e-03 &1.000e-04 \\ \hline 
1.000e-01 & $1.989e+10$  $8.497e+09$  $1.238e+11$  & $2.114e+14$  $4.989e+13$  $7.172e+15$  & $9.311e+16$  $2.483e+16$  $3.521e+19$  & $1.145e+20$  $7.159e+19$  $1.013e+24$  \\ \hline 
1.000e-02 & $3.002e+50$  $1.631e+50$  $3.624e+51$  & $7.628e+80$  $1.226e+80$  $1.901e+82$  & $9.421e+132$  $7.884e+131$  $1.125e+135$  & $1.178e+169$  $inf$  $inf$  \\ \hline 
1.000e-03 & $4.218e+263$  $inf$  $inf$  & $nan$  $-nan$  $-nan$  & $nan$  $-nan$  $-nan$  & $nan$  $-nan$  $-nan$  \\ \hline 
1.000e-04 & $nan$  $-nan$  $-nan$  & $nan$  $-nan$  $-nan$  & $nan$  $-nan$  $-nan$  & $nan$  $-nan$  $-nan$  \\ \hline 

\end{tabular}\\[20pt]
\end{center}


\newpage
\begin{center}
Table of norms for H. $\mu = 0.0100$ \, $C = 100.0000$, $\gamma = 1.0000$
  
\begin{tabular}{|p{1in}|p{1in}|p{1in}|p{1in}|p{1in}|} \hline
$\tau / h$ &1.000e-01 &1.000e-02 &1.000e-03 &1.000e-04 \\ \hline 
1.000e-01 & $2.359e+02$  $9.856e+01$  $1.597e+03$  & $7.350e+03$  $9.909e+02$  $1.700e+05$  & $2.120e+05$  $2.629e+04$  $3.718e+07$  & $1.845e+08$  $2.867e+06$  $4.009e+10$  \\ \hline 
1.000e-02 & $6.080e+01$  $4.236e+01$  $6.189e+02$  & $1.294e+03$  $2.064e+02$  $3.088e+04$  & $1.108e+04$  $5.357e+02$  $7.637e+05$  & $4.299e+03$  $2.828e+02$  $4.900e+06$  \\ \hline 
1.000e-03 & $3.524e+02$  $2.480e+02$  $3.690e+03$  & $nan$  $nan$  $nan$  & $nan$  $nan$  $nan$  & $nan$  $nan$  $nan$  \\ \hline 
1.000e-04 & $nan$  $-nan$  $-nan$  & $nan$  $nan$  $nan$  & $nan$  $nan$  $nan$  & $nan$  $nan$  $nan$  \\ \hline 

\end{tabular}\\[20pt]
\end{center}

\begin{center}
Table of norms for H. $\mu = 0.0100$ \, $C = 10.0000$, $\gamma = 1.0000$
  
\begin{tabular}{|p{1in}|p{1in}|p{1in}|p{1in}|p{1in}|} \hline
$k / \tau = h$ &1.000e-01 &1.000e-02 &1.000e-03 &1.000e-04 \\ \hline 
0.000e+00 & $7.870e+04$  $2.214e+04$  $3.880e+05$  & $8.442e+55$  $9.029e+54$  $1.630e+57$  & $nan$  $-nan$  $-nan$  & $nan$  $-nan$  $-nan$  \\ \hline 
1.000e+00 & $6.980e+24$  $7.809e+23$  $6.386e+25$  & $6.061e+254$  $inf$  $inf$  & $nan$  $-nan$  $-nan$  & $nan$  $-nan$  $-nan$  \\ \hline 
2.000e+00 & $2.057e+48$  $1.663e+47$  $1.740e+49$  & $nan$  $-nan$  $-nan$  & $nan$  $-nan$  $-nan$  & $nan$  $-nan$  $-nan$  \\ \hline 
3.000e+00 & $1.772e+106$  $1.218e+105$  $3.221e+107$  & $nan$  $-nan$  $-nan$  & $nan$  $-nan$  $-nan$  & $nan$  $-nan$  $-nan$  \\ \hline 

\end{tabular}\\[20pt]
\end{center}


\newpage
\begin{center}
Table of norms for H. $\mu = 0.0100$ \, $C = 10.0000$, $\gamma = 1.0000$
  
\begin{tabular}{|p{1in}|p{1in}|p{1in}|p{1in}|p{1in}|} \hline
$\tau / h$ &1.000e-01 &1.000e-02 &1.000e-03 &1.000e-04 \\ \hline 
1.000e-01 & $2.159e+01$  $9.587e+00$  $1.198e+02$  & $2.471e+02$  $3.984e+01$  $4.609e+03$  & $2.133e+05$  $8.967e+03$  $1.237e+07$  & $1.928e+08$  $3.463e+06$  $4.953e+10$  \\ \hline 
1.000e-02 & $3.757e+01$  $1.960e+01$  $2.769e+02$  & $7.268e+03$  $7.591e+02$  $1.296e+05$  & $1.227e+05$  $5.647e+03$  $6.299e+06$  & $3.534e+04$  $1.529e+03$  $2.238e+07$  \\ \hline 
1.000e-03 & $8.250e+01$  $5.484e+01$  $9.331e+02$  & $5.382e+01$  $2.538e+01$  $3.733e+03$  & $nan$  $-nan$  $-nan$  & $nan$  $nan$  $nan$  \\ \hline 
1.000e-04 & $6.557e+01$  $4.134e+01$  $6.523e+02$  & $4.257e-03$  $2.137e-03$  $3.011e-02$  & $nan$  $-nan$  $-nan$  & $nan$  $nan$  $nan$  \\ \hline 

\end{tabular}\\[20pt]
\end{center}

\begin{center}
Table of norms for H. $\mu = 0.0100$ \, $C = 1.0000$, $\gamma = 1.0000$
  
\begin{tabular}{|p{1in}|p{1in}|p{1in}|p{1in}|p{1in}|} \hline
$k / \tau = h$ &1.000e-01 &1.000e-02 &1.000e-03 &1.000e-04 \\ \hline 
0.000e+00 & $6.976e+01$  $3.338e+01$  $6.133e+02$  & $1.138e+36$  $1.241e+35$  $2.314e+37$  & $nan$  $-nan$  $-nan$  & $nan$  $-nan$  $-nan$  \\ \hline 
1.000e+00 & $2.047e+09$  $2.297e+08$  $1.740e+10$  & $1.040e+154$  $3.849e+152$  $inf$  & $nan$  $-nan$  $-nan$  & $nan$  $-nan$  $-nan$  \\ \hline 
2.000e+00 & $1.520e+18$  $1.829e+17$  $1.101e+19$  & $2.689e+271$  $inf$  $inf$  & $nan$  $-nan$  $-nan$  & $nan$  $-nan$  $-nan$  \\ \hline 
3.000e+00 & $3.045e+39$  $1.702e+38$  $3.889e+40$  & $nan$  $-nan$  $-nan$  & $nan$  $-nan$  $-nan$  & $nan$  $-nan$  $-nan$  \\ \hline 

\end{tabular}\\[20pt]
\end{center}


\newpage
\begin{center}
Table of norms for V. $\mu = 0.0100$ \, $C = 1.0000$, $\gamma = 1.0000$
  
\begin{tabular}{|p{0.8in}|p{0.8in}|p{0.8in}|p{0.8in}|p{0.8in}|p{0.8in}|p{0.8in}|} \hline
$K$ &$N_0$ &$N_0 \tau$ &$n = \frac{N_0}{4}$ &$n = \frac{N_0}{2}$ &$n = \frac{3N_0}{4}$ &$n = N_0$ \\ \hline 
0 &943131 &9.431e+02 &3.344e-02 &1.147e-02 &4.332e-03 &9.996e-04 \\ \hline 
1 &1886261 &9.431e+02 &3.311e-02 &1.129e-02 &4.185e-03 &9.050e-04 \\ \hline 
2 &3772521 &9.431e+02 &3.293e-02 &1.120e-02 &4.114e-03 &8.613e-04 \\ \hline 
3 &7545041 &9.431e+02 &3.285e-02 &1.115e-02 &4.079e-03 &8.405e-04 \\ \hline 

\end{tabular}\\[20pt]
\end{center}

\begin{center}
Table of norms for H. $\mu = 0.0100$ \, $C = 1.0000$, $\gamma = 1.4000$
  
\begin{tabular}{|p{1in}|p{1in}|p{1in}|p{1in}|} \hline
$k / \tau = h$ &1.000e-01 &1.000e-02 &1.000e-03 \\ \hline 
0.000e+00 & $nan$  $nan$  $nan$  & $nan$  $nan$  $nan$  & $7.351e-03$  $2.605e-03$  $5.011e-02$  \\ \hline 
1.000e+00 & $nan$  $nan$  $nan$  & $nan$  $-nan$  $-nan$  & $3.532e-03$  $1.278e-03$  $2.051e-02$  \\ \hline 
2.000e+00 & $nan$  $nan$  $nan$  & $nan$  $nan$  $nan$  & $1.733e-03$  $6.340e-04$  $1.002e-02$  \\ \hline 
3.000e+00 & $nan$  $-nan$  $-nan$  & $nan$  $nan$  $nan$  & $8.586e-04$  $3.159e-04$  $4.956e-03$  \\ \hline 
4.000e+00 & $nan$  $-nan$  $-nan$  & $4.447e-03$  $1.604e-03$  $2.586e-02$  & $4.274e-04$  $1.577e-04$  $2.466e-03$  \\ \hline 

\end{tabular}\\[20pt]
\end{center}


\newpage
\begin{center}
Table of norms for V. $\mu = 0.0100$ \, $C = 1.0000$, $\gamma = 1.4000$
  
\begin{tabular}{|p{0.8in}|p{0.8in}|p{0.8in}|p{0.8in}|p{0.8in}|p{0.8in}|p{0.8in}|} \hline
$K$ &$N_0$ &$N_0 \tau$ &$n = \frac{N_0}{4}$ &$n = \frac{N_0}{2}$ &$n = \frac{3N_0}{4}$ &$n = N_0$ \\ \hline 
0 &919330 &9.193e+02 &1.980e-02 &8.076e-03 &6.651e-03 &9.998e-04 \\ \hline 
1 &1838659 &9.193e+02 &1.891e-02 &8.395e-03 &5.863e-03 &1.467e-03 \\ \hline 
2 &3677317 &9.193e+02 &1.853e-02 &8.487e-03 &5.518e-03 &1.632e-03 \\ \hline 
3 &7354633 &9.193e+02 &1.835e-02 &8.521e-03 &5.358e-03 &1.701e-03 \\ \hline 

\end{tabular}\\[20pt]
\end{center}

\begin{center}
Table of norms for H. $\mu = 0.1000$ \, $C = 100.0000$, $\gamma = 1.0000$
  
\begin{tabular}{|p{1in}|p{1in}|p{1in}|p{1in}|p{1in}|} \hline
$\tau / h$ &1.000e-01 &1.000e-02 &1.000e-03 &1.000e-04 \\ \hline 
1.000e-01 & $1.091e+08$  $3.948e+07$  $7.057e+08$  & $1.747e+05$  $4.836e+04$  $7.969e+06$  & $8.058e+04$  $5.224e+03$  $7.998e+06$  & $1.843e+04$  $2.690e+02$  $4.799e+06$  \\ \hline 
1.000e-02 & $7.055e+31$  $3.830e+31$  $8.213e+32$  & $1.894e+67$  $2.335e+66$  $4.199e+68$  & $1.794e+101$  $6.181e+99$  $8.807e+102$  & $4.378e+114$  $3.200e+113$  $4.742e+117$  \\ \hline 
1.000e-03 & $7.174e+166$  $inf$  $inf$  & $nan$  $-nan$  $-nan$  & $nan$  $-nan$  $-nan$  & $nan$  $-nan$  $-nan$  \\ \hline 
1.000e-04 & $1.348e+232$  $inf$  $inf$  & $3.819e-03$  $2.353e-03$  $2.399e-02$  & $1.866e-04$  $1.346e-04$  $4.009e-04$  & $2.248e-04$  $1.428e-04$  $5.979e-04$  \\ \hline 

\end{tabular}\\[20pt]
\end{center}


\newpage
\begin{center}
Table of norms for H. $\mu = 0.1000$ \, $C = 100.0000$, $\gamma = 1.0000$
  
\begin{tabular}{|p{1in}|p{1in}|p{1in}|p{1in}|p{1in}|} \hline
$\tau / h$ &1.000e-01 &1.000e-02 &1.000e-03 &1.000e-04 \\ \hline 
1.000e-01 & $1.396e+02$  $7.026e+01$  $1.006e+03$  & $6.052e+04$  $1.539e+04$  $2.373e+06$  & $4.018e+03$  $3.036e+02$  $4.364e+05$  & $3.551e+06$  $4.198e+04$  $5.933e+08$  \\ \hline 
1.000e-02 & $7.060e+01$  $4.460e+01$  $6.486e+02$  & $1.619e+03$  $2.911e+02$  $4.415e+04$  & $4.477e+03$  $6.658e+02$  $9.019e+05$  & $1.287e+05$  $4.590e+03$  $6.990e+07$  \\ \hline 
1.000e-03 & $3.685e+02$  $2.594e+02$  $3.850e+03$  & $nan$  $-nan$  $-nan$  & $nan$  $-nan$  $-nan$  & $nan$  $-nan$  $-nan$  \\ \hline 
1.000e-04 & $3.042e+02$  $2.134e+02$  $3.178e+03$  & $3.322e-03$  $1.908e-03$  $2.599e-02$  & $nan$  $-nan$  $-nan$  & $nan$  $-nan$  $-nan$  \\ \hline 

\end{tabular}\\[20pt]
\end{center}

\begin{center}
Table of norms for H. $\mu = 0.1000$ \, $C = 10.0000$, $\gamma = 1.0000$
  
\begin{tabular}{|p{1in}|p{1in}|p{1in}|p{1in}|p{1in}|} \hline
$\tau / h$ &1.000e-01 &1.000e-02 &1.000e-03 &1.000e-04 \\ \hline 
1.000e-01 & $2.420e+07$  $8.931e+06$  $1.527e+08$  & $4.162e+08$  $1.015e+08$  $1.506e+10$  & $1.769e+11$  $1.181e+10$  $1.676e+13$  & $5.444e+10$  $5.464e+08$  $7.729e+12$  \\ \hline 
1.000e-02 & $5.710e+16$  $2.840e+16$  $5.633e+17$  & $5.663e+59$  $8.800e+58$  $1.602e+61$  & $2.454e+91$  $1.791e+90$  $2.464e+93$  & $3.128e+175$  $inf$  $inf$  \\ \hline 
1.000e-03 & $3.881e+00$  $1.149e+00$  $1.627e+01$  & $2.232e+216$  $inf$  $inf$  & $nan$  $-nan$  $-nan$  & $nan$  $-nan$  $-nan$  \\ \hline 
1.000e-04 & $2.368e+00$  $8.436e-01$  $1.160e+01$  & $4.524e-03$  $2.327e-03$  $2.437e-02$  & $nan$  $-nan$  $-nan$  & $nan$  $-nan$  $-nan$  \\ \hline 

\end{tabular}\\[20pt]
\end{center}


\newpage
\begin{center}
Table of norms for V. $\mu = 0.1000$ \, $C = 10.0000$, $\gamma = 1.0000$
  
\begin{tabular}{|p{1in}|p{1in}|p{1in}|p{1in}|p{1in}|} \hline
$\tau / h$ &1.000e-01 &1.000e-02 &1.000e-03 &1.000e-04 \\ \hline 
1.000e-01 & $2.738e+01$  $1.215e+01$  $1.757e+02$  & $1.947e+01$  $7.508e+00$  $1.838e+02$  & $2.007e+01$  $1.102e+01$  $1.134e+02$  & $3.650e+00$  $2.328e+00$  $2.934e+01$  \\ \hline 
1.000e-02 & $2.981e+01$  $2.047e+01$  $3.420e+02$  & $2.449e+01$  $7.235e+00$  $9.146e+02$  & $5.059e+02$  $3.319e+01$  $4.690e+04$  & $2.571e+01$  $1.635e+01$  $6.742e+01$  \\ \hline 
1.000e-03 & $5.927e+00$  $2.487e+00$  $2.139e+01$  & $4.360e-03$  $2.429e-03$  $2.671e-02$  & $3.065e-03$  $1.361e-03$  $9.235e-03$  & $3.059e-03$  $1.358e-03$  $9.250e-03$  \\ \hline 
1.000e-04 & $2.395e+00$  $1.058e+00$  $8.296e+00$  & $3.182e-03$  $1.827e-03$  $2.550e-02$  & $3.125e-04$  $1.387e-04$  $9.413e-04$  & $3.066e-04$  $1.347e-04$  $9.258e-04$  \\ \hline 

\end{tabular}\\[20pt]
\end{center}

\begin{center}
Table of norms for H. $\mu = 0.1000$ \, $C = 1.0000$, $\gamma = 1.0000$
  
\begin{tabular}{|p{1in}|p{1in}|p{1in}|p{1in}|p{1in}|} \hline
$\tau / h$ &1.000e-01 &1.000e-02 &1.000e-03 &1.000e-04 \\ \hline 
1.000e-01 & $3.067e+03$  $1.629e+03$  $9.858e+03$  & $1.389e+03$  $3.821e+02$  $5.319e+04$  & $4.085e+06$  $1.976e+05$  $2.779e+08$  & $7.405e+06$  $1.222e+06$  $1.732e+10$  \\ \hline 
1.000e-02 & $1.747e+07$  $4.378e+06$  $7.831e+07$  & $2.420e+38$  $2.541e+37$  $3.981e+39$  & $8.977e+68$  $1.280e+68$  $1.862e+71$  & $4.359e+110$  $2.220e+109$  $3.112e+113$  \\ \hline 
1.000e-03 & $4.465e+09$  $1.031e+09$  $2.376e+10$  & $nan$  $-nan$  $-nan$  & $nan$  $-nan$  $-nan$  & $nan$  $-nan$  $-nan$  \\ \hline 
1.000e-04 & $1.774e+09$  $4.033e+08$  $6.395e+09$  & $1.484e-02$  $5.417e-03$  $1.140e-01$  & $nan$  $-nan$  $-nan$  & $nan$  $-nan$  $-nan$  \\ \hline 

\end{tabular}\\[20pt]
\end{center}


\newpage
\begin{center}
Table of norms for V. $\mu = 0.1000$ \, $C = 1.0000$, $\gamma = 1.0000$
  
\begin{tabular}{|p{1in}|p{1in}|p{1in}|p{1in}|} \hline
$k / \tau = h$ &1.000e-01 &1.000e-02 &1.000e-03 \\ \hline 
0.000e+00 & $5.855e+00$  $3.789e+00$  $6.347e+01$  & $2.318e-01$  $7.808e-02$  $1.927e+00$  & $1.392e-02$  $5.080e-03$  $9.453e-02$  \\ \hline 
1.000e+00 & $6.710e+01$  $3.127e+01$  $2.341e+02$  & $3.701e-02$  $1.330e-02$  $2.537e-01$  & $3.365e-03$  $1.242e-03$  $2.281e-02$  \\ \hline 
2.000e+00 & $3.947e+00$  $2.138e+00$  $9.851e+00$  & $1.754e-02$  $6.397e-03$  $1.189e-01$  & $1.673e-03$  $6.188e-04$  $1.134e-02$  \\ \hline 
3.000e+00 & $9.862e-02$  $4.161e-02$  $5.053e-01$  & $8.544e-03$  $3.140e-03$  $5.773e-02$  & $8.341e-04$  $3.088e-04$  $5.653e-03$  \\ \hline 

\end{tabular}\\[20pt]
\end{center}

\begin{center}
Table of norms for H. $\mu = 0.1000$ \, $C = 1.0000$, $\gamma = 1.4000$
  
\begin{tabular}{|p{1in}|p{1in}|p{1in}|p{1in}|p{1in}|} \hline
$\tau / h$ &1.000e-01 &1.000e-02 &1.000e-03 &1.000e-04 \\ \hline 
1.000e-01 & $nan$  $nan$  $nan$  & $nan$  $-nan$  $-nan$  & $nan$  $-nan$  $-nan$  & $nan$  $-nan$  $-nan$  \\ \hline 
1.000e-02 & $nan$  $-nan$  $-nan$  & $1.685e+00$  $2.629e-01$  $1.310e+01$  & $9.675e-01$  $1.687e-01$  $1.871e+01$  & $1.351e+00$  $1.848e-01$  $2.235e+01$  \\ \hline 
1.000e-03 & $nan$  $nan$  $nan$  & $8.272e-03$  $4.593e-03$  $4.257e-02$  & $1.074e-02$  $4.336e-03$  $5.749e-02$  & $1.077e-02$  $4.342e-03$  $5.779e-02$  \\ \hline 
1.000e-04 & $nan$  $nan$  $nan$  & $5.740e-03$  $2.601e-03$  $3.858e-02$  & $1.004e-03$  $4.176e-04$  $5.325e-03$  & $1.031e-03$  $4.221e-04$  $5.625e-03$  \\ \hline 

\end{tabular}\\[20pt]
\end{center}


\newpage
\begin{center}
Table of norms for V. $\mu = 0.1000$ \, $C = 1.0000$, $\gamma = 1.4000$
  
\begin{tabular}{|p{1in}|p{1in}|p{1in}|p{1in}|} \hline
$k / \tau = h$ &1.000e-01 &1.000e-02 &1.000e-03 \\ \hline 
0.000e+00 & $nan$  $-nan$  $-nan$  & $1.079e-01$  $4.405e-02$  $1.733e+00$  & $3.227e-03$  $1.517e-03$  $2.056e-02$  \\ \hline 
1.000e+00 & $nan$  $-nan$  $-nan$  & $8.190e-03$  $3.946e-03$  $5.295e-02$  & $8.015e-04$  $3.715e-04$  $5.057e-03$  \\ \hline 
2.000e+00 & $nan$  $-nan$  $-nan$  & $4.044e-03$  $1.909e-03$  $2.579e-02$  & $4.003e-04$  $1.851e-04$  $2.522e-03$  \\ \hline 
3.000e+00 & $2.354e-02$  $1.143e-02$  $1.210e-01$  & $2.010e-03$  $9.385e-04$  $1.272e-02$  & $2.001e-04$  $9.241e-05$  $1.259e-03$  \\ \hline 

\end{tabular}\\[20pt]
\end{center}


\subsection{Выводы}
По результатам численного эксперимента можно сделать вывод, что схема является условно сходящейся. Также, обратим внимание на то, что сходимость сильно зависит от $C$, $\gamma$, $\mu$. Худшая сходимость при больших $С$ и маленьких $\mu$. Обратим внимание на то, что при $\tau < h$ невзяки наименьшие. Сходимость схемы порядка $\tau + h^2$

\newpage
\subsubsection{Вложенная сетка}
За $k$ обозначим степень разбиения сетки, т.е в строках таблиц будут находится результаты вычислений с измельчениями по $T$ и $X$, $\frac{\tau}{2^k}$ и $\frac{h}{2^k}$ соотвественно.
\begin{center}
Table of times
\\[2.0ex]  
  
\begin{tabular}{|c|c|c|c|c|} \hline
$\tau / h$ &1.000e-01 &1.000e-02 &1.000e-03 &1.000e-04 \\ \hline 
1.000e-01 &8.580e-04 &1.526e-03 &1.949e-02 &2.003e-01 \\ \hline 
1.000e-02 &1.016e-03 &1.885e-02 &1.555e-01 &8.970e-01 \\ \hline 
1.000e-03 &2.114e-02 &1.974e-01 &7.778e-01 &3.687e+00 \\ \hline 
1.000e-04 &2.013e-01 &9.031e-01 &3.373e+00 &1.511e+01 \\ \hline 

\end{tabular}\\[20pt]
\end{center}


\begin{center}
Table of norms for H. $\mu = 0.0010$ \, $C = 100.0000$, $\gamma = 1.0000$
  
\begin{tabular}{|p{1in}|p{1in}|p{1in}|p{1in}|p{1in}|} \hline
$\tau / h$ &1.000e-01 &1.000e-02 &1.000e-03 &1.000e-04 \\ \hline 
1.000e-01 & $2.459e+09$  $1.271e+09$  $1.787e+10$  & $2.617e+14$  $1.726e+14$  $2.442e+16$  & $5.237e+09$  $1.051e+09$  $1.478e+12$  & $1.501e+15$  $1.246e+13$  $1.762e+17$  \\ \hline 
1.000e-02 & $2.295e+47$  $9.846e+46$  $2.914e+47$  & $1.725e+89$  $4.601e+88$  $7.394e+90$  & $6.910e+94$  $8.469e+93$  $1.209e+97$  & $5.618e+174$  $inf$  $inf$  \\ \hline 
1.000e-03 & $4.877e+259$  $inf$  $inf$  & $nan$  $-nan$  $-nan$  & $nan$  $-nan$  $-nan$  & $nan$  $-nan$  $-nan$  \\ \hline 
1.000e-04 & $nan$  $-nan$  $-nan$  & $nan$  $-nan$  $-nan$  & $nan$  $-nan$  $-nan$  & $nan$  $-nan$  $-nan$  \\ \hline 

\end{tabular}\\[20pt]
\end{center}

\newpage

\begin{center}
Table of norms for H. $\mu = 0.0010$ \, $C = 100.0000$, $\gamma = 1.0000$
  
\begin{tabular}{|p{1in}|p{1in}|p{1in}|p{1in}|p{1in}|} \hline
$\tau / h$ &1.000e-01 &1.000e-02 &1.000e-03 &1.000e-04 \\ \hline 
1.000e-01 & $8.432e+01$  $4.331e+01$  $6.501e+02$  & $5.324e+02$  $1.322e+02$  $1.830e+04$  & $1.777e+07$  $3.365e+06$  $5.984e+09$  & $6.521e+07$  $1.029e+06$  $1.502e+10$  \\ \hline 
1.000e-02 & $6.617e+01$  $3.743e+01$  $5.206e+02$  & $5.845e+02$  $1.259e+02$  $2.333e+04$  & $3.629e+03$  $2.689e+02$  $4.544e+05$  & $4.387e+02$  $1.104e+02$  $1.412e+06$  \\ \hline 
1.000e-03 & $3.510e+02$  $2.474e+02$  $3.672e+03$  & $nan$  $-nan$  $-nan$  & $nan$  $nan$  $nan$  & $nan$  $nan$  $nan$  \\ \hline 
1.000e-04 & $nan$  $-nan$  $-nan$  & $nan$  $nan$  $nan$  & $nan$  $-nan$  $-nan$  & $nan$  $nan$  $nan$  \\ \hline 

\end{tabular}\\[20pt]
\end{center}

\newpage
\begin{center}
Table of norms for H. $\mu = 0.0010$ \, $C = 10.0000$, $\gamma = 1.0000$
  
\begin{tabular}{|p{1in}|p{1in}|p{1in}|p{1in}|} \hline
$k / \tau = h$ &1.000e-01 &1.000e-02 &1.000e-03 \\ \hline 
0.000e+00 & $7.614e+04$  $3.000e+04$  $1.997e+05$  & $9.181e+60$  $1.911e+60$  $2.248e+62$  & $nan$  $-nan$  $-nan$  \\ \hline 
1.000e+00 & $4.572e+11$  $1.272e+11$  $9.031e+11$  & $1.543e+114$  $1.092e+113$  $1.506e+115$  & $nan$  $nan$  $nan$  \\ \hline 
2.000e+00 & $2.800e+23$  $6.592e+22$  $1.097e+24$  & $1.253e+227$  $inf$  $inf$  & $nan$  $-nan$  $-nan$  \\ \hline 
3.000e+00 & $2.884e+49$  $6.678e+48$  $7.670e+49$  & $nan$  $nan$  $nan$  & $nan$  $nan$  $nan$  \\ \hline 
4.000e+00 & $4.371e+91$  $1.075e+91$  $1.399e+92$  & $nan$  $nan$  $nan$  & $nan$  $nan$  $nan$  \\ \hline 

\end{tabular}\\[20pt]
\end{center}

\newpage
\begin{center}
Table of norms for H. $\mu = 0.0010$ \, $C = 10.0000$, $\gamma = 1.0000$
  
\begin{tabular}{|p{1in}|p{1in}|p{1in}|p{1in}|p{1in}|} \hline
$k / \tau = h$ &1.000e-01 &1.000e-02 &1.000e-03 &1.000e-04 \\ \hline 
0.000e+00 & $1.819e+01$  $9.487e+00$  $1.608e+02$  & $4.016e+01$  $1.084e+01$  $1.486e+03$  & $nan$  $-nan$  $-nan$  & $nan$  $-nan$  $-nan$  \\ \hline 
1.000e+00 & $3.509e+01$  $8.056e+00$  $4.023e+02$  & $7.851e+02$  $4.521e+01$  $2.854e+04$  & $nan$  $-nan$  $-nan$  & $nan$  $-nan$  $-nan$  \\ \hline 
2.000e+00 & $6.363e+02$  $7.266e+01$  $8.018e+03$  & $nan$  $-nan$  $-nan$  & $nan$  $-nan$  $-nan$  & $nan$  $-nan$  $-nan$  \\ \hline 
3.000e+00 & $1.456e+02$  $1.403e+01$  $3.668e+03$  & $nan$  $-nan$  $-nan$  & $nan$  $-nan$  $-nan$  & $nan$  $-nan$  $-nan$  \\ \hline 

\end{tabular}\\[20pt]
\end{center}

\newpage
\begin{center}
Table of norms for H. $\mu = 0.0010$ \, $C = 1.0000$, $\gamma = 1.0000$
  
\begin{tabular}{|p{1in}|p{1in}|p{1in}|p{1in}|} \hline
$k / \tau = h$ &1.000e-01 &1.000e-02 &1.000e-03 \\ \hline 
0.000e+00 & $1.140e+02$  $4.643e+01$  $6.830e+02$  & $2.865e+25$  $3.835e+24$  $7.577e+26$  & $6.365e+257$  $inf$  $inf$  \\ \hline 
1.000e+00 & $2.711e+07$  $7.008e+06$  $6.578e+07$  & $2.876e+116$  $2.036e+115$  $2.955e+117$  & $1.706e+41$  $6.655e+39$  $3.932e+42$  \\ \hline 
2.000e+00 & $3.570e+20$  $8.480e+19$  $1.229e+21$  & $1.518e+182$  $inf$  $inf$  & $1.611e-02$  $2.921e-03$  $3.865e-01$  \\ \hline 
3.000e+00 & $1.044e+50$  $2.336e+49$  $3.300e+50$  & $nan$  $-nan$  $-nan$  & $7.515e-03$  $1.434e-03$  $1.781e-01$  \\ \hline 

\end{tabular}\\[20pt]
\end{center}

\newpage
\begin{center}
Table of norms for H. $\mu = 0.0010$ \, $C = 1.0000$, $\gamma = 1.0000$
  
\begin{tabular}{|p{1in}|p{1in}|p{1in}|p{1in}|p{1in}|} \hline
$k / \tau = h$ &1.000e-01 &1.000e-02 &1.000e-03 &1.000e-04 \\ \hline 
0.000e+00 & $8.932e+00$  $5.407e+00$  $9.547e+01$  & $3.517e+01$  $8.779e+00$  $1.286e+03$  & $nan$  $-nan$  $-nan$  & $nan$  $-nan$  $-nan$  \\ \hline 
1.000e+00 & $6.414e+00$  $1.548e+00$  $1.019e+02$  & $1.870e+02$  $1.009e+01$  $6.358e+03$  & $nan$  $-nan$  $-nan$  & $nan$  $-nan$  $-nan$  \\ \hline 
2.000e+00 & $1.052e+01$  $1.766e+00$  $1.631e+02$  & $5.731e+01$  $3.071e+00$  $3.337e+03$  & $nan$  $-nan$  $-nan$  & $nan$  $-nan$  $-nan$  \\ \hline 
3.000e+00 & $8.281e+00$  $9.728e-01$  $2.262e+02$  & $nan$  $-nan$  $-nan$  & $nan$  $-nan$  $-nan$  & $nan$  $-nan$  $-nan$  \\ \hline 

\end{tabular}\\[20pt]
\end{center}

\newpage
\begin{center}
Table of norms for H. $\mu = 0.0010$ \, $C = 1.0000$, $\gamma = 1.4000$
  
\begin{tabular}{|p{1in}|p{1in}|p{1in}|p{1in}|p{1in}|} \hline
$\tau / h$ &1.000e-01 &1.000e-02 &1.000e-03 &1.000e-04 \\ \hline 
1.000e-01 & $nan$  $-nan$  $-nan$  & $nan$  $-nan$  $-nan$  & $nan$  $-nan$  $-nan$  & $nan$  $-nan$  $-nan$  \\ \hline 
1.000e-02 & $nan$  $-nan$  $-nan$  & $nan$  $-nan$  $-nan$  & $nan$  $-nan$  $-nan$  & $nan$  $-nan$  $-nan$  \\ \hline 
1.000e-03 & $nan$  $-nan$  $-nan$  & $nan$  $-nan$  $-nan$  & $nan$  $-nan$  $-nan$  & $nan$  $-nan$  $-nan$  \\ \hline 
1.000e-04 & $nan$  $-nan$  $-nan$  & $1.498e-02$  $4.873e-03$  $5.353e-01$  & $6.867e-04$  $2.722e-04$  $5.344e-03$  & $nan$  $-nan$  $-nan$  \\ \hline 

\end{tabular}\\[20pt]
\end{center}

\newpage
\begin{center}
Table of norms for V. $\mu = 0.0010$ \, $C = 1.0000$, $\gamma = 1.4000$
  
\begin{tabular}{|p{1in}|p{1in}|p{1in}|p{1in}|p{1in}|} \hline
$\tau / h$ &1.000e-01 &1.000e-02 &1.000e-03 &1.000e-04 \\ \hline 
1.000e-01 & $nan$  $-nan$  $-nan$  & $nan$  $-nan$  $-nan$  & $nan$  $-nan$  $-nan$  & $nan$  $-nan$  $-nan$  \\ \hline 
1.000e-02 & $nan$  $-nan$  $-nan$  & $nan$  $-nan$  $-nan$  & $nan$  $-nan$  $-nan$  & $nan$  $-nan$  $-nan$  \\ \hline 
1.000e-03 & $nan$  $-nan$  $-nan$  & $nan$  $-nan$  $-nan$  & $nan$  $-nan$  $-nan$  & $nan$  $-nan$  $-nan$  \\ \hline 
1.000e-04 & $nan$  $-nan$  $-nan$  & $7.856e-03$  $2.837e-03$  $2.038e-01$  & $2.161e-04$  $1.035e-04$  $1.978e-03$  & $nan$  $-nan$  $-nan$  \\ \hline 

\end{tabular}\\[20pt]
\end{center}

\newpage
\begin{center}
Table of norms for H. $\mu = 0.0100$ \, $C = 100.0000$, $\gamma = 1.0000$
  
\begin{tabular}{|p{1in}|p{1in}|p{1in}|p{1in}|p{1in}|} \hline
$\tau / h$ &1.000e-01 &1.000e-02 &1.000e-03 &1.000e-04 \\ \hline 
1.000e-01 & $1.989e+10$  $8.497e+09$  $1.238e+11$  & $2.114e+14$  $4.989e+13$  $7.172e+15$  & $9.311e+16$  $2.483e+16$  $3.521e+19$  & $1.145e+20$  $7.159e+19$  $1.013e+24$  \\ \hline 
1.000e-02 & $3.002e+50$  $1.631e+50$  $3.624e+51$  & $7.628e+80$  $1.226e+80$  $1.901e+82$  & $9.421e+132$  $7.884e+131$  $1.125e+135$  & $1.178e+169$  $inf$  $inf$  \\ \hline 
1.000e-03 & $4.218e+263$  $inf$  $inf$  & $nan$  $-nan$  $-nan$  & $nan$  $-nan$  $-nan$  & $nan$  $-nan$  $-nan$  \\ \hline 
1.000e-04 & $nan$  $-nan$  $-nan$  & $nan$  $-nan$  $-nan$  & $nan$  $-nan$  $-nan$  & $nan$  $-nan$  $-nan$  \\ \hline 

\end{tabular}\\[20pt]
\end{center}

\newpage
\begin{center}
Table of norms for H. $\mu = 0.0100$ \, $C = 100.0000$, $\gamma = 1.0000$
  
\begin{tabular}{|p{1in}|p{1in}|p{1in}|p{1in}|p{1in}|} \hline
$\tau / h$ &1.000e-01 &1.000e-02 &1.000e-03 &1.000e-04 \\ \hline 
1.000e-01 & $2.359e+02$  $9.856e+01$  $1.597e+03$  & $7.350e+03$  $9.909e+02$  $1.700e+05$  & $2.120e+05$  $2.629e+04$  $3.718e+07$  & $1.845e+08$  $2.867e+06$  $4.009e+10$  \\ \hline 
1.000e-02 & $6.080e+01$  $4.236e+01$  $6.189e+02$  & $1.294e+03$  $2.064e+02$  $3.088e+04$  & $1.108e+04$  $5.357e+02$  $7.637e+05$  & $4.299e+03$  $2.828e+02$  $4.900e+06$  \\ \hline 
1.000e-03 & $3.524e+02$  $2.480e+02$  $3.690e+03$  & $nan$  $nan$  $nan$  & $nan$  $nan$  $nan$  & $nan$  $nan$  $nan$  \\ \hline 
1.000e-04 & $nan$  $-nan$  $-nan$  & $nan$  $nan$  $nan$  & $nan$  $nan$  $nan$  & $nan$  $nan$  $nan$  \\ \hline 

\end{tabular}\\[20pt]
\end{center}

\newpage
\begin{center}
Table of norms for H. $\mu = 0.0100$ \, $C = 10.0000$, $\gamma = 1.0000$
  
\begin{tabular}{|p{1in}|p{1in}|p{1in}|p{1in}|p{1in}|} \hline
$k / \tau = h$ &1.000e-01 &1.000e-02 &1.000e-03 &1.000e-04 \\ \hline 
0.000e+00 & $7.870e+04$  $2.214e+04$  $3.880e+05$  & $8.442e+55$  $9.029e+54$  $1.630e+57$  & $nan$  $-nan$  $-nan$  & $nan$  $-nan$  $-nan$  \\ \hline 
1.000e+00 & $6.980e+24$  $7.809e+23$  $6.386e+25$  & $6.061e+254$  $inf$  $inf$  & $nan$  $-nan$  $-nan$  & $nan$  $-nan$  $-nan$  \\ \hline 
2.000e+00 & $2.057e+48$  $1.663e+47$  $1.740e+49$  & $nan$  $-nan$  $-nan$  & $nan$  $-nan$  $-nan$  & $nan$  $-nan$  $-nan$  \\ \hline 
3.000e+00 & $1.772e+106$  $1.218e+105$  $3.221e+107$  & $nan$  $-nan$  $-nan$  & $nan$  $-nan$  $-nan$  & $nan$  $-nan$  $-nan$  \\ \hline 

\end{tabular}\\[20pt]
\end{center}

\newpage
\begin{center}
Table of norms for H. $\mu = 0.0100$ \, $C = 10.0000$, $\gamma = 1.0000$
  
\begin{tabular}{|p{1in}|p{1in}|p{1in}|p{1in}|p{1in}|} \hline
$\tau / h$ &1.000e-01 &1.000e-02 &1.000e-03 &1.000e-04 \\ \hline 
1.000e-01 & $2.159e+01$  $9.587e+00$  $1.198e+02$  & $2.471e+02$  $3.984e+01$  $4.609e+03$  & $2.133e+05$  $8.967e+03$  $1.237e+07$  & $1.928e+08$  $3.463e+06$  $4.953e+10$  \\ \hline 
1.000e-02 & $3.757e+01$  $1.960e+01$  $2.769e+02$  & $7.268e+03$  $7.591e+02$  $1.296e+05$  & $1.227e+05$  $5.647e+03$  $6.299e+06$  & $3.534e+04$  $1.529e+03$  $2.238e+07$  \\ \hline 
1.000e-03 & $8.250e+01$  $5.484e+01$  $9.331e+02$  & $5.382e+01$  $2.538e+01$  $3.733e+03$  & $nan$  $-nan$  $-nan$  & $nan$  $nan$  $nan$  \\ \hline 
1.000e-04 & $6.557e+01$  $4.134e+01$  $6.523e+02$  & $4.257e-03$  $2.137e-03$  $3.011e-02$  & $nan$  $-nan$  $-nan$  & $nan$  $nan$  $nan$  \\ \hline 

\end{tabular}\\[20pt]
\end{center}

\newpage
\begin{center}
Table of norms for H. $\mu = 0.0100$ \, $C = 1.0000$, $\gamma = 1.0000$
  
\begin{tabular}{|p{1in}|p{1in}|p{1in}|p{1in}|p{1in}|} \hline
$k / \tau = h$ &1.000e-01 &1.000e-02 &1.000e-03 &1.000e-04 \\ \hline 
0.000e+00 & $6.976e+01$  $3.338e+01$  $6.133e+02$  & $1.138e+36$  $1.241e+35$  $2.314e+37$  & $nan$  $-nan$  $-nan$  & $nan$  $-nan$  $-nan$  \\ \hline 
1.000e+00 & $2.047e+09$  $2.297e+08$  $1.740e+10$  & $1.040e+154$  $3.849e+152$  $inf$  & $nan$  $-nan$  $-nan$  & $nan$  $-nan$  $-nan$  \\ \hline 
2.000e+00 & $1.520e+18$  $1.829e+17$  $1.101e+19$  & $2.689e+271$  $inf$  $inf$  & $nan$  $-nan$  $-nan$  & $nan$  $-nan$  $-nan$  \\ \hline 
3.000e+00 & $3.045e+39$  $1.702e+38$  $3.889e+40$  & $nan$  $-nan$  $-nan$  & $nan$  $-nan$  $-nan$  & $nan$  $-nan$  $-nan$  \\ \hline 

\end{tabular}\\[20pt]
\end{center}

\newpage
\begin{center}
Table of norms for V. $\mu = 0.0100$ \, $C = 1.0000$, $\gamma = 1.0000$
  
\begin{tabular}{|p{0.8in}|p{0.8in}|p{0.8in}|p{0.8in}|p{0.8in}|p{0.8in}|p{0.8in}|} \hline
$K$ &$N_0$ &$N_0 \tau$ &$n = \frac{N_0}{4}$ &$n = \frac{N_0}{2}$ &$n = \frac{3N_0}{4}$ &$n = N_0$ \\ \hline 
0 &943131 &9.431e+02 &3.344e-02 &1.147e-02 &4.332e-03 &9.996e-04 \\ \hline 
1 &1886261 &9.431e+02 &3.311e-02 &1.129e-02 &4.185e-03 &9.050e-04 \\ \hline 
2 &3772521 &9.431e+02 &3.293e-02 &1.120e-02 &4.114e-03 &8.613e-04 \\ \hline 
3 &7545041 &9.431e+02 &3.285e-02 &1.115e-02 &4.079e-03 &8.405e-04 \\ \hline 

\end{tabular}\\[20pt]
\end{center}

\newpage
\begin{center}
Table of norms for H. $\mu = 0.0100$ \, $C = 1.0000$, $\gamma = 1.4000$
  
\begin{tabular}{|p{1in}|p{1in}|p{1in}|p{1in}|} \hline
$k / \tau = h$ &1.000e-01 &1.000e-02 &1.000e-03 \\ \hline 
0.000e+00 & $nan$  $nan$  $nan$  & $nan$  $nan$  $nan$  & $7.351e-03$  $2.605e-03$  $5.011e-02$  \\ \hline 
1.000e+00 & $nan$  $nan$  $nan$  & $nan$  $-nan$  $-nan$  & $3.532e-03$  $1.278e-03$  $2.051e-02$  \\ \hline 
2.000e+00 & $nan$  $nan$  $nan$  & $nan$  $nan$  $nan$  & $1.733e-03$  $6.340e-04$  $1.002e-02$  \\ \hline 
3.000e+00 & $nan$  $-nan$  $-nan$  & $nan$  $nan$  $nan$  & $8.586e-04$  $3.159e-04$  $4.956e-03$  \\ \hline 
4.000e+00 & $nan$  $-nan$  $-nan$  & $4.447e-03$  $1.604e-03$  $2.586e-02$  & $4.274e-04$  $1.577e-04$  $2.466e-03$  \\ \hline 

\end{tabular}\\[20pt]
\end{center}

\newpage
\begin{center}
Table of norms for V. $\mu = 0.0100$ \, $C = 1.0000$, $\gamma = 1.4000$
  
\begin{tabular}{|p{0.8in}|p{0.8in}|p{0.8in}|p{0.8in}|p{0.8in}|p{0.8in}|p{0.8in}|} \hline
$K$ &$N_0$ &$N_0 \tau$ &$n = \frac{N_0}{4}$ &$n = \frac{N_0}{2}$ &$n = \frac{3N_0}{4}$ &$n = N_0$ \\ \hline 
0 &919330 &9.193e+02 &1.980e-02 &8.076e-03 &6.651e-03 &9.998e-04 \\ \hline 
1 &1838659 &9.193e+02 &1.891e-02 &8.395e-03 &5.863e-03 &1.467e-03 \\ \hline 
2 &3677317 &9.193e+02 &1.853e-02 &8.487e-03 &5.518e-03 &1.632e-03 \\ \hline 
3 &7354633 &9.193e+02 &1.835e-02 &8.521e-03 &5.358e-03 &1.701e-03 \\ \hline 

\end{tabular}\\[20pt]
\end{center}

\newpage
\begin{center}
Table of norms for H. $\mu = 0.1000$ \, $C = 100.0000$, $\gamma = 1.0000$
  
\begin{tabular}{|p{1in}|p{1in}|p{1in}|p{1in}|p{1in}|} \hline
$\tau / h$ &1.000e-01 &1.000e-02 &1.000e-03 &1.000e-04 \\ \hline 
1.000e-01 & $1.091e+08$  $3.948e+07$  $7.057e+08$  & $1.747e+05$  $4.836e+04$  $7.969e+06$  & $8.058e+04$  $5.224e+03$  $7.998e+06$  & $1.843e+04$  $2.690e+02$  $4.799e+06$  \\ \hline 
1.000e-02 & $7.055e+31$  $3.830e+31$  $8.213e+32$  & $1.894e+67$  $2.335e+66$  $4.199e+68$  & $1.794e+101$  $6.181e+99$  $8.807e+102$  & $4.378e+114$  $3.200e+113$  $4.742e+117$  \\ \hline 
1.000e-03 & $7.174e+166$  $inf$  $inf$  & $nan$  $-nan$  $-nan$  & $nan$  $-nan$  $-nan$  & $nan$  $-nan$  $-nan$  \\ \hline 
1.000e-04 & $1.348e+232$  $inf$  $inf$  & $3.819e-03$  $2.353e-03$  $2.399e-02$  & $1.866e-04$  $1.346e-04$  $4.009e-04$  & $2.248e-04$  $1.428e-04$  $5.979e-04$  \\ \hline 

\end{tabular}\\[20pt]
\end{center}

\newpage
\begin{center}
Table of norms for H. $\mu = 0.1000$ \, $C = 100.0000$, $\gamma = 1.0000$
  
\begin{tabular}{|p{1in}|p{1in}|p{1in}|p{1in}|p{1in}|} \hline
$\tau / h$ &1.000e-01 &1.000e-02 &1.000e-03 &1.000e-04 \\ \hline 
1.000e-01 & $1.396e+02$  $7.026e+01$  $1.006e+03$  & $6.052e+04$  $1.539e+04$  $2.373e+06$  & $4.018e+03$  $3.036e+02$  $4.364e+05$  & $3.551e+06$  $4.198e+04$  $5.933e+08$  \\ \hline 
1.000e-02 & $7.060e+01$  $4.460e+01$  $6.486e+02$  & $1.619e+03$  $2.911e+02$  $4.415e+04$  & $4.477e+03$  $6.658e+02$  $9.019e+05$  & $1.287e+05$  $4.590e+03$  $6.990e+07$  \\ \hline 
1.000e-03 & $3.685e+02$  $2.594e+02$  $3.850e+03$  & $nan$  $-nan$  $-nan$  & $nan$  $-nan$  $-nan$  & $nan$  $-nan$  $-nan$  \\ \hline 
1.000e-04 & $3.042e+02$  $2.134e+02$  $3.178e+03$  & $3.322e-03$  $1.908e-03$  $2.599e-02$  & $nan$  $-nan$  $-nan$  & $nan$  $-nan$  $-nan$  \\ \hline 

\end{tabular}\\[20pt]
\end{center}

\newpage
\begin{center}
Table of norms for H. $\mu = 0.1000$ \, $C = 10.0000$, $\gamma = 1.0000$
  
\begin{tabular}{|p{1in}|p{1in}|p{1in}|p{1in}|p{1in}|} \hline
$\tau / h$ &1.000e-01 &1.000e-02 &1.000e-03 &1.000e-04 \\ \hline 
1.000e-01 & $2.420e+07$  $8.931e+06$  $1.527e+08$  & $4.162e+08$  $1.015e+08$  $1.506e+10$  & $1.769e+11$  $1.181e+10$  $1.676e+13$  & $5.444e+10$  $5.464e+08$  $7.729e+12$  \\ \hline 
1.000e-02 & $5.710e+16$  $2.840e+16$  $5.633e+17$  & $5.663e+59$  $8.800e+58$  $1.602e+61$  & $2.454e+91$  $1.791e+90$  $2.464e+93$  & $3.128e+175$  $inf$  $inf$  \\ \hline 
1.000e-03 & $3.881e+00$  $1.149e+00$  $1.627e+01$  & $2.232e+216$  $inf$  $inf$  & $nan$  $-nan$  $-nan$  & $nan$  $-nan$  $-nan$  \\ \hline 
1.000e-04 & $2.368e+00$  $8.436e-01$  $1.160e+01$  & $4.524e-03$  $2.327e-03$  $2.437e-02$  & $nan$  $-nan$  $-nan$  & $nan$  $-nan$  $-nan$  \\ \hline 

\end{tabular}\\[20pt]
\end{center}

\newpage
\begin{center}
Table of norms for V. $\mu = 0.1000$ \, $C = 10.0000$, $\gamma = 1.0000$
  
\begin{tabular}{|p{1in}|p{1in}|p{1in}|p{1in}|p{1in}|} \hline
$\tau / h$ &1.000e-01 &1.000e-02 &1.000e-03 &1.000e-04 \\ \hline 
1.000e-01 & $2.738e+01$  $1.215e+01$  $1.757e+02$  & $1.947e+01$  $7.508e+00$  $1.838e+02$  & $2.007e+01$  $1.102e+01$  $1.134e+02$  & $3.650e+00$  $2.328e+00$  $2.934e+01$  \\ \hline 
1.000e-02 & $2.981e+01$  $2.047e+01$  $3.420e+02$  & $2.449e+01$  $7.235e+00$  $9.146e+02$  & $5.059e+02$  $3.319e+01$  $4.690e+04$  & $2.571e+01$  $1.635e+01$  $6.742e+01$  \\ \hline 
1.000e-03 & $5.927e+00$  $2.487e+00$  $2.139e+01$  & $4.360e-03$  $2.429e-03$  $2.671e-02$  & $3.065e-03$  $1.361e-03$  $9.235e-03$  & $3.059e-03$  $1.358e-03$  $9.250e-03$  \\ \hline 
1.000e-04 & $2.395e+00$  $1.058e+00$  $8.296e+00$  & $3.182e-03$  $1.827e-03$  $2.550e-02$  & $3.125e-04$  $1.387e-04$  $9.413e-04$  & $3.066e-04$  $1.347e-04$  $9.258e-04$  \\ \hline 

\end{tabular}\\[20pt]
\end{center}

\newpage
\begin{center}
Table of norms for H. $\mu = 0.1000$ \, $C = 1.0000$, $\gamma = 1.0000$
  
\begin{tabular}{|p{1in}|p{1in}|p{1in}|p{1in}|p{1in}|} \hline
$\tau / h$ &1.000e-01 &1.000e-02 &1.000e-03 &1.000e-04 \\ \hline 
1.000e-01 & $3.067e+03$  $1.629e+03$  $9.858e+03$  & $1.389e+03$  $3.821e+02$  $5.319e+04$  & $4.085e+06$  $1.976e+05$  $2.779e+08$  & $7.405e+06$  $1.222e+06$  $1.732e+10$  \\ \hline 
1.000e-02 & $1.747e+07$  $4.378e+06$  $7.831e+07$  & $2.420e+38$  $2.541e+37$  $3.981e+39$  & $8.977e+68$  $1.280e+68$  $1.862e+71$  & $4.359e+110$  $2.220e+109$  $3.112e+113$  \\ \hline 
1.000e-03 & $4.465e+09$  $1.031e+09$  $2.376e+10$  & $nan$  $-nan$  $-nan$  & $nan$  $-nan$  $-nan$  & $nan$  $-nan$  $-nan$  \\ \hline 
1.000e-04 & $1.774e+09$  $4.033e+08$  $6.395e+09$  & $1.484e-02$  $5.417e-03$  $1.140e-01$  & $nan$  $-nan$  $-nan$  & $nan$  $-nan$  $-nan$  \\ \hline 

\end{tabular}\\[20pt]
\end{center}

\newpage
\begin{center}
Table of norms for V. $\mu = 0.1000$ \, $C = 1.0000$, $\gamma = 1.0000$
  
\begin{tabular}{|p{1in}|p{1in}|p{1in}|p{1in}|} \hline
$k / \tau = h$ &1.000e-01 &1.000e-02 &1.000e-03 \\ \hline 
0.000e+00 & $5.855e+00$  $3.789e+00$  $6.347e+01$  & $2.318e-01$  $7.808e-02$  $1.927e+00$  & $1.392e-02$  $5.080e-03$  $9.453e-02$  \\ \hline 
1.000e+00 & $6.710e+01$  $3.127e+01$  $2.341e+02$  & $3.701e-02$  $1.330e-02$  $2.537e-01$  & $3.365e-03$  $1.242e-03$  $2.281e-02$  \\ \hline 
2.000e+00 & $3.947e+00$  $2.138e+00$  $9.851e+00$  & $1.754e-02$  $6.397e-03$  $1.189e-01$  & $1.673e-03$  $6.188e-04$  $1.134e-02$  \\ \hline 
3.000e+00 & $9.862e-02$  $4.161e-02$  $5.053e-01$  & $8.544e-03$  $3.140e-03$  $5.773e-02$  & $8.341e-04$  $3.088e-04$  $5.653e-03$  \\ \hline 

\end{tabular}\\[20pt]
\end{center}

\newpage
\begin{center}
Table of norms for H. $\mu = 0.1000$ \, $C = 1.0000$, $\gamma = 1.4000$
  
\begin{tabular}{|p{1in}|p{1in}|p{1in}|p{1in}|p{1in}|} \hline
$\tau / h$ &1.000e-01 &1.000e-02 &1.000e-03 &1.000e-04 \\ \hline 
1.000e-01 & $nan$  $nan$  $nan$  & $nan$  $-nan$  $-nan$  & $nan$  $-nan$  $-nan$  & $nan$  $-nan$  $-nan$  \\ \hline 
1.000e-02 & $nan$  $-nan$  $-nan$  & $1.685e+00$  $2.629e-01$  $1.310e+01$  & $9.675e-01$  $1.687e-01$  $1.871e+01$  & $1.351e+00$  $1.848e-01$  $2.235e+01$  \\ \hline 
1.000e-03 & $nan$  $nan$  $nan$  & $8.272e-03$  $4.593e-03$  $4.257e-02$  & $1.074e-02$  $4.336e-03$  $5.749e-02$  & $1.077e-02$  $4.342e-03$  $5.779e-02$  \\ \hline 
1.000e-04 & $nan$  $nan$  $nan$  & $5.740e-03$  $2.601e-03$  $3.858e-02$  & $1.004e-03$  $4.176e-04$  $5.325e-03$  & $1.031e-03$  $4.221e-04$  $5.625e-03$  \\ \hline 

\end{tabular}\\[20pt]
\end{center}

\newpage
\begin{center}
Table of norms for V. $\mu = 0.1000$ \, $C = 1.0000$, $\gamma = 1.4000$
  
\begin{tabular}{|p{1in}|p{1in}|p{1in}|p{1in}|} \hline
$k / \tau = h$ &1.000e-01 &1.000e-02 &1.000e-03 \\ \hline 
0.000e+00 & $nan$  $-nan$  $-nan$  & $1.079e-01$  $4.405e-02$  $1.733e+00$  & $3.227e-03$  $1.517e-03$  $2.056e-02$  \\ \hline 
1.000e+00 & $nan$  $-nan$  $-nan$  & $8.190e-03$  $3.946e-03$  $5.295e-02$  & $8.015e-04$  $3.715e-04$  $5.057e-03$  \\ \hline 
2.000e+00 & $nan$  $-nan$  $-nan$  & $4.044e-03$  $1.909e-03$  $2.579e-02$  & $4.003e-04$  $1.851e-04$  $2.522e-03$  \\ \hline 
3.000e+00 & $2.354e-02$  $1.143e-02$  $1.210e-01$  & $2.010e-03$  $9.385e-04$  $1.272e-02$  & $2.001e-04$  $9.241e-05$  $1.259e-03$  \\ \hline 

\end{tabular}\\[20pt]
\end{center}


\subsection{Выводы}
По результатам численного эксперимента со вложенными сетками можно увидеть оценку точности снизу для схемы для случаев, когда схема сходится. Наименьшая ошибка имеет порядок $10^{-5}$.