\section{Введение}
\subsection{Постановка задачи}

Рассмотрим систему уравнений, описывающую нестационарное  движение вязкого баротропного газа:

\begin{equation} \label{eq:task}
	\begin{cases}
		\begin{array}{l}
			\frac{\partial \rho}{\partial t} + div(\rho u) = \rho f_0\\
			\rho \left(\frac{\partial u}{\partial t} + (u, \triangledown)u\right) + \triangledown p = Lu + \rho f\\
			Lu = div(\mu \triangledown u) + \frac{1}{3}\triangledown(\mu div(u)) \\
			p = p(\rho)
		\end{array}
	\end{cases}
\end{equation}

В нашей задаче $L$ - линейный симметричный положительно определеннный оператор. Через $\mu$ обозначен коэффициент вязкости газа, который будем считать известной положительной константой. Известными также будем считать функцию давления газа $p$ (в данной работе будем рассматривать $p(\rho) = C\rho$, где $C$ - положительная константа) и вектор внешних сил $f$, где $f$ - функция переменных Эйлера: $$(t, \, x) \in Q = \Omega_t \times \Omega_x = [0; \, T] \times \mathbb{R}^d$$

Неизвестные функции: плотность $\rho$ и скорость $u = (u_1,\dots,u_d)$ также являются функциями переменных Эйлера.

Система \eqref{eq:task} дополнена начальными и граничными условиями:
\begin{equation} \label{eq:terms}
	\begin{array}{lc}
		(\rho, \, u)|_{t = 0} = (\rho_0, \, u_0), &\quad x \in [0; \, X] \\
		u (t, \, x) = 0, &\quad (t, \, x) \in [0; \, T] \times \partial \Omega_x
	\end{array}
\end{equation}

Перепишем систему $(1)$ в эквивалентный вид, при условии того, что мы рассматриваем двумерную по пространству задачу : 

\begin{equation} \label{eq:task_reformulate}
	\begin{cases}
		\begin{array}{l}
			\frac{\partial \rho}{\partial t} + \frac{\partial \rho u_1}{\partial x_1} + \frac{\partial \rho u_2}{\partial x_2} = f_0 \\
			\frac{\partial \rho u_1}{\partial t} + \frac{\partial \rho u_1^2}{\partial x_1} + \frac{\partial \rho u_2 u_1}{\partial x_2} + \frac{\partial p}{\partial x_1} = \mu \left( \frac{4}{3} \frac{\partial^2 u_1}{\partial x_1^2} + \frac{\partial^2 u_1}{\partial x_2^2} + \frac{1}{3} \frac{\partial ^2 u_2}{\partial x_1 \partial x_2} \right) + \rho f_1 \\
			\frac{\partial \rho u_2}{\partial t} + \frac{\partial \rho u_2^2}{\partial x_2} + \frac{\partial \rho u_2 u_1}{\partial x_1} + \frac{\partial p}{\partial x_2} = \mu \left( \frac{4}{3} \frac{\partial^2 u_2}{\partial x_2^2} + \frac{\partial^2 u_2}{\partial x_1^2} + \frac{1}{3} \frac{\partial^2 u_1}{\partial x_1 \partial x_2} \right) + \rho f_2
		\end{array}
	\end{cases}
\end{equation}

\subsection{Основные обозначения}
Введем на $\Omega_x = \Omega_{x_1} \times \dots \times \Omega_{x_d}$, где $\Omega_{x_s} = [0; X_s], \, s=1,\dots, d$ и $\Omega_t$ сетки:
\begin{equation}
	\begin{array}{lc}
		\omega_t = \{n\tau: n = 0, \dots, N\}, \tau = \frac{T}{N}\\
		\omega_{h_s} = \{mh_s: m = 0, \dots, M_s\}, h_s = \frac{X_s}{M_s}\\
	\end{array}
\end{equation}

Введем следующие обозначения:
\begin{equation}
	\begin{array}{lc}
		h=(h_1, \dots, h_d)\\
		\omega_h=\omega_{h_1}\times\dots\times \omega_{h_d}\\
		\omega_{\tau, h} = \omega_{\tau} \times \omega_h\\
		\gamma_{h,s}^- = \omega_{h_1} \times \dots \times \omega_{h_{s-1}}\times\{0\}\times \omega_{h_{s+1}} \dots \times \omega_{h_d} \\
		\gamma_{h,s}^+ = \omega_{h_1} \times \dots \times \omega_{h_{s-1}}\times\{X_s\}\times \omega_{h_{s+1}} \dots \times \omega_{h_d} \\
		\gamma_{h,s} = \gamma_{h,s}^- \cup \gamma_{h,s}^+ \\
		\gamma_{h} = \gamma_{h,1} \cup \dots \cup \gamma_{h,d}
	\end{array}
\end{equation}

Для сокращения записи значение обозначим $m=(m_1,\dots, m_d)$, $m \pm q_s = (m_1, \dots, m_{s-1}, m\pm q, m_{s+1}, \dots, m_d)$, значение для произвольной функции $f$ в узле $(n,m)$ через $f_n^m$. Для простоты вместо вместо $f_m^n$, $f_m^{n+1}$ и $f_{m \pm q_s}^n$ будем писать $f$, $\hat{f}$ и $f^{\pm q_s}$ соотвественно.

Введем обозначения для среднего значения величин сеточной функции в двух соседних узлах, а так же для разностных операторов:
\begin{equation}
	\begin{array}{lc}
		f_{avg_s} = \frac{f_m^n + f^n_{m + 1_s}}{2} \\
		f_{\overline{avg}_s} =  \frac{ f^n_{m - 1_s} + f_m^n}{2} \\
		f_t = \frac{f_m^{n+1} - f_m^n}{\tau}\\
		f_{x_s} = \frac{f_{m+1_s}^n - f^n_m}{h_s}\\
		f_{\bar{x}_s} = \frac{f_m^n - f^n_{m-1_s}}{h_s}\\
		f_{x_s\bar{x}_s} = \frac{f^n_{m-1_s} - 2f_m^n + f^n_{m+1_s}}{h_s^2}\\
		f_{\mathring{x}_s} = \frac{f_{m+1_s}^n - f^n_{m-1_s}}{2h_s}\\
		f_{\mathring{x}_s \mathring{x}_q} = \frac{f_{m+1_s}^{n \,\, +1_q} + f_{m-1_s}^{n \,\, +1_q} - f_{m+1_s}^{n \,\, -1_q} - f_{m-1_s}^{n \,\, -1_q}}{4h_sh_q}\\
	\end{array}
\end{equation}

Введем нормы, для определения невязок при выполнении заданий практикума. Обозначим $int\omega_h = \omega_h \setminus \gamma_h$. Тогда для произвольной сеточной функции:
$$||v||_C = \max_{x\in \omega_h}$$
$$||v||_L = \sqrt{\Pi_h \left(\sum_{x\in int\omega_h} v^2(x) + \frac{1}{2}\sum_{x\in \gamma_h} v^2(x)\right) }$$
$$||v||_W = \sqrt{||v||_L^2 + \Pi_h\sum_{i = 1}^{d}\sum_{x\in int\omega_h \cup \gamma_{h,i}^-} v^2(x)}$$
где $\Pi_h = h_1\cdot \, \dots \, \cdot h_d$

