\documentclass[12pt]{extarticle}
\usepackage[utf8]{inputenc}
\usepackage[english,russian]{babel}
\usepackage{vmargin}
\usepackage{indentfirst}
\usepackage[T2A]{fontenc}
\usepackage{graphics}
\usepackage{amsthm}
\usepackage{amsbsy}
\usepackage{amsmath}
\usepackage{amssymb}
\usepackage{amsfonts}
\usepackage{mathtext}
\usepackage[pdftex,a4paper,colorlinks,linkcolor=blue,citecolor=blue]{hyperref}	

\usepackage{mathtext}
\usepackage{mathenv}
\usepackage[pdftex]{graphicx}
\usepackage{array}
\usepackage{graphicx,xcolor}
\usepackage{xcolor}
\usepackage{float}
\usepackage{longtable}

\usepackage{tikz}
\usepackage{verbatim}

\usepackage{hyperref}
\usepackage{cite,enumerate,float,indentfirst}
\linespread{1.5}

\usepackage{amscd}
\usepackage{amsthm}
\usepackage{amsbsy}
\usepackage{amsmath}
\usepackage{amssymb}
\usepackage{amsfonts}
\usepackage{mathtext}
\usepackage{float}

\usepackage{array}
\usepackage{longtable}


\newtheorem{example}{Пример}[section]
\newtheorem{definition}{Определение}[section]
\newtheorem{proposition}{Предложение}
\newtheorem{theorem}{Теорема}[section]
\newtheorem{corollary}{Следствие}[section]
\newtheorem{lemma}{Лемма}

\DeclareMathOperator{\tr}{tr}

\newcommand{\dif}[2]{\frac{\partial #1}{\partial #2}}
\DeclareMathOperator{\Log}{Log}
\begin{document}

\renewcommand{\contentsname}{Содержание}
\newenvironment{changemargin}[1]{
  \begin{list}{}{
    \setlength{\voffset}{#1}
  }
  \item[]}{\end{list}}


\begin{titlepage}
	\begin{center}
		
		Федеральное государственное бюджетное образовательное учреждение высшего образования 
		<<Московский Государственный Университет им.\,М.\,В.\,Ломоносова>>\\
		
		Механико-математический факультет
		
		\begin{figure}[!htp]
			\begin{center}
				{\includegraphics[width=20mm]{mmlogo.png}}
			\end{center}
		\end{figure}
		
		\vspace{3cm}
		
		Реферат по истории\\
		{\bf Августовский путч 1991 года. ГКЧП: события и версии.}
		
		\vspace{8cm}
		\begin{flushright}
			{\bfРаботу выполнил:}\\
			студент 4 курса Сибгатуллин Артур Петрович\\[0.5cm]
		\end{flushright}
		\vspace{1cm}
		
		\normalsize Москва, 2022
	\end{center}
\end{titlepage}


\tableofcontents

\newpage
\section{Введение}
Сегодняшнее положение России резко отличается от ее положения еще 20-25 лет назад. Страна, которая живет в условиях всемирной изоляции на протяжении полувека, должна была рано или поздно закончить свое существование, к восьмидесятым годам прошлого столетия это было ясно очень многим. Оставался лишь вопрос, как и кем будет совершен этот процесс упадка мировой сверхдержавы, к которой относился СССР. Партийные руководители коммунистической партии Советского Союза к тому времени уже исчерпали все потенциалы своей деятельности. В стране сложилось такое положение, которое историки назвали “стагнацией”. В этих условиях необходимо было кординально менять принципы государственного управления, нужны были новые идеи, новые люди, с несколько иными установками и планами государственного развития.

Такие люди появились, в 1985 году к руководству коммунистической партией СССР, а одновременно и к руководству страной пришел молодой энергичный Михаил Сергеевич Горбачев. Он стал проводить принципиально новые идеи в жизнь государства, была объявлена перестройка, совершенно новое явление в истории советского, да и российского государства в целом. Именно он своими действиями сломил “железный занавес”, который был выстроен западными державами сразу после второй мировой войны и объявлением негласной, скрытой, но очень жестокой “холодной войны”. На первый взгляд его действия должны были лишь укрепить мировой авторитет СССР, улучшить внутреннее положение, благоприятно сказаться на экономике, политике и культуре государства. Однако этого не произошло. Результаты его действий приняли совершенно неожиданный оборот, этого не ожидал никто, хотя абсолютно все понимали что это необходимо и должно было произойти.

В данной работе попытаемся раскрыть причины событий 19-21 августа 1991 года в СССР. Посмотреть на это под другим ракурсом, оценить действия тех людей, которые являлись их участниками. Это сделано для того, чтобы возможно более полно понять происходящие сегодня события, которые являются неизбежным следствием тех дней.

В ходе работы будет сделана попытка рассмотреть причины данных событий, изучить их корни, рассмотреть возможные пути выхода из того кризисного положения в котором оказалась страна на пороге, да и в течении всего 1991 года.


\section{Августовский путч: случайность или закономерность?}
Для СССР специфична территориально-отраслевая структура управления государством. Отрасли народного хозяйства представлены в регионах своими подразделениями, предприятиями и организациями. Управление регионами осуществляется местными органами власти, до недавнего времени только советской, в последнее время заменяемое муниципальной. Координация деятельности отраслевых предприятий и организаций, расположенных на конкретных территориях, осуществлялась аппаратом КПСС. Шесть лет перестройки шел процесс отчуждения отраслевой собственности сначала в пользу советской, а потом и в пользу муниципальной власти. При этом координирующая роль аппарата КПСС постепенно уменьшалась и параллельно увеличивалась дисбалансировка народного хозяйства, приведшая, в конечном счете, к топливно-энергетическому, транспортному и продовольственному кризисам.

Роль Горбачева и Кабинета министров заключалась в том, чтобы по возможности сбалансировать интересы союзных ведомств и отраслей (включая армию, КГБ и МВД) с интересами автономизирующихся территорий. С большим или меньшим успехом Горбачеву и его окружению удавалось перевести конфликты между регионами и союзными ведомствами в относительно мирную форму. Периодические изменения его позиции во многом объяснялись тем, что он был вынужден балансировать между набирающими силу республиканскими органами власти и теряющими свою собственность и власть министерствами и ведомствами.

Ноябрьские события 1990 года свидетельствовали о том, что он был вынужден подчиниться в какой-то степени прямым угрозам со стороны ВПК и армии, стремящийся сохранить статус-кво в своих отношениях с республиками. Изменение его позиции весной и летом 1991 года и достижение соглашения о подписании Союзного договора означали изменение статус-кво в отношениях между республиками и союзными министерствами и ведомствами, причем такого масштаба, что практически устраняли министерства, высший генералитет СА, КГБ и МВД из системы власти. Инстинкт самосохранения заставил членов ГКЧП (\emph{Государственный комитет по чрезвычайному положению}) перейти к активным действиям. Однако эти действия в значительной степени носили рефлекторный характер и по своей логике были сугубо “социалистическими” и не учитывали тех изменений, которые произошли в стране за годы перестройки.

Помимо членов ГКЧП. Несомненно, существовал своего рода оперативный штаб, который осуществлял планирование и координировал практические действия. По мнению многих историков и современников этих событий можно выдвинуть два предположения о локализации штаба:
\begin{itemize}
	\item По первой версии в роли штаба выступало ближайшее окружение ГКЧП - их помощники, секретари и прочие доверенные лица.
	\item Cогласно второму предположению, столь же достоверному, как и первое, в роли штаба могла выступать группа дилетантов, уже длительно предлагавшая услуги такого рода Павлову, руководителям репрессивных ведомств его кабинета и ЦК КПСС \cite{1}
\end{itemize}

Очевидно, что заговорщики считали, что уровень управляемости репрессивными органами и армией со стороны аппарата ЦК КПСС если и снизился, то не сильно. План мероприятий по введению режима чрезвычайного положения существует в любом государстве, и включает в себя разнообразные меры - от превентивных арестов до отключения систем связи и введения особого режима передвижения. Соответствующие инструкции находятся в сейфах специальных отделов МВД, КГБ и подразделений Генерального штаба СА, а также региональных органов власти \cite{2}

В ходе перестройки система введения чрезвычайного положения была нарушена. Вполне возможно, что новые руководители исполкомов местных советов, которые по положению о “ временных совещаниях” являются их руководителями. По режимным соображениям не были ознакомлены с соответствующими инструкциями. Организаторы переворота рассчитывали на то, что управляемость системы и сохранившие жестокие вертикальные связи позволят им без особого напряжения ввести режим ЧП, несмотря на те изменения, которые были произведены в структуре местных органов власти в ходе перестройки. Особым обстоятельством, затруднившим введение режима ЧП и сыгравшим самую неудачную роль в попытке государственного переворота, было то, что режим ЧП вводился не в том объеме и не в той последовательности, которые предполагались действующими инструкциями. Аппарат репрессивных органов и вооруженных сил, как и все подразделения государственного устройства по самому принципу своей организации не могут действовать не по инструкциям, т. к. не существует иных механизмов согласования действий.

\section {Причины переворота: суждения и мнения}
Необходимость введения режима чрезвычайного положения из-за фактического распада систем жизнеобеспечения, катастрофического дефицита энергоносителей и отказа сельскохозяйственных предприятий и местных органов власти обеспечивать выполнение плана поставок продовольствия в госрезервы, судя по многим сообщениям, многократно обсуждались в окружении Горбачева и подчиненных ему органов власти. В интервью Лукьянова группе депутатов ВС СССР, данном им на второй день переворота, говориться о том, что чрезвычайное положение Горбачев предполагал ввести после подписания Союзного договора, на основании соглашения “9+1”.

Однако подписание Союзного договора автоматически устраняло руководителей ГКЧП от власти и, по мнению теперь уже бывших руководителей базовых отраслей народного хозяйства, делало невозможным стабилизацию экономики и поддержание систем жизнеобеспечения в работоспособном состоянии ввиду предстоящей зимы. Подписание Союзного договора ускорило бы распад единой финансовой системы и экономического пространства СССР в целом.

Из событий, которые, несомненно, стимулировали попытку государственного переворота необходимо отметить следующие:
\begin{enumerate}
\item Невыполнение планов госпоставок зерна нового урожая и замыкание экономических пространств зернопроизводящими союзными республиками.

\item Сокращение на 50\% оборонных заказов и грядущий паралич оборонной промышленности, социальные последствия необдуманной конверсии оборонных отраслей.

\item Лавинообразная коммерциализация отношений между руководителями крупных предприятий и подотраслей народного хозяйства, ведущая к потере плановых компонент управления ими.

\item Указ Ельцина о департизации, устраняющей аппарат КПСС из сферы принятия каких-либо решений по управлению экономикой и социальной жизнью.

\item Необходимость введения чрезвычайного положения и после неудачи переворота сохраняется. Вполне вероятно, что оно будет введено, но в других формах и с другими руководителями.
\end{enumerate}

\section{Рассуждения о причинах неудачи переворота}
Основной причиной неудачи переворота, по мнению многих историков, стала несогласованность действий членов ГКЧП и тех органов, которые должны были бы беспрекословно исполнять их распоряжения. Высшие военачальники - командующие военными округами и родами войск не были заранее проинформированы о предстоящих действиях, не были подготовлены к жестким решениями к необходимости применить оружие на территории РСФСР, Москвы и Санкт-Петербурга.

Отсутствовала система согласования действий не только между армией, КГБ и МВД, но и между частями разных родов армии. Специальные подразделения войск КГБ не смогли (или не захотели) выполнен поставленные перед ними задачи по нейтрализации противников переворота. Жесткая управляемость армией, КГБ и МВД со стороны отдела административных органов ЦК КПСС оказалась мифом.

Особым обстоятельством, затруднившим управление войсками КГБ и МВД, было то, что эти части были переданы репрессивным ведомствам в ходе сокращения СА в 1990 году и, очевидно, ни офицерский состав, ни солдаты не были психологически подготовлены к действиям в столицах страны против русскоязычного населения. В составе ГКЧП не оказалось ни одного человека, который соответствовал бы роли руководителя переворота. Более того, большинство членов ГКЧП имели устойчивый негативный имидж у большей части населения страны.

Реакция мирового сообщества на переворот оказалась столь быстрой и жесткой, что не оставила его руководителям пространства для маневра. Кроме того, обращение Янаева к Ясиру Арафату вполне однозначно определило отношение ГКЧП к террористическим организациям и режимам, сделав невозможными какие-либо позитивные контакты между членами ГКЧП и руководителями ведущих стран мирового сообщества.

Разногласия среди членов ГКЧП относительно нейтрализации Президента РСФСР Ельцина и российского парламента дали Ельцину возможность обострить ситуацию до уровня вооруженного конфликта. К которому руководители ГКЧП были, очевидно, не готовы. Горизонтальные связи между руководителями местных органов власти, с одной стороны, и командующими дислоцированных на их территориях войск, руководителями территориальных органов КГБ и МВД, с другой оказались несколько сильнее, чем вертикальные связи воинской подчиненности. Поэтому выжидательная позиция руководителей союзных республик и областей РСФСР по отношению к борьбе между властями РСФСР и ГКЧП резко ограничила возможность маневрирования и применения силовых методов руководителями переворота.

\section{Последствия путча в оценках историков и современников}
Ельцин говорил о том, что необходимо признать, что о путче так, и не сказано всей правды. Предпринимаются попытки обелить участников путча. Предать забвению их зловещие замыслы, по его мнению, тот отпор, который дали путчистам в августе 1991года, был, по сути, первой революцией в нашей истории, произошедшей не против закона, а в защиту его: уже в ноябре указом Ельцина деятельность КПСС была запрещена. Следствием этого стала ликвидация КПСС как единой общесоюзной партии.

Не подлежит сомнению, что <<Центр>> в его прежнем статусе прекратил свое существование. Теперь объем и содержание власти Президента СССР и тех органов, которые заменят Кабинет министров, Совет безопасности, Совет федерации, Верховный совет СССР, будут определяться президентами тех республик, которые подпишут вариант нового Союзного договора. Президент РСФСР получил в результате переворота особый статус. Ставивший его в исключительное положение “первого среди равных”. Республики не собиравшиеся подписывать Союзный договор выйдут из состава нового Союза, и вполне возможно, будут стремиться оформиться в некоторое подобие “санитарного кордона” между Европейским сообществом, в которое так или иначе будут интегрированы пограничные бывшие социалистические страны, с одной стороны, и государствами, объединенными новым Союзным договором, - с другой.

Бывшие автономии РСФСР, претендующие на статус союзных республик, будут резко ограничены в своих возможностях, из-за усиления власти и авторитета Президента РСФСР.

Августовский кризис привел к разрушению институтов управления, стержнем которых были КПСС и КГБ. В результате чего Россию поразил глубочайший кризис управления, от которого страна не могла оправиться долгие годы. Оборвав эволюционность политического развития, августовский путч способствовал усилению поляризации политических сил, что вылилось, в конечном счете, в кровавую драму октября 1993 года.

Филатов указывал на положительные результаты после августовского развития. Главный из них - принятие 12 декабря 1993 года новой Конституции Российской Федерации, что дало возможность введения политического процесса в нормальное русло и развитие гражданского общества. В начале апреля 1994 года Конституция была подкреплена Договором об общественном согласии, основная цель которого-создание механизмов и обработка процедур согласования интересов различных социальных сил и групп. Особенно в условиях дефицита законодательных актов. Стала возможной экономическая реформа и, хотя мы были вынуждены пройти через шоковый этап, уже реальны ее следующие шаги.

В августе 1990 года вышла книга “Седьмой секретарь”\cite{3} в ней задавался вопрос: почему Горбачев затеял перестройку? Сегодня, по мнению многих историков, исследователей этого периода, становится ясно, чтобы оттянуть как можно дальше гибель советской системы. В результате действий М. Горбачева система развалилась гораздо раньше, чем кто-либо это предвидел. Он этого не хотел. Он этому изо всех сил препятствовал. Как считает Геллер, “путч” являлся ни чем иным, кроме как хорошо разыгранным спектаклем, поставленным перед всем миром. Это объясняется тем, что главные роли в “путче” сыграли люди, каждого из которых тщательно выбирал и поставил на занимаемое место сам Горбачев. Это были его ближайшие соратники. Августовский “путч”, хотя Горбачев представляет его как предательство близких, носил иной характер. До последней минуты “заговорщики” убеждали Горбачева возглавить Комитет, начать действовать решительно, чтобы навести порядок в стране. 18 августа в Форос прилетела делегация от будущих “путчистов” упрашивать президента объявить чрезвычайное положение. После своего ареста “путчисты” утверждали, что Горбачев знал об их намерениях и уехал в Форос с напутствием: делайте, как хотите. Это наверное следует понимать: удастся - я буду с вами, не удастся - отвечайте вы.

По мнению Соколова\cite{4} Августовский путч стал именно тем событием, после которого центробежные силы вышли на качественно новый уровень. Начался распад СССР. Сразу после прихода к власти ГКЧП, 20 августа 1991 года парламент Эстонии принял постановление о государственной независимости республики. Аналогичный документ на следующий день принял парламент Латвии. 24 августа, “исходя из смертельной опасности, нависшей над Украиной”, Верховный Совет республики объявил ее независимым государством. До конца августа такие же документы были приняты в Белоруссии, Молдове, Азербайджане, Киргизии и Узбекистане.
\section {Заключение}

Подводя итоги нашего исследования, мы можем сделать следующие выводы:
\begin{enumerate}
	\item События, которые произошли 19 - 21 августа 1991 года были неизбежны. К этому СССР шел все последние годы. Они были необходимы для выхода на новый, качественно новый уровень развития государства.
	\item Единого мнения на произошедшие события выработано не было, до сих пор историки спорят, как их называть. Одни считают эти три дня ни чем иным как переворотом в истории государства, другие рассматривают его просто как путч, а третьи вообще намекают на некую революцию, которая произошла почти мирным путем. Мы не будем останавливаться на каком-либо одном из мнений, ибо на наш взгляд прошло еще слишком мало времени, чтобы давать этим событиям какую-либо строгую классификацию.
\end{enumerate}

\newpage
\begin{thebibliography}{99}
\bibitem{1} 
\textsc{Дмитренко} История России ХХ век. Москва, АСТ, 1998

\bibitem{2} 
\textsc{} Уроки августа 1991 года. Народ и власть: Научно-практическая конференция. Москва, 1994


\bibitem{3} 
\textsc{Геллер М} Утопия у власти. История Советского союза с 1917 года до наших дней. Книга 3. Седьмой секретарь. Блеск и нищета Горбачева. Москва, МИК,1995

\bibitem{4}
\textsc{Соколов А.К.} Курс советской истории 1941-1991. Москва, Высшая школа, 1999




\end{thebibliography}
\end{document}