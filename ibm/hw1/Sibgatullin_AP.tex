\documentclass[a4paper, fontsize=14pt]{article}
\usepackage[utf8]{inputenc}
\usepackage[T2A]{fontenc}
\usepackage[left=2cm,right=1.5cm,
    top=2cm,bottom=2cm,bindingoffset=0cm]{geometry}
\usepackage[english, russian]{babel}
% \usepackage[left=2.5cm, right=1.5cm, vmargin=2.5cm]{geometry}
\usepackage{indentfirst}
\usepackage{scrextend}
\usepackage{amsfonts}
\usepackage{listings}
\usepackage{amsmath}
\usepackage{amssymb}
\usepackage{listings}




\renewcommand*\contentsname{Содержание}

\title{\textbf{Отчет по заданию \\"Решение системы линейных уравнений методом вращений"}}
\author{Сибгатуллин Артур, 310 учебная группа}
\date{27 сентября, 2020}



\newcounter{subsubsubsection}[subsubsection]
\renewcommand\thesubsubsubsection{\thesubsubsection.\arabic{subsubsubsection}}

\begin{document}

        \maketitle

    \newpage
        \tableofcontents

        \newpage
	\section{Введение}
		 В курсе предмета "Работа на ЭВМ и программирование" перед нами была поставлена задача: решить систему линейных уравнений методом \textbf{"Вращений (Гивенса)"}.
		 Пусть система уравнений $Ax = b $ задана следующими элементами:
		 \begin{equation*}
			 	A = \left(
			 	\begin{array}{cccc}
			 		a_{11} & a_{12} & \ldots & a_{1n}\\
			 		a_{21} & a_{22} & \ldots & a_{2n}\\
			 		\vdots & \vdots & \ddots & \vdots\\
			 		a_{n1} & a_{n2} & \ldots & a_{nn}
			 	\end{array}
			 	\right), 
			 	x = \left(
			 	\begin{array}{cccc}
			 		x_{11} \\
			 		x_{21} \\
			 		\vdots \\
			 		x_{n1} 
			 	\end{array}
			 	\right),	
			 	b = \left(
			 	\begin{array}{cccc}
			 		b_{11} \\
			 		b_{21} \\
			 		\vdots \\
			 		b_{n1} 
			 	\end{array}
			 	\right)
	 	\end{equation*}
 	
 	
		Где матрица $A-$невырожденная, $x-$вектор столбец искомых неизвестных, $b-$вектор столбец свободных членов. Необходимо найти $x$.
		
		\subsection{Алгоритм метода вращений}
		
		\textbf{Лемма 1}: $ \, \forall \, \, r = (x_1,x_2)^T \in \mathbb{R}^2,$ где $r\neq0$ существует матрица $T$:
		 $$T = \begin{pmatrix}
			\cos(\varphi)& \ -sin(\varphi)\\
			\sin(\varphi)& \cos(\varphi)
		\end{pmatrix} \text{и} \, Tr = \Vert r \Vert(1,0)^T \, \text{где} \Vert r \Vert = \sqrt{x_1^2 + x_2^2} $$
		$\Box$
		Просто предъявим такую матрицу: $cos(\varphi) = \frac{x_1}{\sqrt{x_1^2 + x_2^2}}$, $sin(\varphi) = -\frac{x_2}{\sqrt{x_1^2 + x_2^2}}\blacksquare$
		
		$\\$
		Как можно заметить матрица $T$ это матрица вращения на угол $\varphi$ в $\mathbb{R}$ относительно оси $x_1$, поэтому метод называется методом вращения.
		\\Таким же образом мы можем привести вектор-столбцы любого размера к виду $(\Vert x_1\Vert,\dots, 0  )^T$ , например $(x_1, x_2, x_3 )^T$ приводим к виду $(\Vert x_1\Vert, 0, 0 )^T$, сначала преобразовав к виду $(\Vert x_1\Vert, 0, x_3  )^T$, а затем к $(\Vert x_1\Vert, 0, 0  )^T$. Формальное доказательство этого факта приведено в учебнике К.Ю.Богачева <<Практикум на ЭВМ. Методы решения линейных систем и нахождения собственных значений>>, стр.43 Лемма 3.
		\\
		\\Сам алгоритм заключается в приведении матрицы $A$ к верхнетреугольному виду следующим образом. Обозначим первый столбец матрицы $A$ как
		\begin{equation*}
			a^1 = 
			\left( \begin{array}{cccc}
				a_{1,1} \\
				a_{1,1} \\
				\vdots \\
				a_{n,1} 
				\end{array}
			\right)
		\end{equation*}
		 Тогда$\,\exists$ набор матриц $\,T_{1,2},\dots ,T_{1,n}$, таких что $T_{1,2}*\dots*T_{1,n}*a^1 = \Vert a^1\Vert *e_1$,  ($e_1,\dots e_n $) - стандартный базис в $\,\mathbb{R}^n$. Умножим систему $Ax=b$ на $T_{1,2}*\dots*T_{1,n}$ слева. Получим новую систему $A^{(1)}x=b^{(1)}$ эквивалентную исходной,
		 где
		 \begin{equation*}
			A^{(1)} = T_{1,2}*\dots*T_{1,n}*A =
			\left(
			\begin{array}{cccc}
			\Vert a_1\Vert & c_{1,2} & \ldots & c_{1,n}\\
			0 & a_{2,2}^{(1)} & \ldots & a_{2,n}^{(1)}\\
			\vdots & \vdots & \ddots & \vdots\\
			0 & a_{n,2}^{(1)} & \ldots & a_{n,n}^{(1)}
			\end{array}
			\right),
			b^{(1)} = T_{1,2}*\dots*T_{1,n}* \left(
			\begin{array}{cccc}
				b_{1,1} \\
				b_{2,1} \\
				\vdots \\
				b_{n,1} 
			\end{array}
			\right)
		\end{equation*}
		
		Далее применим процесс для подматрицы $(a_{i,j}^{(1)})_{i,j = 2\dots n}\,(1)$. После применения данного процесса ко всем подматрицам вида$\,(1)$ мы получим матрицу 
		\begin{equation*}
			R =
			\left(
			\begin{array}{ccccccc}
				\Vert a_1\Vert & c_{1,2} & c_{1,3} & \ldots & c_{1,n-2} & c_{1,n-1} & c_{1,n}\\
				 					 & \Vert a_1^{(1)}\Vert & c_{2,3} & \ldots & c_{2,n-2} & c_{2,n-1} & c_{2,n}\\
									 		&  & \Vert a_1^{(2)}\Vert & \ldots & c_{3,n-2} & c_{3,n-1} & c_{3,n}\\
									 		&  &  &\ddots & \vdots & \vdots & \vdots\\
											&  &  & & \Vert a_1^{(n-3)}\Vert  & c_{n-2,n-1} & c_{n-2,n}\\
											&  &  & & & \Vert a_1^{(n-2)}\Vert   & c_{n-1,n}\\
											&  &  & & & & a_{n,n}^{(n-1)}\\
			\end{array}
			\right),
			\end{equation*}
		и столбец свободных коэффицентов 
		$y = b^{(n-1)} = \prod_{i = n-1}^{n-1}\prod_{j=n}^{i+1}T_{ij}b$, где $\prod_{j=n}^{i+1}$ означает, что сомножители берутся в порядке $n,\dots,i+1$
		\\Система $Rx=y$, где $R$ - верхнетреугольная матрица, решается обратным ходом метода Гаусса.

	
	 	
 	
	\newpage
 	\section{Точечный метод вращения}
	 	\subsection{Реализация(Псевдокод)}
	 	
	 	\lstinline$for i=1 to n do $
	 	\\
	 	\lstinline$	for j = i+1 to n do$
	 	\\
	 	\lstinline$		cos(phi) = A[i][i] / norm(j) $
	 	\\
	 	\lstinline$		sin(phi) = -A[j][i] / norm(j) $
	 	\\
		\lstinline$		for k = i to n do$
		\\
		\lstinline$			A[i][k] = A[k][j] * cos(phi) - sin(phi) * A[j][k]$
		\\
		\lstinline$			A[j][k] = A[k][j] * cos(phi) + sin(phi) * A[j][k]$
		\\
		\lstinline$		end for$
		\\
		\lstinline$		b[i] = b[i] * cos(phi) - sin(phi) * b[j]$
		\\
		\lstinline$		b[j] = b[i] * cos(phi) + sin(phi) * b[j]$
		\\	
		\lstinline$	if A[i][i] < EPS $
		\\	
		\lstinline$		return ERROR:System is incosistent$
		\\	
		\lstinline$	end for$
		\\
		\lstinline$end for$
		\\
		\lstinline$Back-Subtitution_procedure()$

 	 	\subsection{Применимость}
 	 	Матрица $A$ должна быть не вырожденной.
 	 	
 	    \subsection{Обратный ход метода Гаусса}
 	    Обратный ход метода Гаусса заключается в нахождении $x$ из верхнетреугольной матрицы, следующим образом:
 	    $$x_n = y_n, x_i = y_i - \sum_{j = i+1}^{n} c_{i,j}x_j \,\text{где} \, i=n-1,\dots,1$$
 	    
     	\subsection{Оценка числа арифметических операций}
     	Посчитаем сложность алгоритма для $k$-го шага а затем просуммируем по $k=1,\dots,n-1$
     	\begin{itemize}
     		\item Для вычисления матриц $T_{k,k+1},\dots,T_{k,n}$ необходимо n-k аддитивных, $4*(n-k)$ мультипликативных и $n-k$ операций извлечения корня.
		    \item Для вычисления компонент $k,\dots,n$ $k$-го столбца матрицы $A^{(k)}$, которые мы считаем по формуле $\Vert a_1^{(k-1)} \Vert e_1^{(n-k+1)}$ требуется $(n-k+1)$ мультипликативных операций, $(n-k)$ аддитивных операций и 1 операция извлечения корня 
		    \item Так как в алгоритме мы применяем умножение $n-k$ матриц вращения на подматрицу$\,(a_{i,j}^{(k-1)})_{i = 1,\dots,n,j = k+1\dots n}$, то на это требуется $4(n-k)^2$ мультипликативных и $2(n-k)^2$ аддитивных операций
		   	\item Для вычисления столбца $y$ требуется $4(n-k)$ мультипликативных и $n-k$ аддитивных операций
		   	\item Для вычисления обратного хода метода Гаусса требуется $\frac{n(n-1)}{n}$ аддитивных и $\frac{n(n-1)}{n}$ операций
     	\end{itemize}
     \textbf{Итого:}
      \begin{itemize}
     	\item Аддитивных операций $\sum_{k=1}^{n-1}(4(n-k)^2 + 9(n-k) + 1) = \frac{4n^3}{3} + O(n^2)$ при $n\to\inf$
     	\item Мультипликативных операций $\sum_{k=1}^{n-1}(2(n-k)^2 + 3(n-k)) = \frac{2n^3}{3} + O(n^2)$ при $n\to\inf$
     	\item Операций извлечения корня $\sum_{k=1}^{n-1}(n-k+1) =  O(n^2)$ при $n\to\inf$
     \end{itemize}
 	 Как можно заметить, метод вращений требует в 2 раза больше мультипликативных и в 4 раза больше аддитивных операций, чем метод Гаусса.
     	
 		\subsection{Организация хранения в памяти}
 		Матрица $A$ хранится в памяти в виде массива размером $n^2$, решение $x$ и столбец свободных членов $b$ хранятся в виде массивов длины $n$. Таким потребление памяти $n^2 + 2n + O(1)$. Так как совершается $O(n^3)\,$ операций с $O(n^2)$ элементами, то возникает необходимость в кэшировании часто используемых данных, для оптимизации работы с памятью.
 		 
	\newpage
	\section{Блочный метод вращения}
		Представим нашу матрицу A в блочном виде:
		\begin{equation*}
			A = \left(
			\begin{array}{cccc}
				A_{11} & A_{12} & \ldots & A_{1k}\\
				A_{21} & A_{22} & \ldots & A_{2k}\\
				\vdots & \vdots & \ddots & \vdots\\
				A_{k1} & A_{k2} & \ldots & A_{kk}
			\end{array}
			\right)
		\end{equation*}
	Где размер каждого блока $m$, тогда без ограничения общности $k = n / m$ - нацело.
		\\На первом шаге алгоритма преобразуем блок $A_{11}$ к верхнетреугольному виду с помощью обыкновенного метода вращений. Как побочный продукт алгоритма у нас останутся вычисленными матрицы поворота: $T_{11} = \prod_{i = n-1}^{n-1}\prod_{j=n}^{i+1}T_{ij}$ . Применим их к строке блоков $A_{12}\dots A_{1k}$ и к блоку $B_{1}$ - вектор столбец $B$ в блочном виде. Получим матрицу вида:
				\begin{equation*}
			A = \left(
			\begin{array}{cccc}
				A_{11}^{(1)} & A_{12}^{(1)} & \ldots & A_{1k}^{(1)}\\
				A_{21} & A_{22} & \ldots & A_{2k}\\
				\vdots & \vdots & \ddots & \vdots\\
				A_{k1} & A_{k2} & \ldots & A_{kk}
			\end{array}
			\right)
		\end{equation*}
		Затем обнуляем столбец блоков $A_{21}\dots A_{k1}$, матрицами вращений, строящихся из диагональных элементов $a_{ii} \in  A_{11}^{(1)}$ и элементов из столбца $a_{2i}, \dots a_{ji} \in A_{21}^{(1)}$. Эти же матрицы поворота испольуем для преобразования строк блоков $A_{22},\dots , A_{2k}, \dots  ,A_{k2}, \dots , A_{kk}$.
		Имеем:
		\begin{equation*}
			A = \left(
			\begin{array}{cccc}
				A_{11}^{(1,2)} & A_{12}^{(1,2)} & \ldots & A_{1k}^{(1,2)}\\
				0 & A_{22}^{(2)} & \ldots & A_{2k}^{(2)}\\
				\vdots & \vdots & \ddots & \vdots\\
				0 & A_{k2}^{(2)} & \ldots & A_{kk}^{(2)}
			\end{array}
			\right)
		\end{equation*}
		Теперь применяем тот же алгоритм для подматрицы c начальным элементом $A_{22}^{(2)}$
		
		В конечном итоге получаем верхнетреугольную матрицу, которую можно разрешить обратным ходом метода Гаусса.
	 	\subsection{Обратный ход метода Гаусса}
	    Обратный ход метода Гаусса заключается в нахождении $x$ из верхнетреугольной матрицы, следующим образом:
	 	$$x_n = y_n, x_i = y_i - \sum_{j = i+1}^{n} c_{i,j}x_j \,\text{где} \, i=n-1,\dots,1$$
	 	
		\subsection{Оценка числа арифметических операций}
		$m$ - размер блока, $k$ - количество блоков в строке (столбце). 
		\begin{itemize}
			\item Для приведения блоков $A_{ii}$ к верхнетреугольному виду необходимо $2m^2n$ арифметических операций.
			\item Для умножения строки блоков $A_{i2}\dots A_{ik}$ на матрицы поворота необходимо $\frac{3nm(\frac{n}{m}-2)(m-1)}{2}$ арифметических операций.
			\item Для обнуления блока лежащим под диагональным необходимо $3(m+1)(4 + m^2 +m)$ операций.
			\\Всего таких блоков лежащих под диагональными $\frac{n(m+n)}{2m^2}$.
			\\ Итого: $3(m+1)(4 + m^2 +m) * \frac{n(m+n)}{2m^2}$
			\item Для применения матриц вращения из предыдущего пункта ко всей строчке блоков небходимо $(m+1)^3 *\frac{n(m+n)(m+2n)}{m^3}$ операций
			
		\end{itemize}
		Итого:
	
		\begin{itemize}
			\item $m = 1: 2m^3$
			\item $m = n: 2m^3$
		\end{itemize}
		\subsection{Организация хранения в памяти}
		Матрица $A$ хранится в памяти в виде массива размером $n^2$, решение $x$ и столбец свободных членов $b$ хранятся в виде массивов длины $n$. Также необходимо хранить $2\,$ массива размером $2m^2$ в кэш-памяти (для выгрузки блоков над которыми производятся операции и для хранения матриц вращения). Итого потребление памяти алгоритма: $n^2 + n + 4m^2 + O(1)$
		\\
		\\Подпрограмма загрузки (выгрузки) блока матрицы в кэш-память:
		\begin{lstlisting}
void get_block (double *A, int matrix_size, double *Block,
		int block_size, int i_, int j_,
		int dev, int rem_of_dev)
{
	int block_m_size = (i_ < dev ? block_size : rem_of_dev);
	int block_n_size = (j_ < dev ? block_size : rem_of_dev);
	
	double *Block_i = Block;
	double *Ai = A + (i_ * matrix_size + j_) * block_size;
	
	for (int i = 0; i < block_m_size; i++)
	{
		for (int j = 0; j < block_n_size; j++)
		Block_i[j] = Ai[j];
		
		Ai += matrix_size;
		Block_i += block_size;
	}
}
		\end{lstlisting}

		\begin{lstlisting}
void set_block (double *A, int matrix_size, double *Block,
	int block_size, int i_, int j_,
	int dev, int rem_of_dev)
{
	int block_m_size = (i_ < dev ? block_size : rem_of_dev);
	int block_n_size = (j_ < dev ? block_size : rem_of_dev);
	
	double *Block_i = Block;
	double *Ai = A + (i_ * matrix_size + j_) * block_size;
	
	for (int i = 0; i < block_m_size; i++)
	{
		for (int j = 0; j < block_n_size; j++)
		Ai[j] = Block_i[j];
		
		Ai += matrix_size;
		Block_i += block_size;
	}
}
		\end{lstlisting}
		
\end{document}
